\subsection{Расчет расхода охлаждающего воздуха}

Исходные данные для расчета количества охлаждаемого воздуха представлены в табл.~\ref{cool1:cool1_inlet}.
Расчет проведен по методике~\cite{ivanov}.

В качестве материала лопатки принимается сплав ЖС30, выдерживающий при данном уровне температур 250 МПа в течение 10000 ч ~\cite{js_36_properties}.
Данный уровень напряжений заведомо существенно выше напряжений, действующих в короткой двухопорной лопатке, нагруженной
только газодинамическими силами.

 \begin{enumerate}
 	\item Определим число для газа ():
 	\item Определим число для газа:
 	\item Определим средний коэффициент теплоотдачи от газа к лопатке:
 	\item Определим тепловой поток в сопловую лопатку:
 	\item Определим падение температуры в тенке лопатки:
 		(для ЖС30 при )
 	\item Определим температуру внутренней поверхности стенки лопатки:
 	\item Задаваясь рядом значений расходов охлаждающего воздуха, определим зависимость зазора в лопатке от расхода охлаждающего воздуха:
 		где 
 	Результаты расчета расхода охлаждующего воздуха приведены в таблице~\ref{cool1:mass_rate_result}.
		

 \end{enumerate}