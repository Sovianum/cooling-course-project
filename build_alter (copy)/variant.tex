\subsection{Вариантные расчеты}
Для определения оптимальных степеней повышения давления в компрессорах
построим графики зависимости КПД, удельной мощности и расхода через компрессоры от суммарной степени повышения давления в компрессорах.
При этом для наглядности отнесем абсолютные значения рассматриваемых величин к максимальному значению,
достигающемуся на заданном промежутке.

График зависимостей КПД,
мощности и расхода ГТА от суммарной степени повышения давления в компрессорах представлен на рис.~\ref{img:cycle_eta_plot}.
Распределение степеней повышения давления между компрессорами соответствует оптимальному по КПД:

Экстремум по КПД достигается при следующих значения функций:

Экстремум по удельной мощности достигается при следующих значениях функций:

В связи с высокой температурой за камерой сгорания, для изготовления лопаточных венцов турбины высокого давления требуется
использование крайне дорогих материалов и применение интенсивного охлаждения. Поэтому уменьшение количества ступеней
турбины высокого давления является актуальной технико-экономической задачей. Опираясь на данные вариантного расчета,
можно сказать, что применение в данной схеме двуступенчатой турбины высокого давления приведет к недогрузке
обеих ступеней, а, следовательно, к излишним расходам на материалы. В связи с этим в данной работе принимается,
что позволит создать эффективную одноступенчатую турбину высокого давления.

Ниже представлен расчет цикла ГТА при.