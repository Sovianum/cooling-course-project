\section*{РЕФЕРАТ}
\addcontentsline{toc}{section}{РЕФЕРАТ}
В данном проекте была разработана газотурбинная установака, выполненнная
по трехвальной схеме со свободной турбиной.

Цель рабты - разработка газотурбинной установки (ГТУ) мощностью 16 МВт для 
эксплуатации в кацестве привода центробежного нагнеталя на линейных 
компрессорных станциях магистральных газопроводов.

В настоящее время относительная доля установленной мощности компрессорных станций с газотурбиннм приводом в стистеме ОАО "Газпром" составляет свыше 85\%, что обуславливает актуальность разработки экономичных газотурбинных приводов газоперекачивающих агрегатов (ГПА). При этом важной особенностью работы ГТУ в качестве привода ГПА является практически постоянная работа на режимах частичной мощности, с чем связана необходимость анализа свойств установки в широком диапазоне рабочих режимов.

В данной работе был проведен анализ параметров установок на режимах 100-30\% номинальной мощности, а также проведена оптимизация системы охлаждения турбины высокого давления с целью увеличения ресурса установки.

Разработан маршрутный технологический процесс изготовления рабочей лопатки первой ступени турбины высокого давления.

Проведено сравнение стоимости проектного варианта установки и установки аналогичной мощности ГПА-16 <<Ладога>>, а также прямых эксплуатационных расходов.

Выполнен анализ вредных и опасных производственных факторов установки на этапе эксплуатации. Проведен расчет шума двигателя на номинальном режиме работы, а также оценена зона поражения в случае утечки газа со станции.
