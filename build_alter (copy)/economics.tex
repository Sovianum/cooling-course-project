\subsection{Оценка единовременных затрат на прототип}
Привод газоперекачивающего агрегата (ГПА) – сложное изделие с чрезвычайно широкой номенклатурой используемых материалов.
Точный расчет по всей номенклатуре крайне затруднителен, а на этапе эскизного проектирования – невозможен. Основной
вклад в затраты вносят дорогостоящие сплавы для горячей части двигателя (гранулированные ЭП741НП, жаропрочные для
охлаждаемых лопаток ЖС6К, ЖС6У, ЖС32) и легкие титановые сплавы для холодной части (ВТ3, ВТ6, ВТ8, ВТ9, АЛ4). Для
сравнительно ненапряженных температурных условий используются хромникелевые и нержавеющие стали и сплавы.

Для упрощения задачи, зная массу прототипа (6650 кг), закладываю коэффициент использования материала КИМ=0,05 и умножаю
на осредненную стоимость материалов (2500 р/кг). Для того, чтобы учесть затраты на оплату труда рабочих, на сумму затрат
на материалы вводится коэффициент 1,5.

Итого, получается стоимость прототипа:
\subsection{Оценка снижения затрат в связи с доработкой конструкции}
В научно-исследовательской части настоящей выпускной квалификационной работы была усовершенствована конструкция
компрессора высокого давления: повышены напорности ступеней, в результате чего удалось уменьшить число ступеней с 7 до 5.
Оценка выигрыша массы приведена в таблице~\ref{tab:economics-mass-comparison}.
Экономия массы в сравнении с прототипом, кг составила:
Можно оценить снижение исходной массы материала для производства установки с учетом коэффициента использования материала:
Следовательно, снижение затрат на материалы:
Принимая, что масса остальных деталей и узлов двигателя остается такой же, как в прототипе, вычисляем единовременные
затраты на проектируемый двигатель:
\subsection{Оценка затрат на единицу мощности}
Одной из важнейших характеристик привода газоперекачивающего агрегата является мощность. С повышением требований к
параметрам цикла двигателя ужесточаются условия работы его узлов и деталей. Применяются сплавы, легированные
дорогостоящими металлами, гранулированные сплавы, повышается трудоемкость изготовления и сборки ДСЕ. Важно оценивать
затраты на единицу мощности. Оценка затрат на единицу мощности представлена в таблице~\ref{tab:economics-unit-power}.
Анализ данных, представленных в таблице~\ref{tab:economics-unit-power}, показывает, что проектируемый двигатель является более выгодным с точки зрения
затрат на единицу мощности по сравнению с прототипом.

\subsection{Расчет затрат на эксплуатацию}
Заложен полный ресурс 100 тыс.ч. Межремонтный интервал – 25 тыс. ч. Затраты на один ремонт составляют 0,25 от единовременных затрат.
Таким образом получим стоимость одного ремонта двигателя:
Цена газа для промышленных потребителей.
Примерный среднечасовой расход топлива проектируемого двигателя
Можно посчитать среднегодовые затраты на топливо проектируемого двигателя
А также среднегодовые затраты на топливо прототипа
Введем коэффициент эксплуатационных затрат, связывающий среднегодовые затраты на регламентное и
техническое обслуживание и среднегодовые затраты на капитальный ремонт установки. В результате получим среднегодовые
затраты на регламентное и техническое обслуживание установки:

Сравнительный график суммарных затра представлен на рис.~\ref{img:economics-cost}.

Проведенная в исследовательской части дипломного проекта оптимизация цикла установки позволяет снизить единовременные
затраты за счет уменьшения стоимости компрессора высокого давления, а использование высокой температуры в камере сгорания
увеличивает экономичность двигателя в целом.