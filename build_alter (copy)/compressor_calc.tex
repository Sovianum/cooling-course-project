\subsection{Расчет компрессора низкого давления по средней линии}

Для в данной работе детально представлен расчет первой ступени КНД по средней линии тока по методике~\cite{beknev}.
Параметры остальных ступеней КНД и КВД представлены в табличном виде
(КНД - табл.~\ref{tab:lpc-stage-total}, КВД - табл.~\ref{tab:hpc-stage-total}).
Исходные параметры для расчета КНД по средне линии тока представлены в таблице~\ref{midline:compressor_inlet}.


\begin{enumerate}
	\item Определим окружную скорость на периферии рабочих лопаток на входе в ступень:
	\item Определим теоретический напор ступени:
	\item Определим действительную работу сжатия:
	\item Определим адиабатическую работу сжатия:
	\item Определим повышение полной температуры в ступени:
	\item Определим степень повышения полного давления:
	\item Определим полное давление на выходе из ступени:
	\item Определим критическую скорость потока на входе в ступень:
	\item Определим критическую скорость потока на выходе из ступени:
	\item Определим относительный средний радиус на входе в ступень:
	\item Определим безразмерную окружную составляющую абсолютной скорости на входе в ступень:
	\item Определим направление абсолютной скорости на входе в ступень:
	\item Определим величину осевой скорости на входе в ступень:
	\item Определим приведенную скорость на входе в ступень:
	\item Величина функции, соответствующая полученному значению приведенной скорости равна:
	\item Определим кольцевую площадь на входе в ступень:
	\item Определим внешний и внутренний диаметры на входе в ступень:
	\item Определим ширину ступени:
	\item Определим внешний и внутренний диаметры на выходе из ступени:
	\item Определим кольцевую площадь на выходе из ступени:
	\item Определим относительный диаметр втулки на выходе из ступени:
	\item Определим относительный средний радиус на выходе из ступени:
	\item Определим безразмерную окружную составлющую абсолютной скорости на выходе из ступени:
	\item Для определения приведенной скорости на выходе из ступени, численно решим уравнение:
		Получим значение приведенной скорости на выходе:
	\item Определим направление потока в абсолютном движении на выходе из ступени:
	\item Определим безразмерную окружную составляющую абсолютной скорости на выходе из рабочего колеса:
	\item Определим углы потока в относительном движении:
	\item Определим направление потока в абсолютном движении после рабочего колеса:
	\item Определим относительную скорость на среднем радиусе на входе в рабочее колесо:
	\item Определим относительную скорость на среднем радиусе на входе в НА:
\end{enumerate}



% section расчет_компрессора_низкого_давления_по_средней_линии (end)