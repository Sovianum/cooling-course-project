\section*{Заключение}
\addcontentsline{toc}{section}{Заключение}

Спроектирована газотурбинная установка мощностью 16 МВт для использования на линейных
компрессорных станциях магистральных газопроводов.

В научнои исследовательской части работы проведено анализ различных схем установки в широком
диапазоне рабочим мощностей, а также проведена оптимизация системы охлаждения, позволившая при уменьшении
расхода охлаждающего воздуха снизить неравномерность поля температур в сопловом аппарате
турбины высокого давления с 256,7 до 141,4 К без увеличения максимальной температуры металла.

В конструкторской части проекта проведены расчет основых узлов установки (компрессор низкого давления, компрессор высокого давления,
камера сгорания, турбина высокого давления, турбина низкого давления, силовая турбина) и разработана
конструкия узлов и компоновка установки на станции.

В технологическоц части разработа маршрутный технологический процесс изготовления рабочей лопатки турбины высокого давления.

В организационно-экономической части работы посчитана стоимость проктного варианта двигателя и проведено сравнение прямых
эксплуатационных расходов проектного варианта двигателя с установкой ГПА-16 <<Ладога>>.

Выполнен анализ вредных и опасных производственных факторов установки на этапе эксплуатации. Проведен расчет шума двигателя на
номинальном режим, а также определена зона поражения в случае утечки газа со станции.
