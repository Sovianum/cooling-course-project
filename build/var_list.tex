
\subsection{Используемые обозначения}
\begin{itemize}
    \item[] $\alpha$ - коэффициент избытка воздуха;
    \item[] $C_{pв}$, Дж/(кг К) - теплоемкость воздуха;
    \item[] $C_{pг}$, Дж/(кг К) - теплоемкость продуктов сгорания природного газа;
    \item[] $C_{pm}$, Дж/(кг К) - теплоемкость топлива;
    \item[] $c_{m.a}$, м/с - осевая скорость на выходе из турбины;
    \item[] $C_e$, Вт/ч - экономичность ГТД;
    \item[] $\eta_k$ - адиабатический КПД компрессора;
    \item[] $\eta_м$ - механический КПД турбины компрессора;
    \item[] $\eta_{тк}$ - адиабатический КПД турбины компрессора;
    \item[] $\eta_р$ - КПД регенератора;
    \item[] $\eta_m$ - адиабатический КПД свободной турбины;
    \item[] $F_m, м^2$ - площадь на выходе из турбины;
    \item[] $g_m$ - удельный расход топлива;
    \item[] $g_т$ - удельный расход газа через турбину;
    \item[] $g_{ут}$ - удельный расход утечек;
    \item[] $g_{охл}$ - удельный расход воздуха на охлаждение;
    \item[] $G_m$, кг/с - расход топлива;
    \item[] $k_г$ - показатель адиабаты продуктов сгорания природного газа;
    \item[] $k_в$ - показатель адиабаты воздуха;
    \item[] $l_0$, кг - масса воздуха, необходимая для сжигания 1 кг топлива;
    \item[] $l_m$, м - длина лопатки на выходе из турбиы;
    \item[] $L_к$, Вт/кг - удельная работа компрессора;
    \item[] $L_{тк}$, Вт/кг - удельная работа турбины компрессора;
    \item[] $L_т$, Вт/кг - удельная работа силовой турбины;
    \item[] $n_т$, об/мин - частота вращения ротора турбины компрессора;
    \item[] $N_{e уд}$, Вт/кг - удельная мощность ГТД;
    \item[] $N_e$, Вт - мощность ГТД;
    \item[] $p_a$, Па - атмосферное давление;
    \item[] $p_ф$, Па - давление за входным фильтром;
    \item[] $p_вх$, Па - давление перед помпрессором;
    \item[] $p_к$, Па - давление за компрессором;
    \item[] $p_{р.х.}$, Па - давление за регенератором по холодной стороне;
    \item[] $p_г$, Па - давление перед турбиной компрессора;
    \item[] $p_{тк}$, Па - давление после турбиные компрессора;
    \item[] $p_с$, Па - давление перед силовой турбиной;
    \item[] $p_т$, Па - давление за силовой турбиной;
    \item[] $p_{р.г.}$, Па - давление за регенератором по горячей стороне;
    \item[] $\pi_к$ - степень повышения давления в компрессоре;
    \item[] $\rho_т, кг/м^3$ - плотность газа на выходе из турбины;
    \item[] $Q_н^р$, Дж/кг - низшая теплота сгорания топлива;
    \item[] $\sigma$ - коэффициент регенерации;
    \item[] $\sigma_г$ - коэффициент сохранения полного давления в камере сгорания;
    \item[] $\sigma_ф$ - коэффициент сохранения полного давления входного фильтра;
    \item[] $\sigma_{вх}$ - коэффициент сохранения полного давления входного устройства компрессора;
    \item[] $\sigma_{р.х.}$ - коэффициент сохранения полного давления по холодной стороне регенератора;
    \item[] $\sigma_{тк}$ - коэффициент сохранения полного давления в патрубке после турбины компрессора;
    \item[] $\sigma_{р.г.}$ - коэффициент сохранения полного давления по горячей стороне регенератора;
    \item[] $\sigma_т$ - коэффициент сохранения полного давления в патрубке за силовой турбиной;
    \item[] $\sigma_{вых}$ - коэффициент сохранения полного давления в выходном устройстве;
    \item[] $T_a$, К - температура атмосферного воздуха;
    \item[] $T_к$, К - температура воздуха за компрессором;
    \item[] $T_р$, К - температура воздуха за регенератором по холодной стороне;
    \item[] $T_г$, К - температура газа перед турбиной;
    \item[] $T_0$, К - температура, при которой определяются теплоемкости веществ;
    \item[] $T_{тк}$, К - температура газа за турбиной компрессора;
    \item[] $T_т$, К - температура газа за силовой турбиной;
    \item[] $u_т$, м/с - окружная скорость на выходе из турбины;
\end{itemize}