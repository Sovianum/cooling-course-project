\subsection{Поступенчатый расчет турбины}
Для данного проекта выбрана одноступенчатая турбина.
Исходные параметры для поступенчатого расчета турбины приведены в табл. ~\ref{turbine:midline_inlet}.
Расчет проведен по методике ~\cite{gtd_theory_text_book, mikhaltsev_1, mikhaltsev_2}.
\begin{center}
	\begin{longtable}{|p{4cm}|c|c|c|}
		\caption{Исходные параметры поступенчатого расчета турбины}
		\label{turbine:midline_inlet}
		\endfirsthead
		\caption*{\tabcapalign Продолжение таблицы~\thetable}\\[-0.45\onelineskip]
		\hline
		\textbf{Величина} & \textbf{Обозначение} & \textbf{Размерность} & \textbf{Значение} \\ \hline
		\endhead
		\hline
		\textbf{Величина} & \textbf{Обозначение} & \textbf{Размерность} & \textbf{Значение} \\ \hline
			Реактивность ступени & $\rho$ & - & 0,3  \\ \hline
			Радиальный зазор & $\delta_r$ & м & $1,00 \cdot 10^{-3}$ \\ \hline
			Относительная длина лопатки статора & $\left( \frac{l}{D} \right)_1$ & - & $0,107$ \\ \hline
			Удлинение лопатки статора & $\left( \frac{l}{b_a} \right)_{СА}$ & - & $1,70$ \\ \hline
			Удлинение лопатки ротора & $\left( \frac{l}{b_a} \right)_{РК}$ & - & $2,10$ \\ \hline
			Относительная ширина зазора между лопатками ротора и лопатками статора & $\left( \frac{\delta}{b_a} \right)_{СА}$ & - & $0,10$ \\ \hline
			Угол раскрытия на втулке & $\gamma_{в}$ & \degree & $8,0$ \\ \hline
			Угол раскрытия на периферии & $\gamma_{п}$ & \degree & $20,0$ \\ \hline
			Удельная работа турбины & $H_т$ & Дж/кг & $0,341 \cdot 10^6$ \\ \hline
			Коэффициент скорости статора & $\phi$ & - & 0,94 \\ \hline
			Коэффициент скорости ротора & $\psi$ & - & 0,94 \\ \hline
			Направление скорости на выходе из СА & $\alpha_1$ & $\degree$ & 13,0 \\ \hline
			Частота вращения вала турбины & $n$ & $об/мин$ & 12000,0 \\ \hline
	\end{longtable}
\end{center}

Расчет параметров параметров ТВД приведен ниже. Параметры остальных турбин приведены в табл. ~\ref{tab:turbine-stage-total}.
\begin{enumerate}
	\item Определим теплоперепад на сопловом аппарате:
		$$H_с = \left( 1 - \rho \right) H_т =
		\left( 
			1 - 0,3 
		\right) \cdot 0,341 \cdot 10^6 = 
			0,238 \cdot 10^6 \/\ Дж/кг$$
	\item Определим скорость адиабатного истечения из СА:
		$$c_{1 ад} = \sqrt{2 H_с} = 
			\sqrt{2 \cdot 0,238 \cdot 10^6} = 690,5 \/\ м/с$$
	\item Определим скорость действительного истечения из СА:
		$$c_1 = \phi c_{1 ад} =
			0,94 \cdot 690,5 = 651,4 \/\ м/с$$
	\item Определим температуру на выходе из СА:
		$$T_1 = T_г - \frac{c_1^2}{2c_{pг}} =
			1450 - 
			\frac{
				{651,4}^2
			}{
				2 \cdot 1210,3
			} = 1274,7 \/\ К$$
	\item Определим температуру конца адиабатного расширения:
		$$T_1^\prime = T_г - \frac{H_c}{c_{pг}} =
			1450,0 - 
			\frac{
				0,238 \cdot 10^6
			}{
				1210,3
			} = 1255,0 \/\ К$$
	\item Определим давление на выходе из СА:
		$$p_1 = p_г \left( \frac{T_1^\prime}{T_г} \right)^\frac{k_г}{k_г - 1} =
			1,807 \cdot \left(
				 \frac{
				 	1255,0
				 }{
				 	1450,0
				 } 
			\right)^\frac{
				1,32
			}{
				1,32 - 1
			} = 0,991 \/\ МПа$$
	\item Определим плотность газа на выходе из СА:
		$$\rho_1 = \frac{p_1}{R_г T_1} =
			\frac{
				0,991 \cdot 10^6
			}{
				291,0 \cdot 1274,7
			} = 2,67 \/\ кг/м^3$$
	\item Зададим угол на выходе из СА:
		$$\alpha_1 = 13,0 \degree$$
	\item Определим осевую скорость на выходе из СА:
		$$c_{1a} = c_1 \cdot \sin \alpha_1 =
			651,4 \cdot 
			\sin13,0\degree 
			= 146,5 \/\ м/с$$
	\item Определим площадь на выходе из СА:
		$$A_1 = \frac{G}{c_{1a} \rho_1} =
			\frac{
				53,1
			}{
				146,5 \cdot 2,67
			} = 0,13 \/\ м^2$$
	\item Определим средний диаметр турбины на выходе из СА:
	$$D_1 = \sqrt{
		\frac{A_1}{\pi \left( \frac{l}{D} \right)_1}
		} = \sqrt{
			\frac{
				0,13
			}{
				\pi \cdot 0,107
			}
		} = 0,632 \/\ м $$
	\item Определим окружную скорость на среднем диаметре на входе в РК:
		$$u_1 = \frac{\pi D_1 n}{60} = 
			\frac{
				\pi \cdot 0,653 \cdot 12000,0
			}{60} = 398,7 \/\ м/с$$
	\item Определим относительную скорость на входе в РК:
		$$w_1 = \sqrt{c_1^2 + u_1^2 - 2 c_1 u_1 \cos \alpha_1} =$$
		$$
			=\sqrt{
				{651,4}^2 + 
				{398,7}^2 - 
				2 \cdot 651,4 \cdot 398,7 \cdot 
				\cos 13,0 \degree
			} = 277,9 \/\ м/с
		$$
	\item Определим температуру торможения в относительном движении на входе в РК:
		$$T_{w1} = T_1 + \frac{w_1^2}{2c_{p г}} = 
			1274,7 + 
			\frac{
				277,9^2
			}{
				2 \cdot 1194,1
			} = 1306,9 \/\ К$$
	\item Определим давление торможения в относительном движении на входе в РК:
		$$p_{w1} = p_1 \left( \frac{T_{w1}}{T_1} \right)^\frac{k_г}{k_г - 1} =
	 		0,991 \cdot \left( 
	 			\frac{
	 				1306,9
	 			}{
	 				1274,7
	 			} 
	 		\right)^\frac{
	 			1,32
	 		}{
	 			1,32 - 1
	 		} = 1,099 \/\ МПа$$
	 \item Определим теплоперепад на РК:
	 	$$H_л = H_т \rho \frac{T_1}{T_1^\prime} =
	 		0,341 \cdot 10^6 \cdot 0,3 \cdot \frac{
	 			1274,7
	 		}{
	 			1255,0
	 		} = 0,100 \cdot 10^6 \/\ Дж/кг$$
	\item Определим расстояние в осевом направлении между выходными кромками лопаток СА и выходными кромками лопаток РК:
		$$x = \frac{
		 	\frac{\delta_a}{ \left( \frac{l}{b_a} \right)_1 }	+
		 	\frac{1}{\left( \frac{l}{b_a} \right)_2 }
		}{
		 	1 - \frac{\tan \gamma_п + \tan \gamma_в}
		 	{2 \left( \frac{l}{b_a} \right)_2}
		} D_1 \left( \frac{l}{D} \right)_1 =
		\frac{
		 	\frac{
		 		0,10
		 	}{
		 		1,70
		 	}	+
		 	\frac{
		 		1
		 	}{
		 		2,10
		 	} 
		}{
			1 - \frac{
				\tan 20,0 \degree + \tan 8,0 \degree
			}{
				2 \cdot 2,10
			}
		} \cdot 0,632 \cdot 0,107 =
			0,042 \/\ м
		$$
	 \item Определим средний диаметра на выходе из РК:
		 $$D_2 = D_1 + \frac{\tan \gamma_п - \tan \gamma_в}{2} x =
	   		0,632 + 
	   		\frac{
	   			\tan 20,0 \degree - 
	   			\tan 8,0 \degree
	   		}{2} \cdot 0,042 =
   		0,653 \/\ м$$
	 \item Определим длину лопатки на выходе из РК:
		 $$l_2 = 
		 	D_1 \left( \frac{l}{D} \right)_1 + 
		 	\frac{\tan \gamma_п + \tan \gamma_в}{2} x =
	 	$$
	 	$$
	 		= 0,632 \cdot 
		 	0,107 +
		 	\frac{
		 		\tan 20,0 \degree + 
		 		\tan 8,0 \degree
		 	}{2} \cdot 0,042 =
		 		0,078 \/\ м
	 	$$
	 \item Определим относительную длину лопаток на выходе из РК:
		 $$\left( \frac{l}{D} \right)_2 = \frac{l_2}{D_2} = 
		 	\frac{
		 		0,078
		 	}{
		 		0,653
		 	} = 0,120$$
	 \item Определим окружную скорость на среднем диаметре на выходе из РК:
		 $$u_2 = \frac{\pi D_2 n}{60} = 
		 	\frac{
		 		\pi 
		 		\cdot 0,653 
		 		\cdot 12000,0
		 	}{60} = 411,7 \/\ м/с$$
	 \item Определим адиабатическую относительную скорость истечения газа из РК:
	 	$$w_{2 ад} = \sqrt{w_1^2 + 2H_л + \left( u_2^2 - u_1^2 \right)} =$$
	 	$$
	 		= \sqrt{
	 			{277,9}^2 + 
	 			2 \cdot 0,100 \cdot 10^6 + 
	 			\left( {411,7}^2 - {398,7}^2 \right)
	 		} = 535,8 \/\ м/с
	 	$$
	 \item Определим относительную скорость истечения газа из РК:
	 	$$w_2 = \psi w_{2 ад} =
	 		0,94 \cdot 535,8 = 
	 		505,5 \/\ м/с$$
	 \item Определим статическую температуру на выходе из РК:
		 $$
			 T_2 = T_1 + \frac{
			 	\left(
			 		w_1^2  - w_2^2
			 	\right) + \left(
			 		u_2^2 - u_1^2
			 	\right)
			 }{2 c_{p г}} =
		 $$
		 $$
		 	= 1274,7 + \frac{
			 	\left(
			 		{277,9}^2  - {505,5}^2 
			 	\right) + 
			 	\left( 
			 		{398,7}^2  - {411,7}^2
			 	\right)
		 	}{2 \cdot 1194,1} = 
		 		1204,4 \/\ К
		 $$
	 \item Определим статическую температуру при адиабатическом процессе в РК:
		 $$T_2^\prime = T_1 + \frac{
		 	\left(
		 		w_1^2  - w_{2 ад}^2
		 	\right) + 
		 	\left(
		 		u_2^2 - u_1^2
		 	\right)
		 }{2 c_{p г}} =
		$$
		$$
			= 1274,7 + \frac{
			 	\left(
			 		{277,9}^2  - {535,8}^2 
			 	\right) + 
			 	\left( 
			 		{398,7}^2  - {411,7}^2
			 	\right)
			}{2 \cdot 1194,1} = 
			1191,2 \/\ К
		$$
	 \item Определим давление на выходе из РК:
	 	$$p_2 = p_1 
	 		\left( 
	 			\frac{
	 				T_2^\prime
	 			}{
	 				T_1
	 			} 
	 		\right)^{
	 			\frac{
	 				k_г
	 			}{
	 				k_г - 1
	 			}
	 		} =
	 		0,991 
	 		\left( 
	 			\frac{
	 				1191,2
	 			}{
	 				1274,7
	 			} 
	 		\right)^{
	 			\frac{
	 				1,32
	 			}{
	 				1,32 - 1
	 			}
	 		} = 0,751 \/\ МПа$$
	 \item Определим угол в относительном движении на выходе из РК:
	 	$$\beta_2 = \arcsin\frac{c_{2a}}{w_2} = 
	 	\arcsin\frac{
	 		154,6
	 	}{
	 		505,5
	 	} = 17,8 \degree$$
	 \item Определим угол выхода из РК в абсолютном движении:
	 	$$\alpha_2 = \arctan\frac{w_2 \cos \beta_2 - u_2}{c_{2a}} =
	 	\arctan\frac{
	 		505,5 \cdot 
	 		\cos 17,8 \degree - 
	 		411,7
	 	}{
	 		154,6
	 	} = 65,8 \degree$$
	 \item Определим окружную составляющую скорости на выходе из РК:
	 	$$c_{2u} = w_2 \cos \beta_2 - u_2 =
		 	505,5 \cdot 
		 	\cos 17,8 \degree - 
		 	411,7 = 
		 	69,6 \/\ м/с$$
	 \item Определим скорость потока на выходе из РК:
	 	$$c_2 = \sqrt{c_{2u}^2 + c_{2a}^2} = 
	 		\sqrt{
	 			{69,6}^2 + {154,6}^2
	 		} = 169,6 \/\ м/с$$
	 \item Определим степень понижения давления в турбине:
	 	$$\pi_{т} = \frac{p_г}{p_2} = 
	 		\frac{
	 			1,807
	 		}{
	 			0,751
	 		} = 2,41 $$
	 \item Определим осевую составляющую скорости газа за турбиной:
	 	$$c_{2a} = c_2 \sin \alpha_2 = 
	 		169,6 \cdot
	 		\sin 65,8 \degree = 
	 		154,6 \/\ м/с$$
	 \item Определим плотность газа за турбиной:
	 	$$\rho_2 = \frac{G}{\pi \cdot c_{2a} \cdot D_2 \cdot l_2} = 
	 	\frac{
	 		53,1
	 	}{
	 		\pi \cdot 
	 		154,6 \cdot 
	 		0,653 \cdot 
	 		0,078
	 	} = 2,14 \/\ кг/м^3$$
	 \item Определим работу на окружности колеса:
	 $$L_u = c_{1u} u_1 + c_{2u} u_2 = 
	 	146,5 \cdot 398,7 + 
	 	154,6 \cdot 411,7 = 
	 	0,282 \cdot 10^6 \/\ Дж/кг$$
	 \item Определим КПД на окружности колеса:
	 	$$\eta_u = \frac{L_u}{H_t} = 
	 		\frac{
	 			0,282 \cdot 10^6
	 		}{
	 			0,341 \cdot 10^6
	 		} = 0,83 $$
	 \item Определим удельные потери на статоре:
		 $$h_c = \left( \frac{1}{\phi^2} - 1 \right) \frac{c_1^2}{2} =
		 \left( 
		 	\frac{
		 		1
		 	}{
		 		{0,94 }^2
		 	} - 1 
	 	\right) \frac{
	 		{651,4}^2
	 	}{2} = 24,46 \cdot 10^3 \/\ Дж/кг$$
	 \item Определим удельные потери на роторе:
	 	$$h_р = 
	 		\left( 
	 			\frac{1}{\psi^2} - 1 
	 		\right) \frac{w_2^2}{2} =
	 		\left( 
	 			\frac{1}{{0,94}^2} - 1 
	 		\right) \frac{
	 			{505,5}^2
	 		}{2} = 15,76 \cdot 10^3 \/\ Дж/кг$$
	 \item Определим удельные потери с выходной скоростью:
	 	$$h_{вых} = \frac{c_2^2}{2}= 
	 		\frac{
	 			{169,6}^2
	 		}{2} = 14,38 \cdot 10^3 \/\ Дж/кг$$
	 \item Определим удельные потери в радиальном зазоре:
	 	$$h_з = 1.37 \cdot \left( 1 + 1.6 \rho \right)
	 	\left[ 
	 		1 + 
	 		\left( 
	 			\frac{l}{D} 
	 		\right)_1 
	 	\right] \frac{
	 		\delta_r
	 	}{
	 		l_2
	 	} L_u = $$
	 $$ = 1.37 \cdot 
	 	\left( 
	 		1 + 1.6 \cdot 0,3 
	 	\right)
	 	\left[ 
	 		1 + 0,120
	 	\right] \frac{
	 		1,00 \cdot 10^{-3}
	 	}{
	 		0,078
	 	} \cdot 282 \cdot 10^3 =
	 	0,64 \cdot 10^3 \/\ Дж/кг$$
	 \item Определим удельные потери на вентиляцию:
	 	$$h_{вент} = 1.07 D_2^2 \left( \frac{u_2}{100} \right)^3 \rho_2 \cdot 1000 =$$
	 	$$
	 		=1.07 \cdot {0,653}^2 
	 			\left( 
		 			\frac{
		 				411,7
		 			}{
		 				100
		 			} 
	 			\right)^3 
	 			\cdot 2,14 
	 			\cdot 1000 = 1,29 \cdot 10^3 \/\ Дж/кг
	 	$$
	 \item Определим температуру торможения за РК:
	 	$$T_2^* = T_2 + \frac{h_з + h_{вент} + h_{вых}}{c_{pг}} =$$
	 	$$
	 		1204,4 + 
		 	\frac{
		 		0,64 \cdot 10^3 + 
		 		1,29 \cdot 10^3 + 
		 		14,38 \cdot 10^3
		 	}{
		 		1194,1
		 	} = 1218,1 \/\ К
	 	$$
	 \item Определим давление торможения за РК:
	 	$$p_2^* = p_2 
	 		\left( 
	 			\frac{
	 				T_2^*
	 			}{
	 				T_2
	 			} 
	 		\right)^{
	 			\frac{
	 				k_г
	 			}{
	 				k_г - 1
	 			}
	 		} =
	 	0,751 \cdot 
	 		\left( 
	 			\frac{
	 				1218,1
	 			}{
	 				1204,4
	 			} 
	 		\right)^{
	 			\frac{
	 				1,32
	 			}{
	 				1,32 - 1
	 			}
	 		} = 0,787 \/\ МПа$$
	 \item Определим мощностной КПД турбины:
	 	$$\eta_{т \/\ мощн} = 
	 		\eta_u - 
	 		\frac{
	 			h_з + h_{вент}
	 		}{
	 			H_т
	 		} =
	 		0,83 - 
	 		\frac{
	 			0,64 \cdot 10^3 + 1,29 \cdot 10^3
	 		}{
	 			0,341 \cdot 10^6
	 		} = 0,82$$
	 \item Определим работу турбины:
	 	$$L_т = H_т \eta_т = 
	 		0,341 \cdot 10^6 \cdot 
	 		0,82 = 
	 		0,280 \cdot 10^6 \/\ Дж/кг$$
	 \item Определим теплоперепад по параметрам торможения:
	 	$$H_т^* = c_{pг} T_г 
	 		\left[ 
	 			1 - 
	 				\left( 
	 					\frac{
	 						p_2^*
	 					}{
	 						p_г^*
	 					} 
	 				\right)^\frac{
	 					k_г - 1
	 				}{
	 					k_г
	 				} 
	 		\right] =
	 	$$
	 	$$
	 		= 1206,6 \cdot 1450,0 
	 		\left[ 1 - 
	 			\left( 
	 				\frac{
	 					0,787
	 				}{
	 					1,807
	 				} 
	 			\right)^\frac{
	 				1,32 - 1
	 			}{
	 				1,32
	 			} 
	 		\right] = 0,318 \cdot 10^6 \/\ Дж/кг 
	 	$$
	 \item Определим КПД турбины по параметрам торможения:
	 $$\eta_т^* = \frac{L_т}{H_т^*} =
	 	\frac{
	 		0,280 \cdot 10^6
	 	}{
	 		0,318 \cdot 10^6
	 	} = 0,88$$
\end{enumerate}