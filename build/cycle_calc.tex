\subsection{Расчет цикла при $\pi_{кнд} = 4.75, \pi_{квд} = 4$}
В данном расчете учет изменения теплофизических свойств рабочего тела в зависимости от его температуры производился
путем итерирования на каждом этапе расчета до тех пор, пока изменение искомого теплофизического свойства (теплоемкости или
показателя адиабаты) не составляло менее 0.1\% в сравнении с результатами предыдущей итерации. Ниже везде используются
значения теплофизический свойств на последнем этапе итерационных расчетов.

\begin{enumerate}
	\item Определим давление за входным устройством:
		$$p_{вх}^* = \sigma_{вх}  p_a = 0.98 \cdot 0.100 = 0.098 \/\ МПа$$
	\item Определим давление за КНД:
		$$p_{кнд}^* = \pi_{кнд} p_{вх}^* = 4.8 \cdot 0.098 = 0.466 \/\ МПа$$
	\item Определим адиабатический КПД КНД $\eta_{кнд}$, принимая показатель адиабаты воздуха $k_{в \/\ кнд} = 1.40$:
	    $$
	    	\eta_{кнд} = \frac{
		        \pi_{кнд}^\frac{
		            k_{в \/\ кнд} - 1
		        }{
		            k_{в \/\ кнд}
	            } - 1
		    }{
		        \pi_{кнд}^\frac{
		            k_{в \/\ кнд} - 1
	            }{
	                k_{в \/\ кнд} \cdot \eta_{пол \/\ кнд}
	            } - 1
		    } = \frac{
	            4.8^\frac{
	                1.40 - 1
	            }{
	                1.40
	            } - 1
	        }{
	            4.8^\frac{
	                1.40 - 1
	            }{
	                1.40 \cdot 0.840
	            } - 1
	        } = 0.80
	    $$
	\item Определим температуру газа за КНД:
		$$T_{КНД}^* = T_a 
		\left[ 
			1 + \frac{
				\pi_к^{
					\frac{
						k_{в \/\ кнд} - 1
					}{
						k_{в \/\ кнд}
					}
				} - 1
			}{
				\eta_{кнд}
			}
		\right] =
			288.0 
		\left[
			1 + \frac{
				{4.8}^{
					\frac{
						1.40 - 1
					}{
						1.40
					}
				} - 1
			}{
				0.80
			}
		\right] = 490.0 \/\ К$$
	\item Используя найденный показатель адиабаты воздуха, определим теплоемкость воздуха в процессе сжатия воздуха в КНД:
		$$c_{pв \/\ кнд} = \frac{
			k_{в \/\ кнд}
		}{
			k_{в \/\ кнд} - 1
		} R_в = \frac{
			1.40
		}{
			1.40 - 1
		} \cdot 287.0 = 1012.1 \/\ Дж/(кг \cdot К)$$
	\item Определим работу КНД:
		$$L_{КНД} = c_{pв \/\ кнд} \left( T_{кнд}^* - T_a \right) =
			1012.1 \cdot \left(490.0 - 288.0\right) =
			0.204 \cdot 10^6 \/\ Дж/кг $$
	\item Определим давление перед КВД:
		$$p_{0 \/\ квд}^* = \sigma_{кнд} p_{кнд}^* = 0.98 \cdot 0.466 = 0.456 \/\ МПа$$
	\item Определим давление за КВД:
		$$ p_{квд}^* = \pi_{квд} p_{0 \/\ квд}^* = 4.0 \cdot 0.456 = 1.825 \/\ МПа $$
	\item Определим адиабатический КПД КВД $\eta_{квд}$, принимая показатель адиабаты воздуха $k_{в \/\ КВД} = 1.37$:
	    $$
	    	\eta_{квд} = \frac{
		        \pi_{квд}^\frac{
		            k_{в \/\ квд} - 1
		        }{
		            k_{в \/\ квд}
	            } - 1
		    }{
		        \pi_{кнд}^\frac{
		            k_{в \/\ квд} - 1
	            }{
	                k_{в \/\ квд} \cdot \eta_{пол \/\ квд}
	            } - 1
		    } = \frac{
	            4.0^\frac{
	                1.37 - 1
	            }{
	                1.37
	            } - 1
	        }{
	            4.0^\frac{
	                1.37 - 1
	            }{
	                1.37 \cdot 0.820
	            } - 1
	        } = 0.78
	    $$
	\item Определим температуру газа за КВД:
		$$T_{квд}^* = T_{кнд}^*
		\left[ 
			1 + \frac{
				\pi_к^{
					\frac{
						k_в - 1
					}{
						k_в
					}
				} - 1
			}{
				\eta_{квд}
			}
		\right] =
			490.0 
		\left[
			1 + \frac{
				{4.0}^{
					\frac{
						1.37 - 1
					}{
						1.37
					}
				} - 1
			}{
				0.78
			}
		\right] = 784.3 \/\ К$$
	\item Используя найденный показатель адиабаты воздуха, определим теплоемкость воздуха в процессе сжатия воздуха в КВД:
		$$c_{pв \/\ квд} = \frac{
			k_{в \/\ квд}
		}{
			k_{в \/\ квд} - 1
		} R_в = \frac{
			1.37
		}{
			1.37 - 1
		} \cdot 287.0 = 1061.8 \/\ Дж/(кг \cdot К)$$
	\item Определим работу КВД:
		$$L_{квд} = c_{pв \/\ квд} \left( T_{квд}^* - T_{кнд}^* \right) =
			1061.8 \cdot \left(784.3 - 490.0\right) =
			0.313 \cdot 10^6 \/\ Дж/кг $$
	\item Температура газа за камерой сгорания:
		$$T_г^* = 1450 \/\ К$$
	\item Определим относительный расход топлива. Расчет носит итерационный характер. Ниже описана последняя итерация. Теплоемкость продуктов сгорания природного газа рассчитывается через показатель адиабаты и газовую постоянную газа. При этом газовая постоянная и истинный показатель адиабаты рассчитываются как средневзвешенное соответственных характеристик компонентов продуктов. При расчета приняты следующие значения:
	\begin{enumerate} % список значений для расчета удельного расхода топлива
		\item[1)] теплоемкость топлива:
			$$c_{pm} = 2226.0 \/\ Дж / (кг \cdot К);$$
		\item[2)] температура подачи топлива:
			$$T_m = 300.0 \/\ К;$$
		\item[3)] температура определения теплофизических параметров веществ:
			$$T_0 = 300.0 \/\ К;$$
		\item[4)] истинная теплоемкость воздуха перед камерой сгорания:
			$$c_{pв \/\ г}\left( T_{КВД} \right) = 1095.4 \/\ Дж/(кг \cdot К);$$
		\item[5)] истинная теплоемкость воздуха при температуре определения теплофизических параметров веществ:
			$$c_{pв \/\ г}\left( T_0 \right) = 1002.7 \/\ Дж/(кг \cdot К);$$
		\item[6)] низшая теплота сгорания топлива:
			$$Q_н^р = 49030 \cdot 10^3 \/\ Дж / кг;$$
		\item[7)] полнота сгорания:
			$$\eta_г = 0.98;$$
		\item[8)] масса воздуха, необходимая для сжигания 1 кг топлива:
			$$l_0 = 17.3 \/\ кг;$$
	\end{enumerate}
	
	\begin{enumerate}
		\item Зададимся коэффициентом избытка воздуха: $$\alpha = 3.32;$$
		\item Теплоемкость продуктов сгорания природного газа $c_{pг \/\ г}$ при данном значении коэффициента избытка воздуха при температуре $T_г$ составляет:
			$$c_{pг \/\ г}\left( T_г \right) = 1222.3 \/\ Дж/(кг \cdot К);$$
		\item Теплоемкость продуктов сгорания природного газа $c_{pг \/\ г}$ при данном значении коэффициента избытка воздуха при температуре $T_0$ составляет:
			$$c_{pг \/\ г}\left( T_0 \right) = 995.0 \/\ Дж / (кг \cdot К);$$
		\item Определим относительный расход топлива:
			$$
				a = c_{pг \/\ г} \left( T_г \right) T_г - c_{pв \/\ г} \left( T_{квд} \right) T_{квд} = 
			$$
			$$
				= 1222.3 \cdot 1450.0 -
				1222.3 \cdot 784.310 = 
				0.913 \cdot 10^6 \/\ Дж/кг
			$$
			$$
				b = \left(
					c_{pг \/\ г}\left( T_0 \right) - c_{pв \/\ г}\left( T_0 \right) = 
				\right) T_0 = 
			$$
			$$
				= \left(
					995.0 - 1002.7
				\right) \cdot 300.0 = 
				-2.327 \cdot 10^3 \/\ Дж/кг
			$$
			$$
				c = c_{pг \/\ г} \left( T_г \right) T_г - c_{pг \/\ г} \left( T_0 \right) T_0 = 
			$$
			$$
				= 1222.3 \cdot 1450.0 -
				995.0 \cdot 300.0 = 
				1.474 \cdot 10^6 \/\ Дж/кг
			$$
			$$
				d = c_{pm} \left( T_m - T_0 \right) = 
			$$
			$$
				= 2226.0 \left( 300.0 - 300.0 \right) =
				0 \/\ Дж/кг
			$$
			$$g_m = \frac{G_m}{G_в^г} =
				\frac{
					a - b
				}{
					Q_н^р \eta_г -
					c + d
				} = 
			$$
			$$
				= \frac{
					0.913 \cdot 10^6 + 2.327 \cdot 10^3
				}{
					49030 \cdot 10^3 \cdot 0.98 -
					1473.793 \cdot 10^6 + 0
				} = 0.017
			$$
		\item Определим коэффициент избытка воздуха:
			$$\alpha^\prime = \frac{1}{g_m l_0} =
		\frac{1}{0.017 \cdot 17.3} = 3.32$$
	\end{enumerate}

	\item Определим удельный расход через ТВД:
		$$g_{твд} = \left( 1 + g_m \right) \left( 1 - g_{ут \/\ твд} - g_{охл \/\ твд} \right) = $$
		$$
		= \left(
		    1 + 0.017
		\right) \left(
		    1 - 0.010 -
		    0.100
        \right) = 1.027$$
	\item Определим удельную работу ТВД:
		$$L_{твд} = \frac{L_{квд}}{g_{твд}\eta_{м \/\ вд}} = \frac{
			0.313 \cdot 10^6
		}{
			1.027 \cdot 0.990
		} = 0.280 \cdot 10^6 \/\ Дж/кг$$
	\item Определим давление газа перед ТВД:
		$$p_{г}^* = p_{тнд}^* \sigma_г = 1.825 \cdot 0.99 = 1.807 \/\ МПа$$
	\item Определим среднюю теплоемкость газа в процессе расширения газа в турбине, принимая показатель адиабаты газа $k_{г \/\ твд} = 1.32$:
		$$c_{pг \/\ твд} = \frac{k_{г \/\ твд}}{k_{г \/\ твд} - 1} R_г =
			\frac{
				1.32
			}{
				1.32 - 1
			} \cdot 291.0 = 1206.6 \/\ Дж/(кг \cdot К) $$
	\item Определим давление воздуха за ТВД:
		$$p_{твд}^* = p_г^*
			\left[
				1 - \frac{L_{твд}}{c_{pг \/\ твд} T_г \eta_{твд}}
			\right] ^ \frac{k_{г \/\ твд}}{k_{г \/\ твд} - 1} =
		$$
		$$
			= 1.807
			\left[
				1 - \frac{0.313 \cdot 10^6}
				{1206.6 \cdot 1450.0 \cdot 0.880}
			\right] ^ \frac{1.32}{1.32 - 1} =
			 0.787 \/\ МПа
		$$
	\item Определим температуру газа за ТВД:
	 	$$
	 		T_{твд}^* = T_г^*
			\left\lbrace
			 	1 -
			 	\left[
			 		1 -
			 			\left(
			 				\frac{p_{твд}^*}{p_г^*}
			 			\right) ^ \frac{k_{г \/\ твд}}{k_{г \/\ твд} - 1}
			 	\right] \eta_{ТВД}
			\right\rbrace =
		$$
		$$
			= 1450.0
			\left\lbrace
			 	1 -
			 	\left[
			 		1 -
			 			\left(
			 				\frac{0.787}{1.807}
			 			\right) ^ \frac{1.32}{1.32 - 1}
			 	\right] \cdot 0.880
			\right\rbrace = 1218.1 \/\ К
		$$
	\item Определим давление перед ТНД:
		$$p_{0 \/\ тнд}^* = p_{твд}^*\sigma_{твд} = 0.787 \cdot 0.98 = 0.771 \/\ МПа$$

	\item Определим удельный расход через ТНД:
		 $$g_{тнд} = g_{твд} \left( 1 - g_{ут \/\ тнд} - g_{охл \/\ тнд} + g_{охл \/\ твд}\right) = $$
		 $$=1.027 \cdot
		 	\left(
		 	    1 - 0.010 -
		 	    0.000 +
		 	    0.100
		 	\right) = 1.037$$
	\item Определим удельную работу ТНД:
		$$L_{тнд} = \frac{L_{кнд}}{g_{тнд}\eta_{м \/\ нд}} = \frac{
			0.204 \cdot 10^6
		}{
			1.037 \cdot 0.99
		} = 0.199 \cdot 10^6 \/\ Дж/кг$$
	\item Определим среднюю теплоемкость газа в процессе расширения газа в ТНД, принимая показатель адиабаты газа $k_{г \/\ тнд} = 1.33$:
		$$c_{pг \/\ тнд} = \frac{k_{г \/\ тнд}}{k_{г \/\ тнд} - 1} R_г =
			\frac{
				1.33
			}{
				1.33 - 1
			} \cdot 291.0 = 1178.1 \/\ Дж/(кг \cdot К) $$
	\item Определим давление воздуха за ТНД:
		$$
			p_{тнд}^* = p_{0 \/\ тнд}^*
				\left[
					1 - \frac{L_{тнд}}{c_{pг \/\ тнд} T_г \eta_{тнд}}
				\right] ^ \frac{k_{г \/\ тнд}}{k_{г \/\ тнд} - 1} =
		$$
		$$
			= 0.771
				\left[
					1 - \frac{
						0.204 \cdot 10^6
					}
					{
						1178.1 \cdot 1218.1 \cdot 0.90
					}
				\right] ^ \frac{1.33}{1.33 - 1} =
				 0.391 \/\ МПа
		$$
	\item Определим температуру газа за ТНД:
	 	$$
	 		T_{тнд}^* = T_{твд}^*
			\left\lbrace
			 	1 -
			 	\left[
			 		1 -
			 			\left(
			 				\frac{p_{тнд}^*}{p_{тнд \/\ 0}^*}
			 			\right) ^ \frac{k_{г \/\ тнд}}{k_{г \/\ тнд} - 1}
			 	\right] \eta_{тнд}
			\right\rbrace =
		$$
		$$
			= 1218.1
			\left\lbrace
			 	1 -
			 	\left[
			 		1 -
			 			\left(
			 				\frac{0.391}{0.771}
			 			\right) ^ \frac{1.33}{1.33 - 1}
			 	\right] \cdot 0.90
			\right\rbrace = 1049.1 \/\ К
		$$
	\item Определим давление перед свободной турбиной:
		$$p_{0 \/\ тс}^* = p_{тнд}^*\sigma_{тнд} = 0.391 \cdot 0.98 = 0.384 \/\ МПа$$
	\item Определим удельный расход через силовую турбину:
	    $$g_{тс} = g_{тнд} \left( 1 - g_{ут \/\ тс} - g_{охл \/\ тс} \right) =
            1.037 \cdot
            \left(
                1 - 0.010 -
                0.000
            \right) = 1.027$$
    \item Определим давление торможения на выходе из свободной турбины $p_{тс}^*$:
		$$p_{тс}^* = p_a / \sigma_{вых} = 0.100 \cdot 0.93 = 0.108 \/\ МПа$$
	\item Зададим значение приведенной скорости на выходе из свободной турбины:
		$$\lambda_{вых} = 0.30$$
	\item Определим статическое давление на выходе из свободной турбины, принимая показатель адиабаты газа на выходе из свободной турбины $k_{тс \/\ вых} = 1.36$:
		$$p_{тс} = p_{тс}^* \cdot \pi \left( \lambda_{вых}, \/\ k_{тс \/\ вых} \right)
        =
			0.108
			\cdot \pi \left( 0.30, \/\ 1.36 \right)
        = 0.102 \/\ МПа$$
	\item Определим статическую температуру на выходе из свободной турбины, принимая показатель адиабаты газа $k_{г \/\ тс} = 1.35$::
		$$
			T_{тс} = T_{тнд}^*
			\left\lbrace
			 	1 -
			 	\left[
			 		1 -
			 			\left(
			 				\frac{p_{0 \/\ тс}^*}{p_{тс}}
			 			\right) ^ \frac{k_{г \/\ тс}}{k_{г \/\ тс} - 1}
			 	\right] \eta_{тс}
			\right\rbrace =
		$$
		$$
			= 1049.1
			\left\lbrace
			 	1 -
			 	\left[
			 		1 -
			 			\left(
			 				\frac{
			 					0.384
			 				}{
			 					0.102
			 				}
			 			\right) ^ \frac{1.35}{1.35 - 1}
			 	\right] \cdot 0.92
			\right\rbrace = 769.6 \/\ К
		$$
	\item Определим температуру торможения на выходе из силовой турбины:
		$$T_{тс}^* = 
			\frac{T_{тс}}{\tau\left( \lambda_{вых}, \/\ k_{тс \/\ вых} \right)} =
			\frac{T_{тс}}{\tau\left( 0.30, \/\ 1.36 \right)} =
			= 780.2 \/\ К$$
	\item Определим значение теплоемкости газа в свободной турбине:
		$$c_{p \/\ тс} = 
			\frac{k_{г \/\ тс}}{k_{г \/\ тс} - 1} = 
			\frac{1.35}{1.35 - 1} = 1133.2 \/\ Дж / \left( кг \cdot К \right)$$
	\item Определим удельную работу силовой турбины:
		$$L_{тс} = c_{p \/\ тс} \left( T_{тнд}^* - T_{тс}^* \right) = 
			1133.2 \cdot \left( 1049.1 - 780.2 \right) =
			0.305 \cdot 10^6\/\ Дж/кг$$
	\item Определим удельную работу ГТД:
		$$L = L_{тс} \/\ g_{тс} =
			0.30 \cdot 10^6 \cdot 1.027 =
			0.305 \cdot 10^6 Дж/кг$$
	\item Определим экономичность ГТД:
		$$C_e = \frac{3600}{N_{e уд}} g_{тс} =
			\frac{3600}{0.305 \cdot 10^6} \cdot 1.03 =
			0.000 \cdot кг/\left( кВт/ч \right)$$
	\item Определим КПД ГТД:
		$$\eta_e = \frac{3600}{C_e Q_н^р} =
			\frac{3600}{0.000 \cdot 49.030 }
			= 0.382$$
	\item Определим потребную мощность ГТД:
		$$
			N = N_e / \eta_р = 16000 \cdot 10^3 \cdot \ 0.98 = 16327 \cdot 10^3 \/\ Вт
		$$
	\item Определим расход воздуха:
		$$G_в = \frac{N}{L} =
			\frac{16327 \cdot 10^3}{0.305 \cdot 10^6} =
			51.1 \/\ кг/с$$
\end{enumerate}