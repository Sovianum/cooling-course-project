\section*{ВВЕДЕНИЕ}
\addcontentsline{toc}{section}{ВВЕДЕНИЕ}

В данной выпускной квалификационной работе спроектирована газотурбинная установка
(ГТУ) по трехвальной схеме со свободной турбиной для эксплуатации в качестве привода
газоперекачивающего агрегата (ГПА) на линейной компрессорной стации магистрального
газопровода.

Задача проекта - создание конкурентоспособной высокоэффективной установки с высокой
экономичностью в широком диапазоне рабочих режимов, а также обладающей высоким межремонтным ресурсом.
Ключевыми факторами, позволившими решить эту проблему стали:
\begin{itemize}
    \item применение трехвальной схемы, имеющей хорошие регуляторные свойства;
    \item применение в качестве протипа хорошо зарекомендовавшей себя конструкции опор;
    \item предварительное захолаживание охлаждающего воздуха, отбираемого из компрессора во внешнем воздухо-водяном теплообменном аппарате;
    \item оптимизация системы охлаждения турбины высокого давления, позволившая снизить неравномерность температуры в
    сопловом аппарате турбины с 256,7 до 141,4 К без увеличения максимальной температуры металла и при снижении
    расхода охлаждающего воздуха на сопловой аппарат на 9\%.
\end{itemize}

Проектируемый двигатель состоит из следующих составных частей:
\begin{itemize}
    \item Компрессор низкого давления семиступенчатый;
    \item Компрессор высокого давления пятиступенчатый;
    \item Трубчато-кольцевая противоточная камера сгорания с выносными жаровыми трубами;
    \item Одноступенчатая турбина высокого давления;
    \item Одноступенчатая турбина низкого давления;
    \item Двухступенчатая силовая турбина;
    \item Выходное устройство с газосборником.
\end{itemize}

Конструктивно двигатель выполнен в виде двух модулей: газогенератор и силовая турбина. Каждый модуль установлен на отдельной
раме, что позволяет проводить их обслуживание и ремонт независимо.
