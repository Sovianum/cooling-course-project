\section{Расчет компрессора низкого давления по средней линии}

Для в данной работе детально представлен расчет первой ступени КНД по средней линии тока. Параметры остальных ступеней КНД и КВД представлены в табличном виде.

Зададим параметры, необходимы для расчета КНД по средней линии тока:
Исходные данные:
\begin{center}
	\begin{longtable}{|p{7cm}|c|c|c|}
		\hline
		\textbf{Величина} & \textbf{Обозначение} & \textbf{Размерность} & \textbf{Значение} \\ \hline
		\endhead
		Начальная температура воздуха & $T_1$ & К & $288.0$ \\ \hline
		Начальное давление воздуха & $p_1$ & $10^6 \/\ Па$ & $0.098$ \\ \hline
		Частота вращения вала & $n$ & об/мин & $9500$ \\ \hline
		Коэффициент напора текущей ступени & $\overline{H_{тi}}$ & - & $0.284$ \\ \hline
		Коэффициент напора следующей ступени & $\overline{H_{т \/\ i+1}}$ & - & $0.302$ \\ \hline
		Реактивность текущей ступени & $R_{i}$ & - & $0.500$ \\ \hline
		Реактивность следующей ступени & $R_{i+1}$ & - & $0.500$ \\ \hline
		Коэффициент работы & $k_H$ & - & $0.99$ \\ \hline
		КПД ступени & $\eta_{ад}^*$ & - & $0.817$ \\ \hline
		Безразмерная осевая скорость на входе в ступень & $\overline{c_a}$ & - & $0.50$ \\ \hline
		Относительный диаметр втулки на входе в ступень & $\overline{d_1}$ & - & $0.440$ \\ \hline
		Угол наклона внутреннего обвода проточной части & $\gamma_{в}$ & $\degree$ & 8.0 \\ \hline
		Угол наклона наружного обвода проточной части & $\gamma_{н}$ & $\degree$ & -8.0 \\ \hline
		Удлинение лопатки ротора & $\overline{b_{aр}}$ & - & 3.6 \\ \hline
		Удлинение лопатки статора & $\overline{b_{aс}}$ & - & 4.4 \\ \hline
		Относительная ширина зазора за лопатками ротора & $\delta_р$ & - & 0.10 \\ \hline
		Относительная ширина зазора за лопатками статора & $\delta_с$ & - & 0.10 \\ \hline
	\end{longtable}
\end{center}


\begin{enumerate}
	\item Определим окружную скорость на периферии рабочих лопаток на входе в ступень:
		$$
			u_к = \frac{\pi D_1 n}{60} = \frac{\pi \cdot 0.691 \cdot 9500}{60} = 343.67 \/\ м/с.
		$$
	\item Определим теоретический напор ступени:
		$$
			H_т = 
				\overline{H_{тi}} \cdot u_{кi}^2 = 
				0.284 \cdot 343.67^2 = 0.336 \cdot 10^5 \/\ Дж/кг.
		$$ 
	\item Определим действительную работу сжатия:
		$$
			L_z = 
				k_H \cdot H_т = 
				0.99 \cdot 0.336 \cdot 10^5 = 0.333 \cdot 10^5 \/\ Дж/кг.
		$$
	\item Определим адиабатическую работу сжатия:
		$$
			H_{ад} = L_z \cdot \eta_{ад}^* = 
				0.333 \cdot 10^5 \cdot 0.817 = 0.272 \cdot 10^5 \/\ Дж/кг.
		$$
	\item Определим повышение полной температуры в ступени:
		$$
			\Delta T^* = L_z / c_{pв} = 
				0.333 \cdot 10^5 / 1003.0 = 32.9 \/\ К.
		$$
	\item Определим степень повышения полного давления:
		$$
			\pi^* = 
			\left[ 
				1 + \frac{H_{ад}}{c_{pв} T_1^*}
			\right]^\frac{k_в}{k_в - 1} = 
			\left[ 
				1 + \frac{0.272 \cdot 10^5}{1003.0 \cdot 288.0}
			\right]^\frac{1.40}{1.40 - 1} = 1.37.
		$$
	\item Определим полное давление на выходе из ступени:
		$$
			p_3^* = p_1^* \cdot \pi^* = 
				0.098 \cdot 10^6 \cdot 1.37 = 
				0.134 \cdot 10^6  \/\ Па.
		$$
	\item Определим критическую скорость потока на входе в ступень:
		$$
			a_{кр1} = \sqrt{
				\frac{2 k_в}{k_в + 1} R_в T_1^*
			} = \sqrt{
				\frac{
					2 \cdot 1.40
				}{
					1.40 + 1
				} \cdot 287.0 \cdot 288.0
			} = 310.65 \/\ м/с.
		$$ 	
	\item Определим критическую скорость потока на выходе из ступени:
		$$
			a_{кр3} = \sqrt{
				\frac{2 k_в}{k_в + 1} R_в T_3^*
			} = \sqrt{
				\frac{
					2 \cdot 1.40
				}{
					1.40 + 1
				} \cdot 287.0 \cdot 320.9
			} = 327.78 \/\ м/с.
		$$ 	
	\item Определим относительный средний радиус на входе в ступень:
		$$
			\overline{r_{ср1}} = 
				\sqrt{\frac{1 + \overline{d_1}}{2}} = 
				\sqrt{\frac{1 + 0.44}{2}} = 0.85.
		$$
	\item Определим безразмерную окружную составляющую абсолютной скорости на входе в ступень:
		$$
			\overline{c_{u1}} = 
				\overline{r_{ср1}} \cdot \left( 
					1 - R_{ср \ i}
				\right) - 
				\frac{
					\overline{H_т}
				}{
					2 \overline{r_{ср1}}
				} = 
				0.85 \cdot
				\left( 
					1 - 0.50
				\right) - 
				\frac{
					0.28
				}{
					2 \cdot 0.85
				} = 0.26.
		$$
	\item Определим направление абсолютной скорости на входе в ступень:
		$$
			\alpha_1 = \arctan{\frac{
				\overline{c_{a1}}
			}{
				\overline{c_{u1}}
			}} = \arctan{\frac{
				0.50
			}{
				0.26
			}} = 62.8 \degree.
		$$
	\item Определим величину осевой скорости на входе в ступень:
		$$
			c_{a1} = u_к \cdot \overline{c_a} = 343.67 \cdot 0.50 = 171.83 \/\ м/с.
		$$
	\item Определим приведенную скорость на входе в ступень:
		$$
			\lambda_1 = 
				\frac{
					c_{a1}
				}{
					\sin{\alpha_1} \cdot a_{кр1}
				} = 
				\frac{
					171.83
				}{
					\sin{
						62.8 \degree
					} \cdot 310.65
				} = 0.62.
		$$
	\item Величина функции $Q\left( 
		\lambda, k_в, R_в
	\right) = \frac{
		m\left( k_в \right) q\left( \lambda \right)
	}{
		\sqrt{R_в}
	}$, соответствующая полученному значению приведенной скорости равна:
		$$
			Q\left( \lambda_1, k_в, R_в \right) = 0.03 \left( \frac{Дж}{кг \cdot К} \right)^{0.5}.
		$$
	\item Определим кольцевую площадь на входе в ступень:
		$$
			F_1 = 
			\frac{
				G \sqrt{T_1^*}
			}{
				p_1^* Q\left( \lambda_1, k_в, R_в\right) \sin{\alpha_1}
			} = 
			\frac{
				52.2 \cdot \sqrt{
					288.0
				}
			}{
				0.10 \cdot 10^6 \cdot 
				0.03 \cdot \sin{62.8 \degree}
			} = 0.30 \/\ м^2.
		$$
	\item Определим внешний и внутренний диаметры на входе в ступень:
		$$
			D_1 = \sqrt{
				\frac{4}{\pi} \cdot 
				\frac{1}{1 - \overline{d}^2} \cdot
				F_1
			} = 
			\sqrt{
				\frac{4}{\pi} \cdot 
				\frac{1}{1 - 0.44} \cdot
				0.30
			} = 0.691 \/\ м,	
		$$
		$$
			d_1 = D_1 \cdot \overline{d_1}^2 = 
				0.691 \cdot 0.440 = 
				0.304 \/\ м.
		$$
	\item Определим ширину ступени:
		$$
			x_{ступ} = 
				D_1 \cdot \frac{
					1 - \overline{d_1}
				}{2} \cdot \left(
					\frac{
						1 + \overline{\delta_р}
					}{
						\overline{b_{aр}}
					} + 
					\frac{
						1 + \overline{\delta_с}
					}{
						\overline{b_{aс}}
					}
				\right) =
		$$
		$$
				= 0.691 \cdot \frac{
					1 - 0.440
				}{2} \cdot \left(
					\frac{
						1 + 0.10
					}{
						3.6
					} + 
					\frac{
						1 + 0.10
					}{
						4.4
					}
				\right) = 0.103 \/\ м.
		$$
	\item Определим внешний и внутренний диаметры на выходе из ступени:
		$$
			D_3 = 
				D_1 + 2 \cdot x_{ступ} \tan{\gamma_{н}} = 
				0.691 + 2 \cdot 
				0.103 \cdot \tan{-8.0 \degree} =
				0.662 \/\ м, 
		$$
		$$
			d_3 =
				d_1 + 2 \cdot x_{ступ} \tan{\gamma_{в}} = 
				0.304 + 2 \cdot 
				0.103 \cdot \tan{8.0 \degree} =
				0.333 \/\ м 
		$$
	\item Определим кольцевую площадь на выходе из ступени:
		$$
			F_3 = 
				\frac{\pi}{4} \left( D_3^2 - d_3^2 \right) = 
				\frac{\pi}{4} \left( 
					0.662^2 - 0.333^2
				\right) = 0.257 \/\ м^2.
		$$
	\item Определим относительный диаметр втулки на выходе из ступени:
		$$
			\overline{d_3} = \frac{d_3}{D_3} = 
			\frac{0.333}{0.662} = 0.503.
		$$
	\item Определим относительный средний радиус на выходе из ступени:
		$$
			\overline{r_{ср \ 3}} = \sqrt{
				\frac{1 + \overline{d_3}^2}{2}
			} = 
			\sqrt{
				\frac{1 + 0.503^2}{2}
			} = 0.867.
		$$ 
	\item Определим безразмерную окружную составлющую абсолютной скорости на выходе из ступени:
		$$
			\overline{c_{u3}} = 
				\overline{r_3} \cdot \left( 
					1 - R_{ср \ i+1}
				\right) - 
				\frac{
					\overline{H_{т \ i+1}}
				}{
					2 \cdot \overline{r_{ср \ 3}}
				} =
				0.87 \cdot \left( 
					1 - 0.50
				\right) - 
				\frac{
					0.30
				}{
					2 \cdot 0.87
				} = 0.259. 
		$$
	\item Для определения приведенной скорости на выходе из ступени, численно решим уравнение:
		$$
			\frac{
				Q \left( 
				\lambda_3, k_в, R_в
			\right)
			}{
				\lambda_3
			} = \frac{
				a_{кр3}
			}{
				c_{a3}
			} \cdot \frac{
				G
			}{
				F_3
			} \cdot \frac{
				\sqrt{T_3^*}
			}{
				p_3^*
			}
		$$.
		Получим значение приведенной скорости на выходе:
		$$
			\lambda_3 = 0.56.
		$$
	\item Определим направление потока в абсолютном движении на выходе из ступени:
		$$
			\alpha_3 = \arcsin{
				\frac{
					c_{a3}
				}{
					\lambda_3 \cdot a_{кр \ 3}
				}
			} = \arcsin{
				\frac{
					158.78
				}{
					0.56 \cdot 327.78
				}
			} = 60.7 \degree.
		$$
	\item Определим безразмерную окружную составляющую абсолютной скорости на выходе из рабочего колеса:
		$$
			\overline{c_{u2}} = \frac{1}{\overline{r_{ср \ 2}}} 
			\left( 
				\overline{
					H_т
				} + \overline{c_{u1}} \overline{r_{ср1}}
			\right) = 
			\frac{1}{0.86} 
			\left( 
				0.28 + 
				0.26 \cdot 0.86
			\right) = 0.48.
		$$
	\item Определим углы потока в относительном движении:
		$$
			\beta_1 = \arctan{
				\frac{
					\overline{c_{a1}}
				}{
					\overline{r_{ср1}} - \overline{c_{u1}}
				}
			} = \arctan{
				\frac{
					0.50
				}{
					0.85 - 
					0.26
				}
			} = 40.2 \degree.
		$$
		$$
			\beta_2 = \arctan{
				\frac{
					\overline{c_{a2}}
				}{
					\overline{r_{ср2}} - \overline{c_{u2}}
				}
			} = \arctan{
				\frac{
					0.48
				}{
					0.86 - 
					0.58
				}
			} = 62.2\degree,
		$$
	\item Определим направление потока в абсолютном движении после рабочего колеса:
		$$
			\alpha_2 = \arctan{
				\frac{
					\overline{c_{a2}}
				}{
					\overline{c_{u2}}
				}
			} = \arctan{
				\frac{
					0.48
				}{
					0.58
				}
			} = 39.4\degree.
		$$
	\item Определим относительную скорость на среднем радиусе на входе в рабочее колесо:
		$$
			w_1 = \frac{c_{a1}}{\sin{\beta_1}} =
				\frac{
					171.8
				}{\sin{
					40.2
				}} = 266.3 \/\ м/с. 
		$$
	\item Определим относительную скорость на среднем радиусе на входе в НА:
		$$
			c_2 = \frac{
				c_{a2}
			}{
				\sin{\alpha_2}
			} = \frac{
				165.3
			}{
				\sin{
					39.4 \degree
				}
			} = 260.22 \/\ м/с.
		$$
\end{enumerate}



% section расчет_компрессора_низкого_давления_по_средней_линии (end)