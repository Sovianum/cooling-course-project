\subsection{Оптимизация системы охлаждения ГТУ}
Одной из особенностей проектируемой ГТУ является предварительное захолаживание воздуха, охлаждающего турбину высокого
давления, во внешнем воздухо-водяном теплообменном аппарате. Данная конструкция позволяет сильно уменьшить температуру
охлаждающего воздуха (с 771 К - температура нв выходе из КВД до 500 К - температура на входе в ТВД). Кроме того,
вывод охлаждающего воздуха за пределы корпуса позволяет также использовать дожимающий компрессор для увеличения
давления воздуха, что позволит осущестлять его выдув в лобовой точке соплового аппарата турбины высокого давления.

Схема установки с дожимающим компрессором представлена на рис.~\ref{img:sub_compress_scheme}.

В данной работе был проведен анализ эффективности этого конструктивного решения, с точки зрения параметров установки на номинальном режиме, а также с точки зрения оптимизации системы охлаждения. 

Подробный расчет системы охлаждения соплового аппарата приведен ниже.

При оценке влияния дожимающего компрессор на цикл установки, его параметры принимались следующими: степень
повышения давления, а его КПД.
Расход воздуха через дожимающий компрессор определялся из условия обеспечения наибольшей температуры металла соплового аппарата 1000 К (графики распределения температуры будут показаны ниже).

Без выдува в лобовую точку относительная доля охлаждающего воздуха, отводимого из компрессора высокого давления, составляет 10\%, что в абсолютном значении составляет 5,11 кг/c. Применение дожимающего компрессора позволило уменьшить эту величину до 9,64\%, что в абсолютном значении составляет 4,92\%.

Сравнение параметров исходной установки и установки с дожимающим компрессором представлено в табл.~\ref{tab:sub_compress_comparison}.
При этом на привод дожимающего компрессора тратится, что составляет 1\% от номинальной мощности
установки.

Как можно заметить, использование дожимающего компрессора приводит к небольшому снижению КПД установки, однако данное снижение представляется оправданным в связи с уменьшением температурной неравномерности в материале соплового аппарата турбины выского давления (показано ниже).