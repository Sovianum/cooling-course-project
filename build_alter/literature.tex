%\section*{Список использованных источников}
\addcontentsline{toc}{section}{СПИСОК ИСПОЛЬЗОВАННЫХ ИСТОЧНИКОВ}
\begin{thebibliography}{99}
    \bibitem{heat_exchangers} Теплообменные аппараты и системы охлаждения газотурбинных и комбинированных установок: учебник для
    вузов / Иванов В. Л., Леонтьев А. И., Манушин Э. А., Осипов М. И. ; ред. Леонтьев А. И. - 2-е изд.,
    стер. - М. : Изд-во МГТУ им. Н. Э. Баумана, 2004. - 591 с. : ил. - Библиогр.: с. 576-577. - ISBN 5-7038-2138-X.
    \bibitem{js_36_properties} Голубовский Е.Р., Светлов И.Л., Хвацкий К.К. Длительная прочность никелевых сплавов для
    монокристаллических лопаток газотурбинных установок // Журнал <<Конверсия в машиностроении>>. - 2005. - №3.
    \bibitem{gtd_theory_text_book} Теория и проектирование газотурбинных и комбинированных установок: учебник для вузов
    / Манушин Э.А., Михальцев В.Е., Чернобровкин А.П. - М. : Изд-во МГТУ им. Н.Э. Баумана, 1997.
    \bibitem{gtd_oil_and_gas} Б.П. Поршаков, А.А. Апостолов, В.И. Никишин. Газотурбинные установки: -М: ГУП Издательство
    «Нефть и газ» РГУ нефти и газа им. И.М. Губкина, 2003. – 240 с.
    \bibitem{gtd_tomsk} Рудаченко А.В. Газотурбинные установки для транспорта природного газа: учебное пособие второе издание
    переработанное: учебное пособие / А.В. Рудаченко, Н.В. Чухарева; Томский политехнический университет. – Томск:
    Изд-во Томского политехнического университета, 2012. – 213 с.
    \bibitem{cycle_methodics} Михальцев, В.Е. Расчет параметров цикла при проектировании газотурбинных двигателей и комбинированных установок : учеб. пособие / В.Д. Моляков, ред.: И.Г. Суровцев, В.Е. Михальцев .— Новосибирск : Изд-во НГТУ, 2014 .— 60 с. — ISBN 978-5-7038-3814-3.
    \bibitem{shlyakhtenko} Шляхтенко С.М., Сосунов В.А. Теория двухконтурных турбореактивных двигателе – М.: Машиностроение, 1979. – 432с.
    \bibitem{comp_char} Ланшин А.И., Зудов С.М., Умнов Е.И. Алгоритм обобщенного представления характеристик свехзвуковых
    компрессоров при математическом моделировании двигателей высокоскоростных летательных аппаратов. // Вопросы авиационной науки и техники. 1995. №2. С. 52–61.
    \bibitem{kazandjan} Теория авиационных двигателей. Теория лопаточных машин: Учебник для студентов, обучающихся по специ­альности
    «Эксплуатация летательных аппаратов и двигателей». /Под ред. П. К. Казанджана.— М.: Машиностро­ение, 1983.— 217 с., ил.
    \bibitem{radial_compressors} Ивановский Н.Н., Криворотько В.Н. Центробежные нагнетатели природного газа: Учебн. пособие для техникумов.
    – М.: Недра, 1994. – 176 с.: ил.
    \bibitem{mikhaltsev_1} В.Е. Михальцев, В.Д. Моляков. Теория и проектирование газовой
    турбины: Учеб. пособие по курсу «Лопаточные машины газотурбинных и
    комбинированных установок. Газовые турбины». – Ч.1: Теория и
    проектирование ступени газовой турбины / под ред. М.И. Осипова. – М.: Изд-во
    МГТУ им. Н.Э. Баумана, 2006. – 104 с.
    \bibitem{mikhaltsev_2} В.Е. Михальцев, В.Д. Моляков. Теория и проектирование газовой
    турбины: Учеб. пособие по курсу «Лопаточные машины газотурбинных и
    комбинированных установок. Газовые турбины». – Ч.2: Теория и
    проектирование многоступенчатой газовой турбины / под ред. М.И. Осипова. -
    М.: Изд-во МГТУ им. Н.Э. Баумана, 2008. – 116 с.
    \bibitem{ivanov} В.Л. Иванов. Воздушное охлаждение лопаток газовых турбин :
    Учеб. пособие по курсу «Системы охлаждения газотурбинных двигателей,
    газотурбинных и комбинированных установок» / под ред. М. И. Осипова – М.:
    Изд-во МГТУ им. Н.Э. Баумана, 2013. – 94 с.
    \bibitem{beknev} В.С. Бекнев. Расчет осевого компрессора. Методические указания
    по выполнению курсовых и дипломных проектов; под ред. Р.З. Тумашева. М.:
    Изд-во МГТУ им. Н.Э. Баумана, 1981. – 39 с.
    \bibitem{kondakov} А.И. Кондаков. Курсовое проектирование по технологии
    машиностроения. – М.: КНОРУС, 2012. – 400 с.
\end{thebibliography}