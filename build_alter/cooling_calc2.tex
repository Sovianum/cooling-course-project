\subsection{Расчет профиля температур}

Для расчета профиля температур лопатки принимаем расход воздуха, а величину зазора между дефлектором и
внутренней поверхностью лопатки.

При расчете профиля температур лопатки при конвективно-пленочно охлаждении будем пользоваться следующей методикой:
\begin{enumerate}
	\item Зададим распределение приведенной скорости по корыту и спинке:
		где - длина профиля со стороны корыта, - длина профиля со стороны спинки - приведенная скорость на входе в лопаточный венец,  - приведенная скорость на выходе из лопаточного венца.

	\item Определим критическую скорость звука:
	\item Определим скорость газа на корыте и на спинке:
	Дальнейший расчет идентичен для спинки и корыта, поэтому скорость газа будем обозначать как.
	\item Определим эквивалентную ширину щели:
		где - количество отверстий, - диаметр отверстия, - высота профильной части лопатки.
	\item Определим скорость газа в точке выдува воздуха:
		где - криволинейная координата отверстия.
	\item Определим статическую температуру газа в точке выдува воздуха:
	\item Определим статическое давление газа в точке выдува воздуха:
	\item Определим статическую плотность газа в точке выдува воздуха:
	\item Определим скорость истечения воздуха из отверстия:
	 	где - коэффициент скорости - температура воздуха в точке выдува, - давление воздуха.
	\item Определим статическую плотность воздуха на выходе из отверстия:
	\item Определим плотность торможения воздуха на входе в отверстия:
	\item Определим параметр вдува:
	\item Определим число Рейнольдса по ширине щели:
	\item Определим температурный фактор:
	\item Определим эффективность пленки:
	\item Определим темперутуру пленки в случае нескольких рядов отверстий:
	\item Определим коэффициент теплоотдачи пленки в случае нескольких рядов отверстий:
	\item По формуле истечения из сопла определим расход через ряд отверстий:
	\item В общем случае зависимость расхода воздуха в зазоре от криволинейной координаты имеет вид:

В данном расчете суммарный расход на охлаждение сопловых лопаток принимается равным
на лопатку, что при числе лопаток статора, равном 54, равно 4.89\% от суммарного расхода
воздуха.
В результате расчетов получим значения характерных параметров в отверстиях.

Значения характерных параметров в отверстиях корыта представлены в табл.~\ref{cool2:ps_hole_parameters}.

Значения характерных параметров в отверстиях спинки представлены в табл.~\ref{cool2:ps_hole_parameters}.


	\item Определим коэффициент теплоотдачи от газа на входной кромке лопатки:
	\item Определим коэффициент теплоотдачи на спинке на расстоянии:
	\item Определим коэффициент теплоотдачи на остальной выпуклой части (спинке):
	\item Определим коэффициет теплоотдачи на вогнутой части профиля (корыте):
	\item Коэффициент теплоотдачи от стенки к охлаждающему воздуху зависит от его температуры и определяется следующим уравнением:
	\item Уравнение теплообмена между охлаждающим воздухом и газом имеет вид:
	где - коэффициент теплопередачи, определяемый уравнением
	\item Уравнение теплового баланса малого элемента стенки лопатки
	где, К - температура материала лопатки,
	Вт/м - теплопроводность материала лопатки,
	, м - толщина стенки,
	, - коэффициент теплоотдачи пленки пленки газа снаружи лопатки,
	, T_{ст} - коэффициент теплоотдачи воздуха внутри лопатки,
	, К - температура пленки,
	, К - температура охлаждаюущего воздуха.

	Численно решая уравнение теплообмена, получим распределение параметров по спинке и корыту.
	Распределение параметров газа по спинке представлено в табл.~\ref{cool2:ss_gas_parameters}.
		
	Распределение параметров газа по корыту представлено в табл.~\ref{cool2:ps_gas_parameters}.
\end{enumerate}

Распределение температуры газа, воздуха и металла по профилю лопатки при исходном варианте установки
(без дожимающего компрессора) показано на рис.~\ref{img:cool_gas_parameters_no_front}.

Вариант с выдувом в лобовой точке (с дожимающим компрессором) показан на рис.~\ref{img:cool_gas_parameters_front}.

Таким образом, из расчета следует, что ни в одной точке температура материала лопатки не превышает 1000 К (максимальная температура
равна 998 К), что обеспечивает достаточную прочность лопаток
~\cite{js_36_properties}.

Из сравнения полученных распределений можно заметить, что выдув в лобовой точке приводит к существенному уменьшнию
неравномерности температуры материала сопловой лопатки (с 256,7 до 141,4 К), что приводит к увеличению ресурса горячей
части, так как коррозия и термические напряжения являются основными причинами разрушения сопловых лопаток турбины.