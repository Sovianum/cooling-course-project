\subsection{Поступенчатый расчет турбины}
Для данного проекта выбрана одноступенчатая турбина.
Исходные параметры для поступенчатого расчета турбины приведены в табл.~\ref{turbine:midline_inlet}.
Расчет проведен по методике ~\cite{gtd_theory_text_book, mikhaltsev_1, mikhaltsev_2}.
Параметры остальных турбин представлены в табличном виде в таблице~\ref{tab:turbine-stage-total}.

Расчет параметров параметров ТВД приведен ниже. Параметры остальных турбин приведены в табл.~\ref{tab:turbine-stage-total}.
\begin{enumerate}
	\item Определим теплоперепад на сопловом аппарате:
	\item Определим скорость адиабатного истечения из СА:
	\item Определим скорость действительного истечения из СА:
	\item Определим температуру на выходе из СА:
	\item Определим температуру конца адиабатного расширения:
	\item Определим давление на выходе из СА:
	\item Определим плотность газа на выходе из СА:
	\item Зададим угол на выходе из СА:
	\item Определим осевую скорость на выходе из СА:
	\item Определим площадь на выходе из СА:
	\item Определим средний диаметр турбины на выходе из СА:
	\item Определим окружную скорость на среднем диаметре на входе в РК:
	\item Определим относительную скорость на входе в РК:
	\item Определим температуру торможения в относительном движении на входе в РК:
	\item Определим давление торможения в относительном движении на входе в РК:
	 \item Определим теплоперепад на РК:
	\item Определим расстояние в осевом направлении между выходными кромками лопаток СА и выходными кромками лопаток РК:
	 \item Определим средний диаметра на выходе из РК:
	 \item Определим длину лопатки на выходе из РК:
	 \item Определим относительную длину лопаток на выходе из РК:
	 \item Определим окружную скорость на среднем диаметре на выходе из РК:
	 \item Определим адиабатическую относительную скорость истечения газа из РК:
	 \item Определим относительную скорость истечения газа из РК:
	 \item Определим статическую температуру на выходе из РК:
	 \item Определим статическую температуру при адиабатическом процессе в РК:
	 \item Определим давление на выходе из РК:
	 \item Определим угол в относительном движении на выходе из РК:
	 \item Определим угол выхода из РК в абсолютном движении:
	 \item Определим окружную составляющую скорости на выходе из РК:
	 \item Определим скорость потока на выходе из РК:
	 \item Определим степень понижения давления в турбине:
	 \item Определим осевую составляющую скорости газа за турбиной:
	 \item Определим плотность газа за турбиной:
	 \item Определим работу на окружности колеса:
	 \item Определим КПД на окружности колеса:
	 \item Определим удельные потери на статоре:
	 \item Определим удельные потери на роторе:
	 \item Определим удельные потери с выходной скоростью:
	 \item Определим удельные потери в радиальном зазоре:
	 \item Определим удельные потери на вентиляцию:
	 \item Определим температуру торможения за РК:
	 \item Определим давление торможения за РК:
	 \item Определим мощностной КПД турбины:
	 \item Определим работу турбины:
	 \item Определим теплоперепад по параметрам торможения:
	 \item Определим КПД турбины по параметрам торможения:
\end{enumerate}