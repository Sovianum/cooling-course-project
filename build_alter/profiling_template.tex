\subsection{Профилирование ступени ТВД}
Исходными данными для данного этапа проектирования турбины являются результы расчета по средней линии тока.

Ступень была спрофилирована по закону.

Определим треугольники скоростей на произвольном радиусе лопатки.

\begin{enumerate}

	\item В этом случае значения абсолютной скорости на входе на рабочие лопатки на произвольном радиусе определялись по следующим формулам (в приведенных ниже формулах значения со штрихом относятся к среднему радиусу):
	\item Окружная скорость рабочей лопатки на произвольном радиусе была определена по закону вращения твердого тела:
	\item Относительная скорость на произвольном радиусе на входе в рабочие лопатки была определена по следующим формулам:
	\item Абсолютная скорость на выходе из рабочих лопаток была определена по условию постоянства работы, отводимой от газа на различных радиусах лопатки.

	По формуле Эйлера для правила отсчета углов, принятого в теории турбин удельная работа на окружности колеса $L_u$ определяется слеюущей формулой:
	Таким образом, зная работу на окружности колеса на среднем радиусе лопатки $L_u^\prime$, мы можем определить значение окружной скорости на выходе из рабочих лопаток:
	\item Используя значения окружной и осевой скорости на среднем радиусе лопатки, определим значение осевой скорости на выходе из рабочих лопаток, проинтегрировав уравнение радиального равновесия:
	\item Значения проекций относительной скорости на выходе из лопаток находим так же, как и значения на входе в рабочие лопатки.

\end{enumerate}
Распределение углов на входе в рабочие лопатки турбины и на выходе из них представлено на рис.~\ref{img:profile_inlet_angles}
и~\ref{img:profile_outlet_angles}, соответственно:
	