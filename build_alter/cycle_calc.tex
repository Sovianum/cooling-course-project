\subsection{Расчет цикла}
В данном расчете учет изменения теплофизических свойств рабочего тела в зависимости от его температуры производился
путем итерирования на каждом этапе расчета до тех пор, пока изменение искомого теплофизического свойства (теплоемкости или
показателя адиабаты) не составляло менее 0.1\% в сравнении с результатами предыдущей итерации. Ниже везде используются
значения теплофизический свойств на последнем этапе итерационных расчетов.

\begin{enumerate}
	\item Определим давление за входным устройством:
	\item Определим давление за КНД:
	\item Определим адиабатический КПД КНД, принимая показатель адиабаты воздуха:
	\item Определим температуру газа за КНД:
	\item Используя найденный показатель адиабаты воздуха, определим теплоемкость воздуха в процессе сжатия воздуха в КНД:
	\item Определим работу КНД:
	\item Определим давление перед КВД:
	\item Определим давление за КВД:
	\item Определим адиабатический КПД КВД, принимая показатель адиабаты воздуха:
	\item Определим температуру газа за КВД:
	\item Используя найденный показатель адиабаты воздуха, определим теплоемкость воздуха в процессе сжатия воздуха в КВД:
	\item Определим работу КВД:
	\item Определим относительный расход топлива. Расчет носит итерационный характер. Ниже описана последняя итерация. Теплоемкость продуктов сгорания природного газа рассчитывается через показатель адиабаты и газовую постоянную газа. При этом газовая постоянная и истинный показатель адиабаты рассчитываются как средневзвешенное соответственных характеристик компонентов продуктов. При расчета приняты следующие значения:
	\begin{enumerate} % список значений для расчета удельного расхода топлива
		\item[1)] теплоемкость топлива:
		\item[2)] температура подачи топлива:
		\item[3)] температура определения теплофизических параметров веществ:
		\item[4)] истинная теплоемкость воздуха перед камерой сгорания:
		\item[5)] истинная теплоемкость воздуха при температуре определения теплофизических параметров веществ:
		\item[6)] низшая теплота сгорания топлива:
		\item[7)] полнота сгорания:
		\item[8)] масса воздуха, необходимая для сжигания 1 кг топлива:
	\end{enumerate}
	
	\begin{enumerate}
		\item Зададимся коэффициентом избытка воздуха:
		\item Теплоемкость продуктов сгорания природного газа при данном значении коэффициента избытка воздуха при температуре составляет:
		\item Теплоемкость продуктов сгорания природного газа при данном значении коэффициента избытка воздуха при температуре составляет:
		\item Определим относительный расход топлива:
		\item Определим коэффициент избытка воздуха:
	\end{enumerate}

	\item Определим удельный расход через ТВД:
	\item Определим удельную работу ТВД:
	\item Определим давление газа перед ТВД:
	\item Определим среднюю теплоемкость газа в процессе расширения газа в турбине, принимая показатель адиабаты газа:
	\item Определим давление воздуха за ТВД:
	\item Определим температуру газа за ТВД:
	\item Определим давление перед ТНД:
	\item Определим удельный расход через ТНД:
	\item Определим удельную работу ТНД:
	\item Определим среднюю теплоемкость газа в процессе расширения газа в ТНД, принимая показатель адиабаты газа:
	\item Определим давление воздуха за ТНД:
	\item Определим температуру газа за ТНД:
	\item Определим давление перед свободной турбиной:
	\item Определим удельный расход через силовую турбину:
    \item Определим давление торможения на выходе из свободной турбины:
	\item Зададим значение приведенной скорости на выходе из свободной турбины:
	\item Определим статическое давление на выходе из свободной турбины, принимая показатель адиабаты газа на выходе из свободной турбины:
	\item Определим статическую температуру на выходе из свободной турбины, принимая показатель адиабаты газа:
	\item Определим температуру торможения на выходе из силовой турбины:
	\item Определим значение теплоемкости газа в свободной турбине:
	\item Определим удельную работу силовой турбины:
	\item Определим удельную работу ГТД:
	\item Определим экономичность ГТД:
	\item Определим КПД ГТД:
	\item Определим потребную мощность ГТД:
	\item Определим расход воздуха:
\end{enumerate}