\label{sec:ecology}

\subsection{Назначение двигателя}
\label{sub:ecology_engine_purpose}

Двигатель предназначен для использования в качестве привода нагнетателя на линейных компрессорных станциях природного газа.

Двигатель выполнен трехвальным со свободной турбиной.

Мощность – 16 МВт.

Основные части: компрессор низкого давления (КНД), компрессор высокого давления (КВД), камера сгорания (КС), турбина высокого давления (КВД), турбина низкого давления (КНД), силовая турбина (ТС), выходное устройство.

Частоты вращения валов: высокого давления – 12000 об/мин, среднего давления – 9500, низкого давления – 7800 об/мин.

Топливо – природный газ.

Температура газа за камерой сгорания – 1450 К.

\subsection{Анализ вредных и опасных производственных факторов на этапе эксплуатации} % (fold)
\label{sub:ecology_factor_analisys}

При эксплуатации двигателя к вредным и опасным факторам относятся:
\begin{itemize}
	\item Повышенный уровень шума на рабочем месте, вызванный всасыванием воздуха, колебанием газа в элементах проточной части, колебанием элементов конструкции из-за вращения ротора, истечения реактивной струи из выходного устройства.
	\item Загрязнение воздуха в области, прилегающей к компрессорной станции, продуктам сгорания топлива, содержащими оксиды азота, углерода, сажу; парами масла из системы смазки (Таблица~\ref{ecology:factor_analisys}).
	\item Повышенный уровень вибраций из-за дисбаланса вращающихся масс (Таблица~\ref{ecology:factor_analisys}).
	\item Повышенный уровень температуры в рабочей зоне вследствие нагрева корпуса двигателя (Таблица~\ref{ecology:factor_analisys}).
	\item Повышенный уровень температур поверхностей оборудования и поверхностей проточной части: в компрессоре за счет сжатия воздуха, в турбине – за счет температуры горячего газа (Таблица~\ref{ecology:factor_analisys}).
\end{itemize}

Анализ перечисленных факторов представлен в таблице~\ref{ecology:factor_analisys} с указанием нормативного документа и нормативных значений рассмотренных производственных факторов.

\pagebreak

\subsection{Анализ уровня шума на станции} % (fold)
\label{sub:ecology_noise_analisys}

Расчет производился в программном комплексе АРМ «Акустика».

Расчет был произведен для машинного отделения и двух прилегающих комнат – комнаты управления и электротехнического отсека.
Схема расчетной области представлена на рис.~\ref{img:ecology_plan}.

Соответствующая модель, построенная в программном комплексе АРМ «Акустика» приведен на рис.~\ref{img:ecology_bc}.

Для расчета уровней шума использованы шумовые характеристики вентилятора и выходного устройства двигателя НК38-СТ,
идентичные разрабатываемому двигателю. Уровни звукового давления, дБ в октавных полосах со среднегеометрическими частотами, Гц  представлены в таблице 5.2.

\pagebreak

Изолинии звукового давления, полученные в результате расчета показаны на рис.~\ref{img:ecology_result}.

Сравнение уровней звукового давления в расчетной точке, находящейся в комнате управления с нормативным (согласно СН 2.2.4/2.1.8.562-
96 таблица 2, строка 4) представлено в таблице 5.3.

Из полученных данных следует, что уровень шума на рабочем месте, не превышает нормативного ни в одном из частотных диапазонов.

\subsection{Оценка размера зоны распространения облака горючих газов и паров при аварии} % (fold)
\label{sub:ecology_cloud}

Оценка размера зоны распространения облака горючих газов заключается в определении зоны с концентрацией горючего вещества выше нижнего концентрационного предела воспламенения (НКПВ). Для природного газа эта величина равна 29 мг/л.
Исходные данные для проведения расчета приведены в таблице 5.4.

\pagebreak

Определим массу газа между отсечными клапанами, кг:
Определим массу воздуха, мгновенно вовлекающуюся в облако углеводородов, кг:
таким образом,
Принимается, что образовавшееся облако дрейфует по ветру со скоростью, где – скорость ветра, и имеет в начальный момент форму цилиндра, высота которого равна его радиусу. С течением времени высота облака уменьшается, а радиус растет.

Скорость ветра зависит от класса устойчивость по Паскуиллу. В данном расчете принимается класс по Паскуиллу B, что соответствует наиболее опасному случаю – наибольшему распространению углеводородного облака. Соответствующая этому классу устойчивости скорость ветра.
Изменение во времени радиуса, высоты облака и концентрации газа в нем в начальной фазе (фаза падения) определяется путем решения систем обыкновенных дифференциальных уравнений:
где, кг – масса воздуха в облаке, – плотность воздуха, – радиус облака, – коэффициенты (), – ускорение свободного падения;

Ri – число Ричардсона, определяемое из соотношения:
– высота облака, – температура облака, – плотность паровоздушного воздуха.
Для решения системы уравнений необходимо дополнительное соотношение:
В качестве критерия окончания фазы падения принимается выполнение условие
Зависимость определяется из соотношения:
Концентрация газа в точке с координатами  определяется по формуле:
где – среднеквадратичные отклонения, зависящие от величины – координата центра облака в направлении ветра; – координата точки окончания фазы падения.

При принимается;

при.

Результатом расчета является пространственное распределение концентраций углеводородного облака. Срез такого
распределения на уровне земли () представлен на рис.~\ref{img:ecology_cloud}.

Из полученного решения видно, что зона воспламенения по мере движения облака распространяется вплоть до расстояния 130 м по направлению ветра. Следовательно, открытый огонь недопустим в радиусе 130 м от станции.
