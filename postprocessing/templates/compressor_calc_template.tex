\subsection{Расчет компрессора низкого давления по средней линии}

Для в данной работе детально представлен расчет первой ступени КНД по средней линии тока по методике ~\cite{beknev}.
Параметры остальных ступеней КНД и КВД представлены в табличном виде
(КНД - табл. ~\ref{tab:lpc-stage-total}, КВД - табл. ~\ref{tab:hpc-stage-total}).
Исходные параметры для расчета КНД по средне линии тока представлены в таблице ~\ref{midline:compressor_inlet}.
\begin{center}
	\begin{longtable}{|p{7cm}|c|c|c|}
		\caption{Исходные параметры для расчета КНД по средней линии тока}
		\label{midline:compressor_inlet}
		\endfirsthead
		\caption*{\tabcapalign Продолжение таблицы~\thetable}\\[-0.45\onelineskip]
		\hline
		\textbf{Величина} & \textbf{Обозначение} & \textbf{Размерность} & \textbf{Значение} \\ \hline
		\endhead
		\hline
		\textbf{Величина} & \textbf{Обозначение} & \textbf{Размерность} & \textbf{Значение} \\ \hline
		Начальная температура воздуха & $T_1$ & К & $<-<.T1 | Round1>->$ \\ \hline
		Начальное давление воздуха & $p_1$ & $10^6 \/\ Па$ & $<-<.P1 | DivideE6 | Round3>->$ \\ \hline
		Частота вращения вала & $n$ & об/мин & $<-<.RPM | Round>->$ \\ \hline
		Коэффициент напора текущей ступени & $\overline{H_{тi}}$ & - & $<-<.HtCoefCurr | Round3>->$ \\ \hline
		Коэффициент напора следующей ступени & $\overline{H_{т \/\ i+1}}$ & - & $<-<.HtCoefNext | Round3>->$ \\ \hline
		Реактивность текущей ступени & $R_{i}$ & - & $<-<.ReactivityCurr | Round3>->$ \\ \hline
		Реактивность следующей ступени & $R_{i+1}$ & - & $<-<.ReactivityNext | Round3>->$ \\ \hline
		Коэффициент работы & $k_H$ & - & $<-<.Kh | Round2>->$ \\ \hline
		КПД ступени & $\eta_{ад}^*$ & - & $<-<.Eta | Round3>->$ \\ \hline
		Безразмерная осевая скорость на входе в ступень & $\overline{c_a}$ & - & $<-<.CARel1 | Round2>->$ \\ \hline
		Относительный диаметр втулки на входе в ступень & $\overline{d_1}$ & - & $<-<.RotorDF.DRelIn | Round3>->$ \\ \hline
		Угол наклона внутреннего обвода проточной части & $\gamma_{в}$ & $\degree$ & <-<.RotorDF.GammaIn | Degree | Round1>-> \\ \hline
		Угол наклона наружного обвода проточной части & $\gamma_{н}$ & $\degree$ & <-<.RotorDF.GammaOut | Degree | Round1>-> \\ \hline
		Удлинение лопатки ротора & $\overline{b_{aр}}$ & - & <-<.RotorDF.Elongation | Round1>-> \\ \hline
		Удлинение лопатки статора & $\overline{b_{aс}}$ & - & <-<.StatorDF.Elongation | Round1>-> \\ \hline
		Относительная ширина зазора за лопатками ротора & $\delta_р$ & - & <-<.RotorDF.DeltaRel | Round2>-> \\ \hline
		Относительная ширина зазора за лопатками статора & $\delta_с$ & - & <-<.StatorDF.DeltaRel | Round2>-> \\ \hline
	\end{longtable}
\end{center}


\begin{enumerate}
	\item Определим окружную скорость на периферии рабочих лопаток на входе в ступень:
		$$
			u_к = \frac{\pi D_1 n}{60} = \frac{\pi \cdot <-<.RotorDF.DOutIn | Round3>-> \cdot <-<.RPM | Round>->}{60} = <-<.UOut | Round2>-> \/\ м/с.
		$$
	\item Определим теоретический напор ступени:
		$$
			H_т = 
				\overline{H_{тi}} \cdot u_{кi}^2 = 
				<-<.HtCoefCurr | Round3>-> \cdot <-<.UOut | Round2>->^2 = <-<.Ht | DivideE5 | Round3>-> \cdot 10^5 \/\ Дж/кг.
		$$ 
	\item Определим действительную работу сжатия:
		$$
			L_z = 
				k_H \cdot H_т = 
				<-<.Kh | Round2>-> \cdot <-<.Ht | DivideE5 | Round3>-> \cdot 10^5 = <-<.Lz | DivideE5 | Round3>-> \cdot 10^5 \/\ Дж/кг.
		$$
	\item Определим адиабатическую работу сжатия:
		$$
			H_{ад} = L_z \cdot \eta_{ад}^* = 
				<-<.Lz | DivideE5 | Round3>-> \cdot 10^5 \cdot <-<.Eta | Round3>-> = <-<.HAd | DivideE5 | Round3>-> \cdot 10^5 \/\ Дж/кг.
		$$
	\item Определим повышение полной температуры в ступени:
		$$
			\Delta T^* = L_z / c_{pв} = 
				<-<.Lz | DivideE5 | Round3>-> \cdot 10^5 / <-<.CpAir | Round1>-> = <-<.DT | Round1>-> \/\ К.
		$$
	\item Определим степень повышения полного давления:
		$$
			\pi^* = 
			\left[ 
				1 + \frac{H_{ад}}{c_{pв} T_1^*}
			\right]^\frac{k_в}{k_в - 1} = 
			\left[ 
				1 + \frac{<-<.HAd | DivideE5 | Round3>-> \cdot 10^5}{<-<.CpAir | Round1>-> \cdot <-<.T1 | Round1>->}
			\right]^\frac{<-<.KAir | Round2>->}{<-<.KAir | Round2>-> - 1} = <-<.Pi | Round2>->.
		$$
	\item Определим полное давление на выходе из ступени:
		$$
			p_3^* = p_1^* \cdot \pi^* = 
				<-<.P1 | DivideE6 | Round3>-> \cdot 10^6 \cdot <-<.Pi | Round2>-> = 
				<-<.P3 | DivideE6 | Round3>-> \cdot 10^6  \/\ Па.
		$$
	\item Определим критическую скорость потока на входе в ступень:
		$$
			a_{кр1} = \sqrt{
				\frac{2 k_в}{k_в + 1} R_в T_1^*
			} = \sqrt{
				\frac{
					2 \cdot <-<.KAir | Round2>->
				}{
					<-<.KAir | Round2>-> + 1
				} \cdot <-<.RAir | Round1>-> \cdot <-<.T1 | Round1>->
			} = <-<.ACrit1 | Round2>-> \/\ м/с.
		$$ 	
	\item Определим критическую скорость потока на выходе из ступени:
		$$
			a_{кр3} = \sqrt{
				\frac{2 k_в}{k_в + 1} R_в T_3^*
			} = \sqrt{
				\frac{
					2 \cdot <-<.KAir | Round2>->
				}{
					<-<.KAir | Round2>-> + 1
				} \cdot <-<.RAir | Round1>-> \cdot <-<.T3 | Round1>->
			} = <-<.ACrit3 | Round2>-> \/\ м/с.
		$$ 	
	\item Определим относительный средний радиус на входе в ступень:
		$$
			\overline{r_{ср1}} = 
				\sqrt{\frac{1 + \overline{d_1}}{2}} = 
				\sqrt{\frac{1 + <-<.RotorDF.DRelIn | Round2>->}{2}} = <-<.RotorDF.RRelIn | Round2>->.
		$$
	\item Определим безразмерную окружную составляющую абсолютной скорости на входе в ступень:
		$$
			\overline{c_{u1}} = 
				\overline{r_{ср1}} \cdot \left( 
					1 - R_{ср \ i}
				\right) - 
				\frac{
					\overline{H_т}
				}{
					2 \overline{r_{ср1}}
				} = 
				<-<.RotorDF.RRelIn | Round2>-> \cdot
				\left( 
					1 - <-<.ReactivityCurr | Round2>->
				\right) - 
				\frac{
					<-<.HtCoefCurr | Round2>->
				}{
					2 \cdot <-<.RotorDF.RRelIn | Round2>->
				} = <-<.CURel1 | Round2>->.
		$$
	\item Определим направление абсолютной скорости на входе в ступень:
		$$
			\alpha_1 = \arctan{\frac{
				\overline{c_{a1}}
			}{
				\overline{c_{u1}}
			}} = \arctan{\frac{
				<-<.CARel1 | Round2>->
			}{
				<-<.CURel1 | Round2>->
			}} = <-<.Triangle1.Alpha | Degree | Round1>-> \degree.
		$$
	\item Определим величину осевой скорости на входе в ступень:
		$$
			c_{a1} = u_к \cdot \overline{c_a} = <-<.UOut | Round2>-> \cdot <-<.CARel1 | Round2>-> = <-<.Triangle1.CA | Round2>-> \/\ м/с.
		$$
	\item Определим приведенную скорость на входе в ступень:
		$$
			\lambda_1 = 
				\frac{
					c_{a1}
				}{
					\sin{\alpha_1} \cdot a_{кр1}
				} = 
				\frac{
					<-<.Triangle1.CA | Round2>->
				}{
					\sin{
						<-<.Triangle1.Alpha | Degree | Round1>-> \degree
					} \cdot <-<.ACrit1 | Round2>->
				} = <-<.Lambda1 | Round2>->.
		$$
	\item Величина функции $Q\left( 
		\lambda, k_в, R_в
	\right) = \frac{
		m\left( k_в \right) q\left( \lambda \right)
	}{
		\sqrt{R_в}
	}$, соответствующая полученному значению приведенной скорости равна:
		$$
			Q\left( \lambda_1, k_в, R_в \right) = <-<.Q1 | Round2>-> \left( \frac{Дж}{кг \cdot К} \right)^{0.5}.
		$$
	\item Определим кольцевую площадь на входе в ступень:
		$$
			F_1 = 
			\frac{
				G \sqrt{T_1^*}
			}{
				p_1^* Q\left( \lambda_1, k_в, R_в\right) \sin{\alpha_1}
			} = 
			\frac{
				<-<.MassRate | Round1>-> \cdot \sqrt{
					<-<.T1 | Round1>->
				}
			}{
				<-<.P1 | DivideE6 | Round2>-> \cdot 10^6 \cdot 
				<-<.Q1 | Round2>-> \cdot \sin{<-<.Triangle1.Alpha | Degree | Round1>-> \degree}
			} = <-<.RotorDF.AreaIn | Round2>-> \/\ м^2.
		$$
	\item Определим внешний и внутренний диаметры на входе в ступень:
		$$
			D_1 = \sqrt{
				\frac{4}{\pi} \cdot 
				\frac{1}{1 - \overline{d}^2} \cdot
				F_1
			} = 
			\sqrt{
				\frac{4}{\pi} \cdot 
				\frac{1}{1 - <-<.RotorDF.DRelIn | Round2>->} \cdot
				<-<.RotorDF.AreaIn | Round2>->
			} = <-<.RotorDF.DOutIn | Round3>-> \/\ м,	
		$$
		$$
			d_1 = D_1 \cdot \overline{d_1}^2 = 
				<-<.RotorDF.DOutIn | Round3>-> \cdot <-<.RotorDF.DRelIn | Round3>-> = 
				<-<.RotorDF.DInIn | Round3>-> \/\ м.
		$$
	\item Определим ширину ступени:
		$$
			x_{ступ} = 
				D_1 \cdot \frac{
					1 - \overline{d_1}
				}{2} \cdot \left(
					\frac{
						1 + \overline{\delta_р}
					}{
						\overline{b_{aр}}
					} + 
					\frac{
						1 + \overline{\delta_с}
					}{
						\overline{b_{aс}}
					}
				\right) =
		$$
		$$
				= <-<.RotorDF.DOutIn | Round3>-> \cdot \frac{
					1 - <-<.RotorDF.DRelIn | Round3>->
				}{2} \cdot \left(
					\frac{
						1 + <-<.RotorDF.DeltaRel | Round2>->
					}{
						<-<.RotorDF.Elongation | Round1>->
					} + 
					\frac{
						1 + <-<.StatorDF.DeltaRel | Round2>->
					}{
						<-<.StatorDF.Elongation | Round1>->
					}
				\right) = <-<.StageWidth | Round3>-> \/\ м.
		$$
	\item Определим внешний и внутренний диаметры на выходе из ступени:
		$$
			D_3 = 
				D_1 + 2 \cdot x_{ступ} \tan{\gamma_{н}} = 
				<-<.RotorDF.DOutIn | Round3>-> + 2 \cdot 
				<-<.StageWidth | Round3>-> \cdot \tan{<-<.RotorDF.GammaOut | Degree | Round1>-> \degree} =
				<-<.StatorDF.DOutOut | Round3>-> \/\ м, 
		$$
		$$
			d_3 =
				d_1 + 2 \cdot x_{ступ} \tan{\gamma_{в}} = 
				<-<.RotorDF.DInIn | Round3>-> + 2 \cdot 
				<-<.StageWidth | Round3>-> \cdot \tan{<-<.RotorDF.GammaIn | Degree | Round1>-> \degree} =
				<-<.StatorDF.DInOut | Round3>-> \/\ м 
		$$
	\item Определим кольцевую площадь на выходе из ступени:
		$$
			F_3 = 
				\frac{\pi}{4} \left( D_3^2 - d_3^2 \right) = 
				\frac{\pi}{4} \left( 
					<-<.StatorDF.DOutOut | Round3>->^2 - <-<.StatorDF.DInOut | Round3>->^2
				\right) = <-<.StatorDF.AreaOut | Round3>-> \/\ м^2.
		$$
	\item Определим относительный диаметр втулки на выходе из ступени:
		$$
			\overline{d_3} = \frac{d_3}{D_3} = 
			\frac{<-<.StatorDF.DInOut | Round3>->}{<-<.StatorDF.DOutOut | Round3>->} = <-<.StatorDF.DRelOut | Round3>->.
		$$
	\item Определим относительный средний радиус на выходе из ступени:
		$$
			\overline{r_{ср \ 3}} = \sqrt{
				\frac{1 + \overline{d_3}^2}{2}
			} = 
			\sqrt{
				\frac{1 + <-<.StatorDF.DRelOut | Round3>->^2}{2}
			} = <-<.StatorDF.RRelOut | Round3>->.
		$$ 
	\item Определим безразмерную окружную составлющую абсолютной скорости на выходе из ступени:
		$$
			\overline{c_{u3}} = 
				\overline{r_3} \cdot \left( 
					1 - R_{ср \ i+1}
				\right) - 
				\frac{
					\overline{H_{т \ i+1}}
				}{
					2 \cdot \overline{r_{ср \ 3}}
				} =
				<-<.StatorDF.RRelOut | Round2>-> \cdot \left( 
					1 - <-<.ReactivityNext | Round2>->
				\right) - 
				\frac{
					<-<.HtCoefNext | Round2>->
				}{
					2 \cdot <-<.StatorDF.RRelOut | Round2>->
				} = <-<.CURel3 | Round3>->. 
		$$
	\item Для определения приведенной скорости на выходе из ступени, численно решим уравнение:
		$$
			\frac{
				Q \left( 
				\lambda_3, k_в, R_в
			\right)
			}{
				\lambda_3
			} = \frac{
				a_{кр3}
			}{
				c_{a3}
			} \cdot \frac{
				G
			}{
				F_3
			} \cdot \frac{
				\sqrt{T_3^*}
			}{
				p_3^*
			}
		$$.
		Получим значение приведенной скорости на выходе:
		$$
			\lambda_3 = <-<.Lambda3 | Round2>->.
		$$
	\item Определим направление потока в абсолютном движении на выходе из ступени:
		$$
			\alpha_3 = \arcsin{
				\frac{
					c_{a3}
				}{
					\lambda_3 \cdot a_{кр \ 3}
				}
			} = \arcsin{
				\frac{
					<-<.Triangle3.CA | Round2>->
				}{
					<-<.Lambda3 | Round2>-> \cdot <-<.ACrit3 | Round2>->
				}
			} = <-<.Triangle3.Alpha | Degree | Round1>-> \degree.
		$$
	\item Определим безразмерную окружную составляющую абсолютной скорости на выходе из рабочего колеса:
		$$
			\overline{c_{u2}} = \frac{1}{\overline{r_{ср \ 2}}} 
			\left( 
				\overline{
					H_т
				} + \overline{c_{u1}} \overline{r_{ср1}}
			\right) = 
			\frac{1}{<-<.RotorDF.RRelOut | Round2>->} 
			\left( 
				<-<.HtCoefCurr | Round2>-> + 
				<-<.CURel1 | Round2>-> \cdot <-<.RotorDF.RRelOut | Round2>->
			\right) = <-<.CARel2 | Round2>->.
		$$
	\item Определим углы потока в относительном движении:
		$$
			\beta_1 = \arctan{
				\frac{
					\overline{c_{a1}}
				}{
					\overline{r_{ср1}} - \overline{c_{u1}}
				}
			} = \arctan{
				\frac{
					<-<.CARel1 | Round2>->
				}{
					<-<.RotorDF.RRelIn | Round2>-> - 
					<-<.CURel1 | Round2>->
				}
			} = <-<.Triangle1.Beta | Degree | Round1>-> \degree.
		$$
		$$
			\beta_2 = \arctan{
				\frac{
					\overline{c_{a2}}
				}{
					\overline{r_{ср2}} - \overline{c_{u2}}
				}
			} = \arctan{
				\frac{
					<-<.CARel2 | Round2>->
				}{
					<-<.RotorDF.RRelOut | Round2>-> - 
					<-<.CURel2 | Round2>->
				}
			} = <-<.Triangle2.Beta | Degree | Round1>->\degree,
		$$
	\item Определим направление потока в абсолютном движении после рабочего колеса:
		$$
			\alpha_2 = \arctan{
				\frac{
					\overline{c_{a2}}
				}{
					\overline{c_{u2}}
				}
			} = \arctan{
				\frac{
					<-<.CARel2 | Round2>->
				}{
					<-<.CURel2 | Round2>->
				}
			} = <-<.Triangle2.Alpha | Degree | Round1>->\degree.
		$$
	\item Определим относительную скорость на среднем радиусе на входе в рабочее колесо:
		$$
			w_1 = \frac{c_{a1}}{\sin{\beta_1}} =
				\frac{
					<-<.Triangle1.CA | Round1>->
				}{\sin{
					<-<.Triangle1.Beta | Degree | Round1>->
				}} = <-<.Triangle1.W | Round1>-> \/\ м/с. 
		$$
	\item Определим относительную скорость на среднем радиусе на входе в НА:
		$$
			c_2 = \frac{
				c_{a2}
			}{
				\sin{\alpha_2}
			} = \frac{
				<-<.Triangle2.CA | Round1>->
			}{
				\sin{
					<-<.Triangle2.Alpha | Degree | Round1>-> \degree
				}
			} = <-<.Triangle2.C | Round2>-> \/\ м/с.
		$$
\end{enumerate}



% section расчет_компрессора_низкого_давления_по_средней_линии (end)