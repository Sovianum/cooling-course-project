\subsection{Расчет цикла при $\pi_{КНД} = <-<.LPCompressor.Pi>->, \pi_{КВД} = <-<.HPCompressor.Pi>->$}
% Расчет ГТД носит итерационный характер, так как параметры газа на входе в камеру сгорания
% за счет регенератора зависят от температуры на выходе из силовой турбины.
% Таким образом, расчет, представленный ниже, относится к последней итерации для
% случая $\pi_к = <PiComp>$. При этом принимается температура на выходе из силовой
% турбины: $T_т = <FreeTurbineTOut>$.
\begin{enumerate}
	\item Определим давление за входным устройством:
		$$p_{вх}^* = \sigma_{вх}  p_a = <-<.InletPipe.Sigma>-> \cdot <-<.GasSource.P>-> = <-<.InletPipe.POut>-> \/\ МПа$$


	\item Определим давление за КНД:
		$$p_{КНД}^* = \pi_к p_{вх}^* = <-<.LPCompressor.PIn>-> \cdot <-<.LPCompressor.Pi>-> = <-<.LPCompressor.POut>-> \/\ МПа$$
	\item Определим температуру газа за КНД, принимая показатель адиабаты воздуха $k_{в \/\ КНД} = <-<.LPCompressor.GasData.KMean>->$:
		$$T_{КНД}^* = T_a 
		\left[ 
			1 + \frac{
				\pi_к^{
					\frac{
						k_{в \/\ КНД} - 1
					}{
						k_{в \/\ КНД}
					}
				} - 1
			}{
				\eta_к
			}
		\right] =
			<-<.LPCompressor.TIn>-> 
		\left[
			1 + \frac{
				{<-<.LPCompressor.Pi>->}^{
					\frac{
						<-<.LPCompressor.GasData.KMean>-> - 1
					}{
						<-<.LPCompressor.GasData.KMean>->
					}
				} - 1
			}{
				<-<.LPCompressor.Eta>->
			}
		\right] = <-<.LPCompressor.TOut>-> \/\ К$$
	\item Используя найденный показатель адиабаты воздуха, определим теплоемкость воздуха в процессе сжатия воздуха в КНД:
		$$c_{pв \/\ КНД} = \frac{
			k_{в \/\ КНД}
		}{
			k_{в \/\ КНД} - 1
		} R_в = \frac{
			<-<.LPCompressor.GasData.KMean>->
		}{
			<-<.LPCompressor.GasData.KMean>-> - 1
		} \cdot <-<.LPCompressor.GasData.R>-> = <-<.LPCompressor.GasData.CpMean>-> \/\ Дж/кг$$
	\item Определим работу КНД:
		$$L_{КНД} = c_{pв \/\ КНД} \left( T_{КНД}^* - T_a \right) =
			<-<.LPCompressor.GasData.CpMean>-> \cdot \left(<-<.LPCompressor.TOut>-> - <-<.LPCompressor.TIn>->\right) =
			<-<.LPCompressor.Labour>-> \cdot 10^6 \/\ Дж/кг $$
	



	\item Определим давление перед КВД:
		$$p_{0 \/\ КВД}^* = \sigma_{КНД} p_{КНД}^* = <-<.LPCompressorPipe.Sigma>-> \cdot <-<.LPCompressor.POut>-> = <-<.HPCompressor.PIn>-> \/\ МПа$$




	\item Определим давление за КВД:
		$$ p_{КВД}^* = \pi_{КВД} p_{0 \/\ КВД}^* = <-<.HPCompressor.PIn>-> \cdot <-<.HPCompressor.Pi>-> = <-<.HPCompressor.POut>-> \/\ МПа $$
	\item Определим температуру газа за КВД, принимая показатель адиабаты воздуха $k_{в \/\ КВД} = <-<.HPCompressor.GasData.KMean>->$:
		$$T_{КВД}^* = T_{КНД}^*
		\left[ 
			1 + \frac{
				\pi_к^{
					\frac{
						k_в - 1
					}{
						k_в
					}
				} - 1
			}{
				\eta_к
			}
		\right] =
			<-<.HPCompressor.TIn>-> 
		\left[
			1 + \frac{
				{<-<.HPCompressor.Pi>->}^{
					\frac{
						<-<.HPCompressor.GasData.KMean>-> - 1
					}{
						<-<.HPCompressor.GasData.KMean>->
					}
				} - 1
			}{
				<-<.HPCompressor.Eta>->
			}
		\right] = <-<.HPCompressor.TOut>-> \/\ К$$
	\item Используя найденный показатель адиабаты воздуха, определим теплоемкость воздуха в процессе сжатия воздуха в КВД:
		$$c_{pв \/\ КВД} = \frac{
			k_{в \/\ КВД}
		}{
			k_{в \/\ КВД} - 1
		} R_в = \frac{
			<-<.HPCompressor.GasData.KMean>->
		}{
			<-<.HPCompressor.GasData.KMean>-> - 1
		} \cdot <-<.HPCompressor.GasData.R>-> = <-<.HPCompressor.GasData.CpMean>-> \/\ Дж/кг$$
	\item Определим работу КВД:
		$$L_{КВД} = c_{pв \/\ КВД} \left( T_{КВД}^* - T_{КНД}^* \right) =
			<-<.HPCompressor.GasData.CpMean>-> \cdot \left(<-<.HPCompressor.TOut>-> - <-<.HPCompressor.TIn>->\right) =
			<-<.HPCompressor.Labour>-> \cdot 10^6 \/\ Дж/кг = $$




	\item Температура газа за камерой сгорания:
		$$T_г^* = <-<.Burner.Tg>-> \/\ К$$
	\item Определим относительный расход топлива. Расчет носит итерационный характер. Ниже описана последняя итерация. Теплоемкость продуктов сгорания природного газа рассчитывается через показатель адиабаты и газовую постоянную газа. При этом газовая постоянная и истинный показатель адиабаты рассчитываются как средневзвешенное соответственных характеристик компонентов продуктов). При расчета приняты следующие значения:
	\begin{enumerate} % список значений для расчета удельного расхода топлива
		\item[1)] теплоемкость топлива:
			$$c_{pm} = <-<.Burner.Fuel.C>-> \/\ Дж / (кг \cdot К);$$
		\item[2)] температура подачи топлива:
			$$T_m = <-<.Burner.Fuel.TInit>-> \/\ К;$$
		\item[3)] температура определения теплофизических параметров веществ:
			$$T_0 = <-<.Burner.Fuel.T0>-> \/\ К;$$
		\item[4)] истинная теплоемкость воздуха перед камерой сгорания:
			$$c_{pв \/\ г}\left( T_{КВД} \right) = <-<.Burner.AirDataInlet.Cp>-> \/\ Дж/(кг \cdot К);$$
		\item[5)] истинная теплоемкость воздуха при температуре определения теплофизических параметров веществ:
			$$c_{pв \/\ г}\left( T_0 \right) = <-<.Burner.AirData0.Cp>-> \/\ Дж/(кг \cdot К);$$
		\item[6)] низшая теплота сгорания топлива:
			$$Q_н^р = <-<.Burner.Fuel.QLower>-> \cdot 10^3 \/\ Дж / (кг \cdot К);$$
		\item[7)] полнота сгорания:
			$$\eta_г = <-<.Burner.Eta>->;$$
		\item[8)] масса воздуха, необходимая для сжигания 1 кг топлива:
			$$l_0 = <-<.Burner.Fuel.L0>-> \/\ кг;$$
	\end{enumerate}

	\begin{enumerate}
		\item Зададимся коэффициентом избытка воздуха: $$\alpha = <-<.Burner.Alpha>->;$$
		\item Теплоемкость продуктов сгорания природного газа $c_{pг \/\ г}$ при данном значении коэффициента избытка воздуха при температуре $T_г$ составляет:
			$$c_{pг \/\ г}\left( T_г \right) = <-<.Burner.GasDataOutlet.Cp>-> \/\ Дж/(кг \cdot К);$$
		\item Теплоемкость продуктов сгорания природного газа $c_{pг \/\ г}$ при данном значении коэффициента избытка воздуха при температуре $T_0$ составляет:
			$$c_{pг \/\ г}\left( T_0 \right) = <-<.Burner.GasData0.Cp>-> \/\ Дж / (кг \cdot К);$$
		\item Определим относительный расход топлива:
			$$g_m = \frac{G_m}{G_в^г} =
		\frac{
			c_{pг \/\ г} \left( T_г \right) T_г -
			c_{pв \/\ г} \left( T_{КВД} \right) T_{КВД} -
			\left[
				\left(
					c_{pг \/\ г}\left( T_0 \right) - c_{pв \/\ г}\left( T_0 \right)
				\right) T_0
			\right]
		}{
			Q_н^р \eta_г -
			\left[
				c_{pг \/\ г} \left( T_г \right) T_г -
				c_{pг \/\ г} \left( T_0 \right) T_0
			\right] +
			c_{pm} \left( T_m - T_0 \right)
		} = $$

		$$=
		\frac{
			<-<.Burner.GasDataOutlet.Cp>-> \cdot <-<.Burner.Tg>-> -
			<-<.Burner.GasDataOutlet.Cp>-> \cdot <-<.HPCompressor.TOut>-> -
			\left[
				\left(
					<-<.Burner.GasData0.Cp>-> - <-<.Burner.AirData0.Cp>->
				\right) \cdot <-<.Burner.AirData0.T>->
			\right]
		}{
			<-<.Burner.Fuel.QLower>-> \cdot 10^3 \cdot <-<.Burner.Eta>-> -
			\left[
				<-<.Burner.GasDataOutlet.Cp>-> \cdot <-<.Burner.Tg>-> -
				<-<.Burner.GasData0.Cp>-> \cdot <-<.Burner.AirData0.T>->
			\right] +
			<-<.Burner.Fuel.C>-> \left( <-<.Burner.Fuel.TInit>-> - <-<.Burner.AirData0.T>-> \right)
		} = <-<.Burner.FuelMassRateRel>->$$
		\item Определим коэффициент избытка воздуха:
			$$\alpha^\prime = \frac{1}{g_m l_0} =
		\frac{1}{<-<.Burner.FuelMassRateRel>-> \cdot <-<.Burner.Fuel.L0>->} = <-<.Burner.Alpha>->$$
	\end{enumerate}

	\item Определим удельный расход через ТВД:
		$$g_{ТВД} = \left( 1 + g_m \right) \left( 1 - g_{ут} - g_{охл} \right) =
			\left( 1 + <-<.Burner.FuelMassRateRel>-> \right) \left( 1 - <-<.HPTurbine.LeakMassRateRel>-> - <-<.HPTurbine.CoolMassRateRel>-> \right) = <-<.HPTurbine.MassRateRel>->$$
	\item Определим удельную работу ТВД:
		$$L_{ТВД} = \frac{L_{КВД}}{g_{ТВД}\eta_{м \/\ ВД}} = \frac{<-<.HPCompressor.Labour>-> \cdot 10^6}{<-<.HPTurbine.MassRateRel>-> \cdot <-<.HPShaft.Eta>-> } = <-<.HPTurbine.Labour>-> \cdot 10^6 \/\ Дж/кг$$
	\item Определим давление газа перед ТВД:
		$$p_{г}^* = p_{ТНД}^* \sigma_г = <-<.HPCompressor.POut>-> \cdot <-<.Burner.Sigma>-> = <-<.HPTurbine.PIn>-> \/\ МПа$$
	\item Определим среднюю теплоемкость газа в процессе расширения газа в турбине, принимая показатель адиабаты газа $k_{г \/\ ТВД} = <-<.HPTurbine.GasData.KMean>->$:
		$$c_{pг \/\ ТВД} = \frac{k_{г \/\ ТВД}}{k_{г \/\ ТВД} - 1} R_г =
			\frac{<-<.HPTurbine.GasData.KMean>->}{<-<.HPTurbine.GasData.KMean>-> - 1} \cdot <-<.HPTurbine.GasData.R>-> = <-<.HPTurbine.GasData.CpMean>-> \/\ Дж/(кг \cdot К) $$
	\item Определим давление воздуха за ТВД:
		$$p_{ТВД}^* = p_г^*
			\left[
				1 - \frac{L_{КВД}}{c_{pг \/\ ТВД} T_г \eta_{ТВД}}
			\right] ^ \frac{k_{г \/\ ТВД}}{k_{г \/\ ТВД} - 1} =
			<-<.HPTurbine.PIn>->
			\left[
				1 - \frac{<-<.HPCompressor.Labour>-> \cdot 10^6}
				{<-<.HPTurbine.GasData.CpMean>-> \cdot <-<.HPTurbine.TIn>-> \cdot <-<.HPTurbine.Eta>->}
			\right] ^ \frac{<-<.HPTurbine.GasData.KMean>->}{<-<.HPTurbine.GasData.KMean>-> - 1} =
			 <-<.HPTurbine.POut>-> \/\ МПа$$
	\item Определим температуру газа за ТВД:
	 	$$T_{ТВД}^* = T_г^*
			 \left\lbrace
			 	1 -
			 	\left[
			 		1 -
			 			\left(
			 				\frac{p_{ТВД}^*}{p_г^*}
			 			\right) ^ \frac{k_{г \/\ ТВД}}{k_{г \/\ ТВД} - 1}
			 	\right] \eta_{ТВД}
			 \right\rbrace =
			 <-<.HPTurbine.TIn>->
			 \left\lbrace
			 	1 -
			 	\left[
			 		1 -
			 			\left(
			 				\frac{<-<.HPTurbine.POut>->}{<-<.HPTurbine.PIn>->}
			 			\right) ^ \frac{<-<.HPTurbine.GasData.KMean>->}{<-<.HPTurbine.GasData.KMean>-> - 1}
			 	\right] \cdot <-<.HPTurbine.Eta>->
			 \right\rbrace = <-<.HPTurbine.TOut>-> \/\ К$$
	\item Определим давление перед ТНД:
		$$p_{0 \/\ ТНД}^* = p_{ТВД}^*\sigma_{ТВД} = <-<.HPTurbine.POut>-> \cdot <-<.HPTurbinePipe.Sigma>-> = <-<.LPTurbine.PIn>-> \/\ МПа$$



	\item Определим удельный расход через ТНД:
		% $$g_{ТНД} = \left( 1 + g_m \right) \left( 1 - g_{ут} - g_{охл} \right) =
		% 	\left( 1 + <-<.Burner.FuelMassRateRel>-> \right) \left( 1 - <-<.HPTurbine.LeakMassRateRel>-> - <-<.HPTurbine.CoolMassRateRel>-> \right) = <-<.HPTurbine.MassRateRel>->$$
	\item Определим удельную работу ТНД:
		$$L_{ТНД} = \frac{L_{КНД}}{g_{ТНД}\eta_{м \/\ НД}} = \frac{<-<.LPCompressor.Labour>-> \cdot 10^6}{<-<.LPTurbine.MassRateRel>-> \cdot <-<.LPShaft.Eta>-> } = <-<.LPTurbine.Labour>-> \cdot 10^6 \/\ Дж/кг$$
	\item Определим среднюю теплоемкость газа в процессе расширения газа в турбине, принимая показатель адиабаты газа $k_{г \/\ ТНД} = <-<.LPTurbine.GasData.KMean>->$:
		$$c_{pг \/\ ТВД} = \frac{k_{г \/\ ТВД}}{k_{г \/\ ТВД} - 1} R_г =
			\frac{<-<.LPTurbine.GasData.KMean>->}{<-<.LPTurbine.GasData.KMean>-> - 1} \cdot <-<.LPTurbine.GasData.R>-> = <-<.LPTurbine.GasData.CpMean>-> \/\ Дж/(кг \cdot К) $$
	\item Определим давление воздуха за ТНД:
		$$p_{ТНД}^* = p_{0 \/\ ТНД}^*
			\left[
				1 - \frac{L_{КНД}}{c_{pг \/\ ТНД} T_г \eta_{ТНД}}
			\right] ^ \frac{k_{г \/\ ТНД}}{k_{г \/\ ТНД} - 1} =
			<-<.LPTurbine.PIn>->
			\left[
				1 - \frac{<-<.LPCompressor.Labour>-> \cdot 10^6}
				{<-<.LPTurbine.GasData.CpMean>-> \cdot <-<.LPTurbine.TIn>-> \cdot <-<.LPTurbine.Eta>->}
			\right] ^ \frac{<-<.LPTurbine.GasData.KMean>->}{<-<.LPTurbine.GasData.KMean>-> - 1} =
			 <-<.LPTurbine.POut>-> \/\ МПа$$
	\item Определим температуру газа за ТНД:
	 	$$T_{ТНД}^* = T_{ТВД}^*
			 \left\lbrace
			 	1 -
			 	\left[
			 		1 -
			 			\left(
			 				\frac{p_{ТНД}^*}{p_{ТНД \/\ 0}^*}
			 			\right) ^ \frac{k_{г \/\ ТНД}}{k_{г \/\ ТНД} - 1}
			 	\right] \eta_{ТНД}
			 \right\rbrace =
			 <-<.LPTurbine.TIn>->
			 \left\lbrace
			 	1 -
			 	\left[
			 		1 -
			 			\left(
			 				\frac{<-<.LPTurbine.POut>->}{<-<.LPTurbine.PIn>->}
			 			\right) ^ \frac{<-<.LPTurbine.GasData.KMean>->}{<-<.LPTurbine.GasData.KMean>-> - 1}
			 	\right] \cdot <-<.LPTurbine.Eta>->
			 \right\rbrace = <-<.LPTurbine.TOut>-> \/\ К$$
	\item Определим давление перед свободной турбиной:
		$$p_{0 \/\ ТС}^* = p_{ТНД}^*\sigma_{ТНД} = <-<.LPTurbine.POut>-> \cdot <-<.LPTurbinePipe.Sigma>-> = <-<.FreeTurbine.PIn>-> \/\ МПа$$


	\item Определим удельный расход через силовую турбину:
		% $$g_{ТС} = \left( 1 + g_m \right) \left( 1 - g_{ут} - g_{охл} \right) =
		% 	\left( 1 + <-<.Burner.FuelMassRateRel>-> \right) \left( 1 - <-<.HPTurbine.LeakMassRateRel>-> - <-<.HPTurbine.CoolMassRateRel>-> \right) = <-<.HPTurbine.MassRateRel>->$$
	\item Определим давление торможения на выходе из свободной турбины $p_{ТС}^*$:
		$$p_{ТС}^* = p_a / \sigma_{вых} = <-<.GasSource.P>-> \cdot <-<.OutletPipe.Sigma>-> = <-<.FreeTurbine.POut>-> \/\ МПа$$
	\item Зададим значение приведенной скорости на выходе из свободной турбины:
		$$\lambda_{вых} = <-<.FreeTurbine.LambdaOut>->$$
	\item Определим статическое давление на выходе из свободной турбины, принимая показатель адиабаты газа на выходе из свбодной турбины $k_{СТ \/\ вых} = <-<.FreeTurbine.OutletGasData.K>->$:
		$$p_{ТС} = \frac{
			p_{ТС}^*
		}{
			\pi \left( \lambda_{вых}, \/\ k_{СТ \/\ вых} \right)
		} = \frac{
			<-<.FreeTurbine.POut>->
		}{
			\pi \left( <-<.FreeTurbine.LambdaOut>->, \/\ <-<.FreeTurbine.OutletGasData.K>-> \right)
		}$$
	\item Определим статическую температуру на выходе из силовой турбины, принимая показатель адиабаты газа $k_{г \/\ ТС} = <-<.FreeTurbine.GasData.KMean>->$::
		$$T_{СТ} = T_{ТНД}^*
		 \left\lbrace
		 	1 -
		 	\left[
		 		1 -
		 			\left(
		 				\frac{p_{0 \/\ СТ}^*}{p_{ТС}}
		 			\right) ^ \frac{k_{г \/\ ТС}}{k_{г \/\ ТС} - 1}
		 	\right] \eta_т
		 \right\rbrace =
		 <-<.FreeTurbine.TIn>->
		 \left\lbrace
		 	1 -
		 	\left[
		 		1 -
		 			\left(
		 				\frac{<-<.FreeTurbine.PIn>->}{<-<.FreeTurbine.POutStat>->}
		 			\right) ^ \frac{<-<.FreeTurbine.GasData.KMean>->}{<-<.FreeTurbine.GasData.KMean>-> - 1}
		 	\right] \cdot <-<.FreeTurbine.Eta>->
		 \right\rbrace = <-<.FreeTurbine.TOutStat>-> \/\ К$$
	\item Определим температуру торможения на выходе из силовой турбины:
		$$T_{СТ}^* = 
			\frac{T_{СТ}}{\tau\left( \lambda_{вых}, \/\ k_{СТ \/\ вых} \right)} = 
			\frac{T_{СТ}}{\tau\left( <-<.FreeTurbine.LambdaOut>->, \/\ <-<.FreeTurbine.OutletGasData.K>-> \right)} =
			= <-<.FreeTurbine.TOut>-> \/\ К$$
	\item Определим значение теплоемкости газа в свободной турбине:
		$$c_{p \/\ ТС} = 
			\frac{k_{г \/\ ТС}}{k_{г \/\ ТС} - 1} = 
			\frac{<-<.FreeTurbine.GasData.KMean>->}{<-<.FreeTurbine.GasData.KMean>-> - 1} = <-<.FreeTurbine.GasData.CpMean>-> \/\ Дж / \left( кг \cdot К \right)$$
	\item Определим удельную работу силовой турбины:
		$$L_{ТС} = c_{p \/\ ТС} \left( T_{ТНД}^* - T_{СТ}^* \right) = 
			<-<.FreeTurbine.GasData.CpMean>-> \left( <-<.FreeTurbine.TIn>-> - <-<.FreeTurbine.TOut>-> \right) = 
			<-<.FreeTurbine.Labour>-> \/\ Дж/кг$$


	\item Определим удельную мощность ГТД:
		$$N_{e \/\ уд} = L_{ТС} \/\ g_{ТС} =
			<-<.FreeTurbine.Labour>-> \cdot 10^6 \cdot <-<.FreeTurbine.MassRateRel>-> =
			<-<.EngineLabour>-> \cdot 10^6 Дж/кг$$
	\item Определим экономичность ГТД:
		$$C_e = \frac{3600}{N_{e уд}} g_{ТС} =
			\frac{3600}{<-<.FreeTurbine.Labour>-> \cdot 10^6} \cdot <-<.FreeTurbine.MassRateRel>-> =
			<-<.Ce>-> \cdot кг/\left( кВт/ч \right)$$
	\item Определим КПД ГТД:
		$$\eta_e = \frac{3600}{C_e Q_н^р} =
			\frac{3600}{<-<.Ce>-> \cdot <-<.Burner.Fuel.QLower>-> }
			= <.Eta>$$
	\item Определим расход воздуха:
		$$G_в = \frac{N_e}{N_{e \/\ уд}} =
			\frac{<-<.Ne>->}{<-<.EngineLabour>->} =
			<-<.MassRate>-> кг/с$$
\end{enumerate}