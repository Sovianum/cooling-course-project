\subsection{Расчет цикла}
В данном расчете учет изменения теплофизических свойств рабочего тела в зависимости от его температуры производился
путем итерирования на каждом этапе расчета до тех пор, пока изменение искомого теплофизического свойства (теплоемкости или
показателя адиабаты) не составляло менее 0.1\% в сравнении с результатами предыдущей итерации. Ниже везде используются
значения теплофизический свойств на последнем этапе итерационных расчетов.

\begin{enumerate}
	\item Определим давление за входным устройством:
		$$p_{вх}^* = \sigma_{вх}  p_a = <-<.InletPipe.Sigma | Round2>-> \cdot <-<.GasSource.P | DivideE6 | Round3>-> = <-<.InletPipe.POut | DivideE6 | Round3>-> \/\ МПа$$
	\item Определим давление за КНД:
		$$p_{кнд}^* = \pi_{кнд} p_{вх}^* = <-<.LPCompressor.Pi | Round1>-> \cdot <-<.LPCompressor.PIn | DivideE6 | Round3>-> = <-<.LPCompressor.POut | DivideE6 | Round3>-> \/\ МПа$$
	\item Определим адиабатический КПД КНД $\eta_{кнд}$, принимая показатель адиабаты воздуха $k_{в \/\ кнд} = <-<.LPCompressor.GasData.KMean | Round2>->$:
	    $$
	    	\eta_{кнд} = \frac{
		        \pi_{кнд}^\frac{
		            k_{в \/\ кнд} - 1
		        }{
		            k_{в \/\ кнд}
	            } - 1
		    }{
		        \pi_{кнд}^\frac{
		            k_{в \/\ кнд} - 1
	            }{
	                k_{в \/\ кнд} \cdot \eta_{пол \/\ кнд}
	            } - 1
		    } = \frac{
	            <-<.LPCompressor.Pi | Round1>->^\frac{
	                <-<.LPCompressor.GasData.KMean | Round2>-> - 1
	            }{
	                <-<.LPCompressor.GasData.KMean | Round2>->
	            } - 1
	        }{
	            <-<.LPCompressor.Pi | Round1>->^\frac{
	                <-<.LPCompressor.GasData.KMean | Round2>-> - 1
	            }{
	                <-<.LPCompressor.GasData.KMean | Round2>-> \cdot <-<.LPCompressor.EtaPol | Round3>->
	            } - 1
	        } = <-<.LPCompressor.Eta | Round2>->
	    $$
	\item Определим температуру газа за КНД:
		$$T_{КНД}^* = T_a 
		\left[ 
			1 + \frac{
				\pi_к^{
					\frac{
						k_{в \/\ кнд} - 1
					}{
						k_{в \/\ кнд}
					}
				} - 1
			}{
				\eta_{кнд}
			}
		\right] =
			<-<.LPCompressor.TIn | Round1>-> 
		\left[
			1 + \frac{
				{<-<.LPCompressor.Pi | Round1>->}^{
					\frac{
						<-<.LPCompressor.GasData.KMean | Round2>-> - 1
					}{
						<-<.LPCompressor.GasData.KMean | Round2>->
					}
				} - 1
			}{
				<-<.LPCompressor.Eta | Round2>->
			}
		\right] = <-<.LPCompressor.TOut | Round1>-> \/\ К$$
	\item Используя найденный показатель адиабаты воздуха, определим теплоемкость воздуха в процессе сжатия воздуха в КНД:
		$$c_{pв \/\ кнд} = \frac{
			k_{в \/\ кнд}
		}{
			k_{в \/\ кнд} - 1
		} R_в = \frac{
			<-<.LPCompressor.GasData.KMean | Round2>->
		}{
			<-<.LPCompressor.GasData.KMean | Round2>-> - 1
		} \cdot <-<.LPCompressor.GasData.R | Round1>-> = <-<.LPCompressor.GasData.CpMean | Round1>-> \/\ Дж/(кг \cdot К)$$
	\item Определим работу КНД:
		$$L_{КНД} = c_{pв \/\ кнд} \left( T_{кнд}^* - T_a \right) =
			<-<.LPCompressor.GasData.CpMean | Round1>-> \cdot \left(<-<.LPCompressor.TOut | Round1>-> - <-<.LPCompressor.TIn | Round1>->\right) =
			<-<.LPCompressor.Labour | DivideE6 | Round3>-> \cdot 10^6 \/\ Дж/кг $$
	\item Определим давление перед КВД:
		$$p_{0 \/\ квд}^* = \sigma_{кнд} p_{кнд}^* = <-<.LPCompressorPipe.Sigma | Round2>-> \cdot <-<.LPCompressor.POut | DivideE6 | Round3>-> = <-<.HPCompressor.PIn | DivideE6 | Round3>-> \/\ МПа$$
	\item Определим давление за КВД:
		$$ p_{квд}^* = \pi_{квд} p_{0 \/\ квд}^* = <-<.HPCompressor.Pi | Round1>-> \cdot <-<.HPCompressor.PIn | DivideE6 | Round3>-> = <-<.HPCompressor.POut | DivideE6 | Round3>-> \/\ МПа $$
	\item Определим адиабатический КПД КВД $\eta_{квд}$, принимая показатель адиабаты воздуха $k_{в \/\ КВД} = <-<.HPCompressor.GasData.KMean | Round2>->$:
	    $$
	    	\eta_{квд} = \frac{
		        \pi_{квд}^\frac{
		            k_{в \/\ квд} - 1
		        }{
		            k_{в \/\ квд}
	            } - 1
		    }{
		        \pi_{кнд}^\frac{
		            k_{в \/\ квд} - 1
	            }{
	                k_{в \/\ квд} \cdot \eta_{пол \/\ квд}
	            } - 1
		    } = \frac{
	            <-<.HPCompressor.Pi | Round1>->^\frac{
	                <-<.HPCompressor.GasData.KMean | Round2>-> - 1
	            }{
	                <-<.HPCompressor.GasData.KMean | Round2>->
	            } - 1
	        }{
	            <-<.HPCompressor.Pi | Round1>->^\frac{
	                <-<.HPCompressor.GasData.KMean | Round2>-> - 1
	            }{
	                <-<.HPCompressor.GasData.KMean | Round2>-> \cdot <-<.HPCompressor.EtaPol | Round3>->
	            } - 1
	        } = <-<.HPCompressor.Eta | Round2>->
	    $$
	\item Определим температуру газа за КВД:
		$$T_{квд}^* = T_{кнд}^*
		\left[ 
			1 + \frac{
				\pi_к^{
					\frac{
						k_в - 1
					}{
						k_в
					}
				} - 1
			}{
				\eta_{квд}
			}
		\right] =
			<-<.HPCompressor.TIn | Round1>-> 
		\left[
			1 + \frac{
				{<-<.HPCompressor.Pi | Round1>->}^{
					\frac{
						<-<.HPCompressor.GasData.KMean | Round2>-> - 1
					}{
						<-<.HPCompressor.GasData.KMean | Round2>->
					}
				} - 1
			}{
				<-<.HPCompressor.Eta | Round2>->
			}
		\right] = <-<.HPCompressor.TOut | Round1>-> \/\ К$$
	\item Используя найденный показатель адиабаты воздуха, определим теплоемкость воздуха в процессе сжатия воздуха в КВД:
		$$c_{pв \/\ квд} = \frac{
			k_{в \/\ квд}
		}{
			k_{в \/\ квд} - 1
		} R_в = \frac{
			<-<.HPCompressor.GasData.KMean | Round2>->
		}{
			<-<.HPCompressor.GasData.KMean | Round2>-> - 1
		} \cdot <-<.HPCompressor.GasData.R | Round1>-> = <-<.HPCompressor.GasData.CpMean | Round1>-> \/\ Дж/(кг \cdot К)$$
	\item Определим работу КВД:
		$$L_{квд} = c_{pв \/\ квд} \left( T_{квд}^* - T_{кнд}^* \right) =
			<-<.HPCompressor.GasData.CpMean | Round1>-> \cdot \left(<-<.HPCompressor.TOut | Round1>-> - <-<.HPCompressor.TIn | Round1>->\right) =
			<-<.HPCompressor.Labour | DivideE6 | Round3>-> \cdot 10^6 \/\ Дж/кг $$
	\item Температура газа за камерой сгорания:
		$$T_г^* = <-<.Burner.Tg | Round>-> \/\ К$$
	\item Определим относительный расход топлива. Расчет носит итерационный характер. Ниже описана последняя итерация. Теплоемкость продуктов сгорания природного газа рассчитывается через показатель адиабаты и газовую постоянную газа. При этом газовая постоянная и истинный показатель адиабаты рассчитываются как средневзвешенное соответственных характеристик компонентов продуктов. При расчета приняты следующие значения:
	\begin{enumerate} % список значений для расчета удельного расхода топлива
		\item[1)] теплоемкость топлива:
			$$c_{pm} = <-<.Burner.Fuel.C | Round1>-> \/\ Дж / (кг \cdot К);$$
		\item[2)] температура подачи топлива:
			$$T_m = <-<.Burner.Fuel.TInit | Round1>-> \/\ К;$$
		\item[3)] температура определения теплофизических параметров веществ:
			$$T_0 = <-<.Burner.Fuel.T0 | Round1>-> \/\ К;$$
		\item[4)] истинная теплоемкость воздуха перед камерой сгорания:
			$$c_{pв \/\ г}\left( T_{КВД} \right) = <-<.Burner.AirDataInlet.Cp | Round1>-> \/\ Дж/(кг \cdot К);$$
		\item[5)] истинная теплоемкость воздуха при температуре определения теплофизических параметров веществ:
			$$c_{pв \/\ г}\left( T_0 \right) = <-<.Burner.AirData0.Cp | Round1>-> \/\ Дж/(кг \cdot К);$$
		\item[6)] низшая теплота сгорания топлива:
			$$Q_н^р = <-<.Burner.Fuel.QLower | DivideE3 | Round>-> \cdot 10^3 \/\ Дж / кг;$$
		\item[7)] полнота сгорания:
			$$\eta_г = <-<.Burner.Eta | Round2>->;$$
		\item[8)] масса воздуха, необходимая для сжигания 1 кг топлива:
			$$l_0 = <-<.Burner.Fuel.L0 | Round1>-> \/\ кг;$$
	\end{enumerate}
	
	\begin{enumerate}
		\item Зададимся коэффициентом избытка воздуха: $$\alpha = <-<.Burner.Alpha | Round2>->;$$
		\item Теплоемкость продуктов сгорания природного газа $c_{pг \/\ г}$ при данном значении коэффициента избытка воздуха при температуре $T_г$ составляет:
			$$c_{pг \/\ г}\left( T_г \right) = <-<.Burner.GasDataOutlet.Cp | Round1>-> \/\ Дж/(кг \cdot К);$$
		\item Теплоемкость продуктов сгорания природного газа $c_{pг \/\ г}$ при данном значении коэффициента избытка воздуха при температуре $T_0$ составляет:
			$$c_{pг \/\ г}\left( T_0 \right) = <-<.Burner.GasData0.Cp | Round1>-> \/\ Дж / (кг \cdot К);$$
		\item Определим относительный расход топлива:
			$$
				a = c_{pг \/\ г} \left( T_г \right) T_г - c_{pв \/\ г} \left( T_{квд} \right) T_{квд} = 
			$$
			$$
				= <-<.Burner.GasDataOutlet.Cp | Round1>-> \cdot <-<.Burner.Tg | Round1>-> -
				<-<.Burner.GasDataOutlet.Cp | Round1>-> \cdot <-<.HPCompressor.TOut | Round3>-> = 
				<-<.Burner.A | DivideE6 | Round3>-> \cdot 10^6 \/\ Дж/кг
			$$
			$$
				b = \left(
					c_{pг \/\ г}\left( T_0 \right) - c_{pв \/\ г}\left( T_0 \right) = 
				\right) T_0 = 
			$$
			$$
				= \left(
					<-<.Burner.GasData0.Cp | Round1>-> - <-<.Burner.AirData0.Cp | Round1>->
				\right) \cdot <-<.Burner.AirData0.T | Round1>-> = 
				<-<.Burner.B | DivideE3 | Round3>-> \cdot 10^3 \/\ Дж/кг
			$$
			$$
				c = c_{pг \/\ г} \left( T_г \right) T_г - c_{pг \/\ г} \left( T_0 \right) T_0 = 
			$$
			$$
				= <-<.Burner.GasDataOutlet.Cp | Round1>-> \cdot <-<.Burner.Tg | Round1>-> -
				<-<.Burner.GasData0.Cp | Round1>-> \cdot <-<.Burner.AirData0.T | Round1>-> = 
				<-<.Burner.C | DivideE6 | Round3>-> \cdot 10^6 \/\ Дж/кг
			$$
			$$
				d = c_{pm} \left( T_m - T_0 \right) = 
			$$
			$$
				= <-<.Burner.Fuel.C | Round1>-> \left( <-<.Burner.Fuel.TInit | Round1>-> - <-<.Burner.AirData0.T | Round1>-> \right) =
				<-<.Burner.D | Round>-> \/\ Дж/кг
			$$
			$$g_m = \frac{G_m}{G_в^г} =
				\frac{
					a - b
				}{
					Q_н^р \eta_г -
					c + d
				} = 
			$$
			$$
				= \frac{
					<-<.Burner.A | DivideE6 | Round3>-> \cdot 10^6 + <-<.Burner.B | Abs | DivideE3 | Round3>-> \cdot 10^3
				}{
					<-<.Burner.Fuel.QLower | DivideE3 | Round>-> \cdot 10^3 \cdot <-<.Burner.Eta>-> -
					<-<.Burner.C | DivideE3 | Round3>-> \cdot 10^6 + <-<.Burner.D | Round>->
				} = <-<.Burner.FuelMassRateRel | Round3>->
			$$
		\item Определим коэффициент избытка воздуха:
			$$\alpha^\prime = \frac{1}{g_m l_0} =
		\frac{1}{<-<.Burner.FuelMassRateRel | Round3>-> \cdot <-<.Burner.Fuel.L0 | Round1>->} = <-<.Burner.Alpha | Round2>->$$
	\end{enumerate}

	\item Определим удельный расход через ТВД:
		$$g_{твд} = \left( 1 + g_m \right) \left( 1 - g_{ут \/\ твд} - g_{охл \/\ твд} \right) = $$
		$$
		= \left(
		    1 + <-<.Burner.FuelMassRateRel | Round3>->
		\right) \left(
		    1 - <-<.HPTurbine.LeakMassRateRel | Abs | Round3>-> -
		    <-<.HPTurbine.CoolMassRateRel | Abs | Round3>->
        \right) = <-<.HPTurbine.MassRateRel | Round3>->$$
	\item Определим удельную работу ТВД:
		$$L_{твд} = \frac{L_{квд}}{g_{твд}\eta_{м \/\ вд}} = \frac{
			<-<.HPCompressor.Labour | DivideE6 | Round3>-> \cdot 10^6
		}{
			<-<.HPTurbine.MassRateRel | Round3>-> \cdot <-<.HPShaft.Eta | Round3>->
		} = <-<.HPTurbine.Labour | DivideE6 | Round3>-> \cdot 10^6 \/\ Дж/кг$$
	\item Определим давление газа перед ТВД:
		$$p_{г}^* = p_{тнд}^* \sigma_г = <-<.HPCompressor.POut | DivideE6 | Round3>-> \cdot <-<.Burner.Sigma | Round2>-> = <-<.HPTurbine.PIn | DivideE6 | Round3>-> \/\ МПа$$
	\item Определим среднюю теплоемкость газа в процессе расширения газа в турбине, принимая показатель адиабаты газа $k_{г \/\ твд} = <-<.HPTurbine.GasData.KMean | Round2>->$:
		$$c_{pг \/\ твд} = \frac{k_{г \/\ твд}}{k_{г \/\ твд} - 1} R_г =
			\frac{
				<-<.HPTurbine.GasData.KMean | Round2>->
			}{
				<-<.HPTurbine.GasData.KMean | Round2>-> - 1
			} \cdot <-<.HPTurbine.GasData.R | Round1>-> = <-<.HPTurbine.GasData.CpMean | Round1>-> \/\ Дж/(кг \cdot К) $$
	\item Определим давление воздуха за ТВД:
		$$p_{твд}^* = p_г^*
			\left[
				1 - \frac{L_{твд}}{c_{pг \/\ твд} T_г \eta_{твд}}
			\right] ^ \frac{k_{г \/\ твд}}{k_{г \/\ твд} - 1} =
		$$
		$$
			= <-<.HPTurbine.PIn | DivideE6 | Round3>->
			\left[
				1 - \frac{<-<.HPCompressor.Labour | DivideE6 | Round3>-> \cdot 10^6}
				{<-<.HPTurbine.GasData.CpMean | Round1>-> \cdot <-<.HPTurbine.TIn | Round1>-> \cdot <-<.HPTurbine.Eta | Round3>->}
			\right] ^ \frac{<-<.HPTurbine.GasData.KMean | Round2>->}{<-<.HPTurbine.GasData.KMean | Round2>-> - 1} =
			 <-<.HPTurbine.POut | DivideE6 | Round3>-> \/\ МПа
		$$
	\item Определим температуру газа за ТВД:
	 	$$
	 		T_{твд}^* = T_г^*
			\left\lbrace
			 	1 -
			 	\left[
			 		1 -
			 			\left(
			 				\frac{p_{твд}^*}{p_г^*}
			 			\right) ^ \frac{k_{г \/\ твд}}{k_{г \/\ твд} - 1}
			 	\right] \eta_{ТВД}
			\right\rbrace =
		$$
		$$
			= <-<.HPTurbine.TIn | Round1>->
			\left\lbrace
			 	1 -
			 	\left[
			 		1 -
			 			\left(
			 				\frac{<-<.HPTurbine.POut | DivideE6 | Round3>->}{<-<.HPTurbine.PIn | DivideE6 | Round3>->}
			 			\right) ^ \frac{<-<.HPTurbine.GasData.KMean | Round2>->}{<-<.HPTurbine.GasData.KMean | Round2>-> - 1}
			 	\right] \cdot <-<.HPTurbine.Eta | Round3>->
			\right\rbrace = <-<.HPTurbine.TOut | Round1>-> \/\ К
		$$
	\item Определим давление перед ТНД:
		$$p_{0 \/\ тнд}^* = p_{твд}^*\sigma_{твд} = <-<.HPTurbine.POut | DivideE6 | Round3>-> \cdot <-<.HPTurbinePipe.Sigma | Round2>-> = <-<.LPTurbine.PIn | DivideE6 | Round3>-> \/\ МПа$$

	\item Определим удельный расход через ТНД:
		 $$g_{тнд} = g_{твд} \left( 1 - g_{ут \/\ тнд} - g_{охл \/\ тнд} + g_{охл \/\ твд}\right) = $$
		 $$=<-<.HPTurbine.MassRateRel | Round3>-> \cdot
		 	\left(
		 	    1 - <-<.LPTurbine.LeakMassRateRel | Abs | Round3>-> -
		 	    <-<.LPTurbine.CoolMassRateRel | Abs | Round3>-> +
		 	    <-<.HPTurbine.CoolMassRateRel | Abs | Round3>->
		 	\right) = <-<.LPTurbine.MassRateRel | Round3>->$$
	\item Определим удельную работу ТНД:
		$$L_{тнд} = \frac{L_{кнд}}{g_{тнд}\eta_{м \/\ нд}} = \frac{
			<-<.LPCompressor.Labour | DivideE6 | Round3>-> \cdot 10^6
		}{
			<-<.LPTurbine.MassRateRel | Round3>-> \cdot <-<.LPShaft.Eta | Round2>->
		} = <-<.LPTurbine.Labour | DivideE6 | Round3>-> \cdot 10^6 \/\ Дж/кг$$
	\item Определим среднюю теплоемкость газа в процессе расширения газа в ТНД, принимая показатель адиабаты газа $k_{г \/\ тнд} = <-<.LPTurbine.GasData.KMean | Round2>->$:
		$$c_{pг \/\ тнд} = \frac{k_{г \/\ тнд}}{k_{г \/\ тнд} - 1} R_г =
			\frac{
				<-<.LPTurbine.GasData.KMean | Round2>->
			}{
				<-<.LPTurbine.GasData.KMean | Round2>-> - 1
			} \cdot <-<.LPTurbine.GasData.R | Round1>-> = <-<.LPTurbine.GasData.CpMean | Round1>-> \/\ Дж/(кг \cdot К) $$
	\item Определим давление воздуха за ТНД:
		$$
			p_{тнд}^* = p_{0 \/\ тнд}^*
				\left[
					1 - \frac{L_{тнд}}{c_{pг \/\ тнд} T_г \eta_{тнд}}
				\right] ^ \frac{k_{г \/\ тнд}}{k_{г \/\ тнд} - 1} =
		$$
		$$
			= <-<.LPTurbine.PIn | DivideE6 | Round3>->
				\left[
					1 - \frac{
						<-<.LPCompressor.Labour | DivideE6 | Round3>-> \cdot 10^6
					}
					{
						<-<.LPTurbine.GasData.CpMean | Round1>-> \cdot <-<.LPTurbine.TIn | Round1>-> \cdot <-<.LPTurbine.Eta | Round2>->
					}
				\right] ^ \frac{<-<.LPTurbine.GasData.KMean | Round2>->}{<-<.LPTurbine.GasData.KMean | Round2>-> - 1} =
				 <-<.LPTurbine.POut | DivideE6 | Round3>-> \/\ МПа
		$$
	\item Определим температуру газа за ТНД:
	 	$$
	 		T_{тнд}^* = T_{твд}^*
			\left\lbrace
			 	1 -
			 	\left[
			 		1 -
			 			\left(
			 				\frac{p_{тнд}^*}{p_{тнд \/\ 0}^*}
			 			\right) ^ \frac{k_{г \/\ тнд}}{k_{г \/\ тнд} - 1}
			 	\right] \eta_{тнд}
			\right\rbrace =
		$$
		$$
			= <-<.LPTurbine.TIn | Round1>->
			\left\lbrace
			 	1 -
			 	\left[
			 		1 -
			 			\left(
			 				\frac{<-<.LPTurbine.POut | DivideE6 | Round3>->}{<-<.LPTurbine.PIn | DivideE6 | Round3>->}
			 			\right) ^ \frac{<-<.LPTurbine.GasData.KMean | Round2>->}{<-<.LPTurbine.GasData.KMean | Round2>-> - 1}
			 	\right] \cdot <-<.LPTurbine.Eta | Round2>->
			\right\rbrace = <-<.LPTurbine.TOut | Round1>-> \/\ К
		$$
	\item Определим давление перед свободной турбиной:
		$$p_{0 \/\ тс}^* = p_{тнд}^*\sigma_{тнд} = <-<.LPTurbine.POut | DivideE6 | Round3>-> \cdot <-<.LPTurbinePipe.Sigma | Round2>-> = <-<.FreeTurbine.PIn | DivideE6 | Round3>-> \/\ МПа$$
	\item Определим удельный расход через силовую турбину:
	    $$g_{тс} = g_{тнд} \left( 1 - g_{ут \/\ тс} - g_{охл \/\ тс} \right) =
            <-<.LPTurbine.MassRateRel | Round3>-> \cdot
            \left(
                1 - <-<.FreeTurbine.LeakMassRateRel | Abs | Round3>-> -
                <-<.FreeTurbine.CoolMassRateRel | Abs |Round3>->
            \right) = <-<.FreeTurbine.MassRateRel | Round3>->$$
    \item Определим давление торможения на выходе из свободной турбины $p_{тс}^*$:
		$$p_{тс}^* = p_a / \sigma_{вых} = <-<.GasSource.P | DivideE6 | Round3>-> \cdot <-<.OutletPipe.Sigma | Round2>-> = <-<.FreeTurbine.POut | DivideE6 | Round3>-> \/\ МПа$$
	\item Зададим значение приведенной скорости на выходе из свободной турбины:
		$$\lambda_{вых} = <-<.FreeTurbine.LambdaOut | Round2>->$$
	\item Определим статическое давление на выходе из свободной турбины, принимая показатель адиабаты газа на выходе из свободной турбины $k_{тс \/\ вых} = <-<.FreeTurbine.OutletGasData.K | Round2>->$:
		$$p_{тс} = p_{тс}^* \cdot \pi \left( \lambda_{вых}, \/\ k_{тс \/\ вых} \right)
        =
			<-<.FreeTurbine.POut | DivideE6 | Round3>->
			\cdot \pi \left( <-<.FreeTurbine.LambdaOut | Round2>->, \/\ <-<.FreeTurbine.OutletGasData.K | Round2>-> \right)
        = <-<.FreeTurbine.POutStat | DivideE6 | Round3>-> \/\ МПа$$
	\item Определим статическую температуру на выходе из свободной турбины, принимая показатель адиабаты газа $k_{г \/\ тс} = <-<.FreeTurbine.GasData.KMean | Round2>->$::
		$$
			T_{тс} = T_{тнд}^*
			\left\lbrace
			 	1 -
			 	\left[
			 		1 -
			 			\left(
			 				\frac{p_{0 \/\ тс}^*}{p_{тс}}
			 			\right) ^ \frac{k_{г \/\ тс}}{k_{г \/\ тс} - 1}
			 	\right] \eta_{тс}
			\right\rbrace =
		$$
		$$
			= <-<.FreeTurbine.TIn | Round1>->
			\left\lbrace
			 	1 -
			 	\left[
			 		1 -
			 			\left(
			 				\frac{
			 					<-<.FreeTurbine.PIn | DivideE6 | Round3>->
			 				}{
			 					<-<.FreeTurbine.POutStat | DivideE6 | Round3>->
			 				}
			 			\right) ^ \frac{<-<.FreeTurbine.GasData.KMean | Round2>->}{<-<.FreeTurbine.GasData.KMean | Round2>-> - 1}
			 	\right] \cdot <-<.FreeTurbine.Eta | Round2>->
			\right\rbrace = <-<.FreeTurbine.TOutStat | Round1>-> \/\ К
		$$
	\item Определим температуру торможения на выходе из силовой турбины:
		$$T_{тс}^* = 
			\frac{T_{тс}}{\tau\left( \lambda_{вых}, \/\ k_{тс \/\ вых} \right)} =
			\frac{T_{тс}}{\tau\left( <-<.FreeTurbine.LambdaOut | Round2>->, \/\ <-<.FreeTurbine.OutletGasData.K | Round2>-> \right)} =
			= <-<.FreeTurbine.TOut | Round1>-> \/\ К$$
	\item Определим значение теплоемкости газа в свободной турбине:
		$$c_{p \/\ тс} = 
			\frac{k_{г \/\ тс}}{k_{г \/\ тс} - 1} = 
			\frac{<-<.FreeTurbine.GasData.KMean | Round2>->}{<-<.FreeTurbine.GasData.KMean | Round2>-> - 1} = <-<.FreeTurbine.GasData.CpMean | Round1>-> \/\ Дж / \left( кг \cdot К \right)$$
	\item Определим удельную работу силовой турбины:
		$$L_{тс} = c_{p \/\ тс} \left( T_{тнд}^* - T_{тс}^* \right) = 
			<-<.FreeTurbine.GasData.CpMean | Round1>-> \cdot \left( <-<.FreeTurbine.TIn | Round1>-> - <-<.FreeTurbine.TOut | Round1>-> \right) =
			<-<.FreeTurbine.Labour | DivideE6 | Round3>-> \cdot 10^6\/\ Дж/кг$$
	\item Определим удельную работу ГТД:
		$$L = L_{тс} \/\ g_{тс} =
			<-<.FreeTurbine.Labour | DivideE6 | Round2>-> \cdot 10^6 \cdot <-<.FreeTurbine.MassRateRel | Round3>-> =
			<-<.EngineLabour | DivideE6 | Round3>-> \cdot 10^6 Дж/кг$$
	\item Определим экономичность ГТД:
		$$C_e = \frac{3600}{N_{e уд}} g_{тс} =
			\frac{3600}{<-<.FreeTurbine.Labour | DivideE6 | Round3>-> \cdot 10^6} \cdot <-<.FreeTurbine.MassRateRel | Round2>-> =
			<-<.Ce | MultiplyE3 | Round3>-> \cdot 10^{-3} кг/\left( кВт/ч \right)$$
	\item Определим КПД ГТД:
		$$\eta_e = \frac{3600}{C_e Q_н^р} =
			\frac{3600}{<-<.Ce | MultiplyE3 | Round3>-> \cdot 10^{-3} \cdot <-<.Burner.Fuel.QLower | DivideE6 | Round3>-> }
			= <-<.Eta | Round3>->$$
	\item Определим потребную мощность ГТД:
		$$
			N = N_e / \eta_р = <-<.Ne | DivideE3 | Round>-> \cdot 10^3 \cdot \ <-<.EtaR | Round2>-> = <-<.NeMech | DivideE3 | Round>-> \cdot 10^3 \/\ Вт
		$$
	\item Определим расход воздуха:
		$$G_в = \frac{N}{L} =
			\frac{<-<.NeMech | DivideE3 | Round>-> \cdot 10^3}{<-<.EngineLabour | DivideE6 | Round3>-> \cdot 10^6} =
			<-<.MassRate | Round1>-> \/\ кг/с$$
\end{enumerate}