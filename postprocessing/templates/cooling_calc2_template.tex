\section{Расчет профиля температур}

\begin{enumerate}
	\item Определим коэффициент теплоотдачи от газа на входной кромке лопатки $\alpha_{г.вх.кр.}$:
		$$
			\alpha_{г.вх.кр.} = 0.74 \frac{
				\lambda_г
			}{
				d_{вх.кр.}
			}\sqrt{
				\frac{
					\rho_г \cdot c_a \cdot d_{вх.кр.}
				}{
					\mu_г
				}
			} =
		$$
		$$
			= 0.74 \frac{
				<-<.Gas.LambdaGas | MultiplyE3 | Round1>-> \cdot 10^{-3}
			}{
				<-<.Geom.DInlet | MultiplyE3 | Round2>-> \cdot 10^{-3}
			}\sqrt{
				\frac{
					<-<.Gas.RhoGas | Round1>-> \cdot 
					<-<.Gas.Ca | Round1>-> \cdot 
					<-<.Geom.DInlet | MultiplyE3 | Round2>-> \cdot 10^{-3}
				}{
					<-<.Gas.MuGas | MultiplyE6 | Round1>-> \cdot 10^{-6}
				}
			} = <-<.Gas.AlphaGasInlet | Round1>-> \/\ Вт/\left( м^2 \cdot К\right)
		$$
	\item Определим коэффициент теплоотдачи на спинке на расстоянии $\frac{1}{3} b_a$ $\alpha_{г.вых.кр.}$:
		$$
			\alpha_{г.вых.кр.} = 1.5 \alpha_г = 
			1.5 \cdot <-<.Gas.AlphaMean | Round1>-> = <-<.Gas.AlphaGasOutlet | Round1>-> Вт/\left( м^2 \cdot К\right)
		$$
	\item Определим коэффициент теплоотдачи на остальной выпуклой части (спинке) $\alpha_{г.сп.}$:
		$$
			\alpha_{г.сп.} = 0.6 \alpha_г = 0.6 \cdot <-<.Gas.AlphaMean | Round1>-> = <-<.Gas.AlphaGasSS | Round1>-> Вт/\left( м^2 \cdot К\right)
		$$
	\item Определим коэффициет теплоотдачи на вокнутой части профиля (корыте) $\alpha_{г.кор.}$:
		$$
			\alpha_{г.кор.} = \alpha_г = <-<.Gas.AlphaMean | Round1>-> = <-<.Gas.AlphaGasPS | Round1>-> Вт/\left( м^2 \cdot К\right)
		$$
	\item Коэффициент теплоотдачи от стенки к охлаждающему воздуху зависит от его температуры и определяется следующим уравнением $\alpha_{в}$:
		$$
			\alpha_{в} = 0.02 \cdot \frac{
				\lambda_{в}
			}{
				2\delta
			} \left( 
				\frac{
					G_в
				}{
					l
				} \cdot \frac{
					1
				}{
					\mu_{в}
				}
			\right)^{0.8}
		$$
	\item Уравнение теплообмена между охлаждающим воздухом и газом имеет вид:
		$$
			\frac{d\theta}{dx} = \frac{
				2
			}{
				G_в C_{p \/\ в}
			} \frac{
				k_x
			}{
				\alpha_г
			} \left( 
				T_г^* - \theta
			\right),
		$$
	где $k_x$ - коэффициент теплопередачи, определяемый уравнением
		$$
			k_x = \frac{1}{
				\frac{1}{
					\alpha_г
				} + 
				\frac{1}{
					\alpha_в
				} + 
				\frac{\Delta}{\lambda_м}
			}
		$$
	Численно решая уравнение теплообмена, поулчим распределение параметров по спинке и корыту.
	Распределение параметров газа по спинке:
		\begin{longtable}{|c|c|c|c|c|c|}
		\hline
		\textbf{№} &
		\textbf{$x, \/\ м$} & 
		\textbf{$\alpha_г \/\ Вт/\left(м^2 \cdot К\right)$} & 
		\textbf{$\alpha_в \/\ Вт/\left(м^2 \cdot К\right)$} & 
		\textbf{$\theta_x, \/\ К$} & 
		\textbf{$T_{ст.x}, \/\ К$} 
		\\ \hline
		<-<range .Gas.SSRows>->
			<-<.Id>-> & 
			<-<.X | MultiplyE3 | Round3>-> & 
			<-<.AlphaGas | Round1>-> & 
			<-<.AlphaAir | Round1>-> &
			<-<.TAir | Round1>-> & 
			<-<.TWall | Round1>->
			\\\hline
		<-<end>->
		\end{longtable}

	Распределение параметров газа по корыту:
		\begin{longtable}{|c|c|c|c|c|c|}
		\hline
		\textbf{№} &
		\textbf{$x, \/\ 10^{-3} м$} & 
		\textbf{$\alpha_г \/\ Вт/\left(м^2 \cdot К\right)$} & 
		\textbf{$\alpha_в \/\ Вт/\left(м^2 \cdot К\right)$} & 
		\textbf{$\theta_x, \/\ К$} & 
		\textbf{$T_{ст.x}, \/\ К$} 
		\\ \hline
		<-<range .Gas.PSRows>->
			<-<.Id>-> & 
			<-<.X | MultiplyE3 | Round3>-> & 
			<-<.AlphaGas | Round1>-> & 
			<-<.AlphaAir | Round1>-> &
			<-<.TAir | Round1>-> & 
			<-<.TWall | Round1>->  
			\\\hline
		<-<end>->	
		\end{longtable}

\end{enumerate}
