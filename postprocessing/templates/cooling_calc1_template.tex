\section{Расчет расхода охлаждающего воздуха}

% Добавить исходные данные
Исходные данные:
\begin{center}
	\begin{tabular}{|p{7cm}|c|c|c|}
		\hline
		\textbf{Величина} & \textbf{Обозначение} & \textbf{Размерность} & \textbf{Значение} \\ \hline
		Температура газа & $T_г$ & К & $<-<.Gas.Tg | Round1>->$ \\ \hline
		Длина лопатки & $l$ & м & $<-<.Geom.BladeLength | MultiplyE3 | Round1>-> \cdot 10^{-3}$ \\ \hline
		Осевая проекция хорды & $b_a$ & м & $<-<.Geom.ChordProjection | MultiplyE3 | Round1>-> \cdot 10^{-3}$ \\ \hline
		Поверхность лопатки, соприкасающаяся с газом & $f$ & $м^2$ & $<-<.Geom.BladeArea | MultiplyE3 | Round1>-> \cdot 10^{-3}$ \\ \hline
		Периметр профиля & $u$ & $м$ & $<-<.Geom.Perimeter | MultiplyE3 | Round1>-> \cdot 10^{-3}$ \\ \hline
		Толщина стенки & $\Delta$ & $м$ & $<-<.Geom.WallThk | MultiplyE3 | Round1>-> \cdot 10^{-3}$ \\ \hline
		Средняя температура наружной поверхности лопатки & $T_{ст}$ & $К$ & $<-<.Metal.TWallOuter | Round1>->$ \\ \hline
		Плотность газа & $\rho_г$ & $кг/м^3$ & $<-<.Gas.DensityGas | Round2>->$ \\ \hline
		Осевая скорость & $c_a$ & $м/с$ & $<-<.Gas.CaGas | Round1>->$ \\ \hline
	\end{tabular}
\end{center}

 \begin{enumerate}
 	\item Определим число $Re$ для газа ($\mu_г = Па \cdot с$):
 		$$
 			Re_г = \frac{
 				\rho_г \cdot c_a \cdot b_a
 			}{
 				\mu_г
 			} = \frac{
 				<-<.Gas.DensityGas | Round2>-> \cdot <-<.Gas.CaGas | Round1>-> \cdot <-<.Geom.ChordProjection | MultiplyE3 | Round1>-> \cdot 10^{-3}
 			}{
 				<-<.Gas.MuGas | MultiplyE6 | Round2>-> \cdot 10^{-6} 
 			} = <-<.Gas.ReGas | DivideE3 | Round>-> \cdot 10^3
 		$$
 	\item Определим число $Nu$ для газа:
 		$$
 			Nu = A \cdot Re_г^{0.68} = 
 			<-<.Gas.NuCoef | Round3>-> \cdot \left(
 				<-<.Gas.ReGas | DivideE3 | Round>-> \cdot 10^3
			\right)^{0.68} = <-<.Gas.NuGas | Round>->
 		$$
 	\item Определим средний коэффициент теплоотдачи от газа к лопатке:
 		$$
 			\alpha_г = Nu \frac{\lambda_г}{b_a} = 
 			<-<.Gas.NuGas | Round>-> \cdot \frac{
 				<-<.Gas.LambdaGas | MultiplyE3 | Round1>-> \cdot 10^{-3}
 			}{
 				<-<.Geom.ChordProjection | MultiplyE3 | Round1>-> \cdot 10^{-3}
 			} = <-<.Gas.AlphaGas | Round1>-> \/\ Вт / \left( м^2 \cdot К \right)
 		$$
 	\item Определим тепловой поток в сопловую лопатку:
 		$$
 			Q_л = \alpha_г u l \left( T_г - T_{ст} \right) = 
		$$
		$$
 			=<-<.Gas.AlphaGas | Round1>-> \cdot 
 			<-<.Geom.Perimeter | MultiplyE3 | Round1>-> \cdot 10^{-3} \cdot 
 			<-<.Geom.BladeLength | MultiplyE3 | Round1>-> \cdot 10^{-3} \cdot 
 			\left( 
 				<-<.Gas.Tg | Round1>-> - <-<.Metal.TWallOuter>-> 
			\right) = <-<.Gas.Heat | DivideE3 | Round1>-> \cdot 10^3 \/\ Вт 
 		$$
 	\item Определим падение температуры в тенке лопатки:
 		$$
 			\Delta T_{ст} = \frac{
 				Q_л \cdot \Delta
 			}{
 				f \cdot \lambda_м
 			} = \frac{
 				<-<.Gas.Heat | DivideE3 | Round1>-> \cdot 10^3 \cdot <-<.Geom.WallThk | MultiplyE3 | Round1>-> \cdot 10^{-3}
 			}{
 				<-<.Geom.BladeArea | MultiplyE3 | Round1>-> \cdot 10^{-3} \cdot <-<.Metal.LambdaM | Round1>->
 			} = <-<.Metal.DTWall | Round1>-> \/\ К 
 		$$
 		($
 			\lambda_м = <-<.Metal.LambdaM | Round1>-> \/\ Вт / \left( м \cdot К\right)
 		$ для СПЛАВА при $
 			T_{ср} = T_{ст} - \frac{\Delta T_{ст}}{2} = <-<.Metal.TWallOuter | Round1>-> - \frac{<-<.Metal.DTWall | Round1>->}{2} = <-<.Metal.TWallMean | Round1>-> \/\ К
 		$)
 	\item Определим температуру внутренней поверхности стенки лопатки:
 		$$
 			T_{вн} = T_{ст} - \Delta T_{ст} = <-<.Metal.TWallOuter | Round1>-> - <-<.Metal.DTWall | Round1>-> = <-<.Metal.TWallInner | Round1>-> К
 		$$
 	\item Задаваясь рядом значений расходов охлаждающего воздуха, определим зависимость зазора в лопатке $\delta$ от расхода охлаждающего воздуха:
 		$$
 			\delta = \varepsilon G_в^{0.8} \left( 
 				D - \frac{
 					f
 				}{
 					7200 \cdot G_в \cdot c_p
 				}
 			\right),
 		$$
 		где 
		$$
			D = \frac{
				1
			}{
				\alpha_г
			} \cdot \frac {
				T_г - \theta_0
			}{
				T_г - T_{ст}
			} - \frac{
				1
			}{
				\alpha_г
			} - \frac{
				\Delta
			}{
				\lambda_м
			};
		$$
		$$
			\epsilon = 0.01 \cdot \lambda \left( 
				\frac{
					1
				}{
					l \mu
				}
			\right)^{0.8}
		$$

 	Результаты расчета сведены в таблицу:
		\begin{center}
			\begin{tabular}{|c|c|c|c|c|}
				\hline
				\textbf{№} & 
				\textbf{$G_в, \/\ кг/c$} & 
				\textbf{$D$} & 
				\textbf{$\epsilon$} & 
				\textbf{$\delta$} \\\hline
				<-<range .Gas.TableRows>->
					<-<.Id>-> & 
					$<-<.AirMassRate | Round2>->$ & 
					$<-<.DCoef | MultiplyE3 | Round3>-> \cdot 10^{-3}$ & 
					$<-<.EpsCoef | Round2>->$ & 
					$<-<.AirGap | MultiplyE3 | Round1>-> \cdot 10^-3$ 
					\\\hline
				<-<end>->
			\end{tabular}
		\end{center}

 \end{enumerate}