\documentclass[14pt]{extarticle}
\usepackage{mathtext}
\usepackage[T2A]{fontenc}
\usepackage[utf8]{inputenc}
\usepackage[russian]{babel}
\usepackage{amsmath}
\usepackage{amsfonts}
\usepackage{amssymb}
\usepackage{graphicx}
\usepackage[top=20mm, bottom=20mm, left=30mm, right=15mm]{geometry}
\usepackage{color}
\usepackage{gensymb}
\usepackage{wrapfig}
\usepackage{float}
\usepackage{longtable}

\usepackage{enumitem}
\setlist[enumerate]{label*=\arabic*.}

\usepackage{indentfirst}

\usepackage{titlesec}
\newcommand{\sectionbreak}{\clearpage}
\renewcommand{\baselinestretch}{1.5}

\graphicspath{ {/home/artem/Documents/University/Research/Report/images/} }

\begin{document}
\begin{titlepage}	% Титульник
	\begin{center}
	{ Федеральное государственное бюджетное образовательное учреждеие высшего профессионального образования \\}
	\end{center}


   \begin{minipage}{0.80\textwidth}
		\begin{large}
			\begin{center}
				{
		\textbf{<<Московский государственый технический университет \\ имени Н.Э. Баумана>> \\ (МГТУ им. Н.Э. Баумана) \\}
   				}
			\end{center}
   		\end{large}
	\end{minipage}

   \begin{center}
    \vspace{0.25cm}
	\noindent\makebox[\linewidth]{\rule{\textwidth}{0.4pt}}

    ФАКУЛЬТЕТ "Энергомашиностроение"

    КАФЕДРА Э-3 "Газотурбинные и нетрадиционные энергоустановки"
    \vspace{1cm}


    \begin{large}\textbf{КУРСОВАЯ НАУЧНО-ИССЛЕДОВАТЕЛЬСКАЯ РАБОТА СТУДЕНТА}\end{large}

	\end{center}
    \textbf{на тему:} Применение нейронных сетей для моделирования подсеточных напряжений при использовании метода больших вихрей

    \textbf{группа:} Э3-101

    \begin{small}\textbf{Выполнил(а) студент(ка)} \makebox[9cm]{\hrulefill}  \textbf{А.Клюквин} \\
    \centerline{(подпись, дата)}
    \end{small}

    \begin{small}\textbf{Руководитель} \makebox[11.8cm]{\hrulefill}   \textbf{С.Бурцев} \\
    \centerline{(подпись, дата)}
    \end{small}



\vfill

\begin{center}
  Москва - 2017 г.
\end{center}
\end{titlepage}

\input{val_list_template}
\input{input_data}
\subsection{Расчет цикла}
В данном расчете учет изменения теплофизических свойств рабочего тела в зависимости от его температуры производился
путем итерирования на каждом этапе расчета до тех пор, пока изменение искомого теплофизического свойства (теплоемкости или
показателя адиабаты) не составляло менее 0.1\% в сравнении с результатами предыдущей итерации. Ниже везде используются
значения теплофизический свойств на последнем этапе итерационных расчетов.

\begin{enumerate}
	\item Определим давление за входным устройством:
		$$p_{вх}^* = \sigma_{вх}  p_a = 0,98 \cdot 0,100 = 0,098 \/\ МПа$$
	\item Определим давление за КНД:
		$$p_{кнд}^* = \pi_{кнд} p_{вх}^* = 4,8 \cdot 0,098 = 0,466 \/\ МПа$$
	\item Определим адиабатический КПД КНД $\eta_{кнд}$, принимая показатель адиабаты воздуха $k_{в \/\ кнд} = 1,40$:
	    $$
	    	\eta_{кнд} = \frac{
		        \pi_{кнд}^\frac{
		            k_{в \/\ кнд} - 1
		        }{
		            k_{в \/\ кнд}
	            } - 1
		    }{
		        \pi_{кнд}^\frac{
		            k_{в \/\ кнд} - 1
	            }{
	                k_{в \/\ кнд} \cdot \eta_{пол \/\ кнд}
	            } - 1
		    } = \frac{
	            4,8^\frac{
	                1,40 - 1
	            }{
	                1,40
	            } - 1
	        }{
	            4,8^\frac{
	                1,40 - 1
	            }{
	                1,40 \cdot 0,840
	            } - 1
	        } = 0,80
	    $$
	\item Определим температуру газа за КНД:
		$$T_{КНД}^* = T_a 
		\left[ 
			1 + \frac{
				\pi_к^{
					\frac{
						k_{в \/\ кнд} - 1
					}{
						k_{в \/\ кнд}
					}
				} - 1
			}{
				\eta_{кнд}
			}
		\right] =
			288,0 
		\left[
			1 + \frac{
				{4,8}^{
					\frac{
						1,40 - 1
					}{
						1,40
					}
				} - 1
			}{
				0,80
			}
		\right] = 490,0 \/\ К$$
	\item Используя найденный показатель адиабаты воздуха, определим теплоемкость воздуха в процессе сжатия воздуха в КНД:
		$$c_{pв \/\ кнд} = \frac{
			k_{в \/\ кнд}
		}{
			k_{в \/\ кнд} - 1
		} R_в = \frac{
			1,40
		}{
			1,40 - 1
		} \cdot 287,0 = 1012,1 \/\ Дж/(кг \cdot К)$$
	\item Определим работу КНД:
		$$L_{КНД} = c_{pв \/\ кнд} \left( T_{кнд}^* - T_a \right) =
			1012,1 \cdot \left(490,0 - 288,0\right) =
			0,204 \cdot 10^6 \/\ Дж/кг $$
	\item Определим давление перед КВД:
		$$p_{0 \/\ квд}^* = \sigma_{кнд} p_{кнд}^* = 0,98 \cdot 0,466 = 0,456 \/\ МПа$$
	\item Определим давление за КВД:
		$$ p_{квд}^* = \pi_{квд} p_{0 \/\ квд}^* = 4,0 \cdot 0,456 = 1,825 \/\ МПа $$
	\item Определим адиабатический КПД КВД $\eta_{квд}$, принимая показатель адиабаты воздуха $k_{в \/\ КВД} = 1,37$:
	    $$
	    	\eta_{квд} = \frac{
		        \pi_{квд}^\frac{
		            k_{в \/\ квд} - 1
		        }{
		            k_{в \/\ квд}
	            } - 1
		    }{
		        \pi_{кнд}^\frac{
		            k_{в \/\ квд} - 1
	            }{
	                k_{в \/\ квд} \cdot \eta_{пол \/\ квд}
	            } - 1
		    } = \frac{
	            4,0^\frac{
	                1,37 - 1
	            }{
	                1,37
	            } - 1
	        }{
	            4,0^\frac{
	                1,37 - 1
	            }{
	                1,37 \cdot 0,820
	            } - 1
	        } = 0,78
	    $$
	\item Определим температуру газа за КВД:
		$$T_{квд}^* = T_{кнд}^*
		\left[ 
			1 + \frac{
				\pi_к^{
					\frac{
						k_в - 1
					}{
						k_в
					}
				} - 1
			}{
				\eta_{квд}
			}
		\right] =
			490,0 
		\left[
			1 + \frac{
				{4,0}^{
					\frac{
						1,37 - 1
					}{
						1,37
					}
				} - 1
			}{
				0,78
			}
		\right] = 784,3 \/\ К$$
	\item Используя найденный показатель адиабаты воздуха, определим теплоемкость воздуха в процессе сжатия воздуха в КВД:
		$$c_{pв \/\ квд} = \frac{
			k_{в \/\ квд}
		}{
			k_{в \/\ квд} - 1
		} R_в = \frac{
			1,37
		}{
			1,37 - 1
		} \cdot 287,0 = 1061,8 \/\ Дж/(кг \cdot К)$$
	\item Определим работу КВД:
		$$L_{квд} = c_{pв \/\ квд} \left( T_{квд}^* - T_{кнд}^* \right) =
			1061,8 \cdot \left(784,3 - 490,0\right) =
			0,313 \cdot 10^6 \/\ Дж/кг $$
	\item Температура газа за камерой сгорания:
		$$T_г^* = 1450 \/\ К$$
	\item Определим относительный расход топлива. Расчет носит итерационный характер. Ниже описана последняя итерация. Теплоемкость продуктов сгорания природного газа рассчитывается через показатель адиабаты и газовую постоянную газа. При этом газовая постоянная и истинный показатель адиабаты рассчитываются как средневзвешенное соответственных характеристик компонентов продуктов. При расчета приняты следующие значения:
	\begin{enumerate} % список значений для расчета удельного расхода топлива
		\item[1)] теплоемкость топлива:
			$$c_{pm} = 2226,0 \/\ Дж / (кг \cdot К);$$
		\item[2)] температура подачи топлива:
			$$T_m = 300,0 \/\ К;$$
		\item[3)] температура определения теплофизических параметров веществ:
			$$T_0 = 300,0 \/\ К;$$
		\item[4)] истинная теплоемкость воздуха перед камерой сгорания:
			$$c_{pв \/\ г}\left( T_{КВД} \right) = 1095,4 \/\ Дж/(кг \cdot К);$$
		\item[5)] истинная теплоемкость воздуха при температуре определения теплофизических параметров веществ:
			$$c_{pв \/\ г}\left( T_0 \right) = 1002,7 \/\ Дж/(кг \cdot К);$$
		\item[6)] низшая теплота сгорания топлива:
			$$Q_н^р = 49030 \cdot 10^3 \/\ Дж / кг;$$
		\item[7)] полнота сгорания:
			$$\eta_г = 0,98;$$
		\item[8)] масса воздуха, необходимая для сжигания 1 кг топлива:
			$$l_0 = 17,3 \/\ кг;$$
	\end{enumerate}
	
	\begin{enumerate}
		\item Зададимся коэффициентом избытка воздуха: $$\alpha = 3,32;$$
		\item Теплоемкость продуктов сгорания природного газа $c_{pг \/\ г}$ при данном значении коэффициента избытка воздуха при температуре $T_г$ составляет:
			$$c_{pг \/\ г}\left( T_г \right) = 1222,3 \/\ Дж/(кг \cdot К);$$
		\item Теплоемкость продуктов сгорания природного газа $c_{pг \/\ г}$ при данном значении коэффициента избытка воздуха при температуре $T_0$ составляет:
			$$c_{pг \/\ г}\left( T_0 \right) = 995,0 \/\ Дж / (кг \cdot К);$$
		\item Определим относительный расход топлива:
			$$
				a = c_{pг \/\ г} \left( T_г \right) T_г - c_{pв \/\ г} \left( T_{квд} \right) T_{квд} = 
			$$
			$$
				= 1222,3 \cdot 1450,0 -
				1222,3 \cdot 784,310 = 
				0,913 \cdot 10^6 \/\ Дж/кг
			$$
			$$
				b = \left(
					c_{pг \/\ г}\left( T_0 \right) - c_{pв \/\ г}\left( T_0 \right) = 
				\right) T_0 = 
			$$
			$$
				= \left(
					995,0 - 1002,7
				\right) \cdot 300,0 = 
				-2,327 \cdot 10^3 \/\ Дж/кг
			$$
			$$
				c = c_{pг \/\ г} \left( T_г \right) T_г - c_{pг \/\ г} \left( T_0 \right) T_0 = 
			$$
			$$
				= 1222,3 \cdot 1450,0 -
				995,0 \cdot 300,0 = 
				1,474 \cdot 10^6 \/\ Дж/кг
			$$
			$$
				d = c_{pm} \left( T_m - T_0 \right) = 
			$$
			$$
				= 2226,0 \left( 300,0 - 300,0 \right) =
				0 \/\ Дж/кг
			$$
			$$g_m = \frac{G_m}{G_в^г} =
				\frac{
					a - b
				}{
					Q_н^р \eta_г -
					c + d
				} = 
			$$
			$$
				= \frac{
					0,913 \cdot 10^6 + 2,327 \cdot 10^3
				}{
					49030 \cdot 10^3 \cdot 0.98 -
					1473,793 \cdot 10^6 + 0
				} = 0,017
			$$
		\item Определим коэффициент избытка воздуха:
			$$\alpha^\prime = \frac{1}{g_m l_0} =
		\frac{1}{0,017 \cdot 17,3} = 3,32$$
	\end{enumerate}

	\item Определим удельный расход через ТВД:
		$$g_{твд} = \left( 1 + g_m \right) \left( 1 - g_{ут \/\ твд} - g_{охл \/\ твд} \right) = $$
		$$
		= \left(
		    1 + 0,017
		\right) \left(
		    1 - 0,010 -
		    0,100
        \right) = 1,027$$
	\item Определим удельную работу ТВД:
		$$L_{твд} = \frac{L_{квд}}{g_{твд}\eta_{м \/\ вд}} = \frac{
			0,313 \cdot 10^6
		}{
			1,027 \cdot 0,990
		} = 0,280 \cdot 10^6 \/\ Дж/кг$$
	\item Определим давление газа перед ТВД:
		$$p_{г}^* = p_{тнд}^* \sigma_г = 1,825 \cdot 0,99 = 1,807 \/\ МПа$$
	\item Определим среднюю теплоемкость газа в процессе расширения газа в турбине, принимая показатель адиабаты газа $k_{г \/\ твд} = 1,32$:
		$$c_{pг \/\ твд} = \frac{k_{г \/\ твд}}{k_{г \/\ твд} - 1} R_г =
			\frac{
				1,32
			}{
				1,32 - 1
			} \cdot 291,0 = 1206,6 \/\ Дж/(кг \cdot К) $$
	\item Определим давление воздуха за ТВД:
		$$p_{твд}^* = p_г^*
			\left[
				1 - \frac{L_{твд}}{c_{pг \/\ твд} T_г \eta_{твд}}
			\right] ^ \frac{k_{г \/\ твд}}{k_{г \/\ твд} - 1} =
		$$
		$$
			= 1,807
			\left[
				1 - \frac{0,313 \cdot 10^6}
				{1206,6 \cdot 1450,0 \cdot 0,880}
			\right] ^ \frac{1,32}{1,32 - 1} =
			 0,787 \/\ МПа
		$$
	\item Определим температуру газа за ТВД:
	 	$$
	 		T_{твд}^* = T_г^*
			\left\lbrace
			 	1 -
			 	\left[
			 		1 -
			 			\left(
			 				\frac{p_{твд}^*}{p_г^*}
			 			\right) ^ \frac{k_{г \/\ твд}}{k_{г \/\ твд} - 1}
			 	\right] \eta_{ТВД}
			\right\rbrace =
		$$
		$$
			= 1450,0
			\left\lbrace
			 	1 -
			 	\left[
			 		1 -
			 			\left(
			 				\frac{0,787}{1,807}
			 			\right) ^ \frac{1,32}{1,32 - 1}
			 	\right] \cdot 0,880
			\right\rbrace = 1218,1 \/\ К
		$$
	\item Определим давление перед ТНД:
		$$p_{0 \/\ тнд}^* = p_{твд}^*\sigma_{твд} = 0,787 \cdot 0,98 = 0,771 \/\ МПа$$

	\item Определим удельный расход через ТНД:
		 $$g_{тнд} = g_{твд} \left( 1 - g_{ут \/\ тнд} - g_{охл \/\ тнд} + g_{охл \/\ твд}\right) = $$
		 $$=1,027 \cdot
		 	\left(
		 	    1 - 0,010 -
		 	    0,000 +
		 	    0,100
		 	\right) = 1,037$$
	\item Определим удельную работу ТНД:
		$$L_{тнд} = \frac{L_{кнд}}{g_{тнд}\eta_{м \/\ нд}} = \frac{
			0,204 \cdot 10^6
		}{
			1,037 \cdot 0,99
		} = 0,199 \cdot 10^6 \/\ Дж/кг$$
	\item Определим среднюю теплоемкость газа в процессе расширения газа в ТНД, принимая показатель адиабаты газа $k_{г \/\ тнд} = 1,33$:
		$$c_{pг \/\ тнд} = \frac{k_{г \/\ тнд}}{k_{г \/\ тнд} - 1} R_г =
			\frac{
				1,33
			}{
				1,33 - 1
			} \cdot 291,0 = 1178,1 \/\ Дж/(кг \cdot К) $$
	\item Определим давление воздуха за ТНД:
		$$
			p_{тнд}^* = p_{0 \/\ тнд}^*
				\left[
					1 - \frac{L_{тнд}}{c_{pг \/\ тнд} T_г \eta_{тнд}}
				\right] ^ \frac{k_{г \/\ тнд}}{k_{г \/\ тнд} - 1} =
		$$
		$$
			= 0,771
				\left[
					1 - \frac{
						0,204 \cdot 10^6
					}
					{
						1178,1 \cdot 1218,1 \cdot 0,90
					}
				\right] ^ \frac{1,33}{1,33 - 1} =
				 0,391 \/\ МПа
		$$
	\item Определим температуру газа за ТНД:
	 	$$
	 		T_{тнд}^* = T_{твд}^*
			\left\lbrace
			 	1 -
			 	\left[
			 		1 -
			 			\left(
			 				\frac{p_{тнд}^*}{p_{тнд \/\ 0}^*}
			 			\right) ^ \frac{k_{г \/\ тнд}}{k_{г \/\ тнд} - 1}
			 	\right] \eta_{тнд}
			\right\rbrace =
		$$
		$$
			= 1218,1
			\left\lbrace
			 	1 -
			 	\left[
			 		1 -
			 			\left(
			 				\frac{0,391}{0,771}
			 			\right) ^ \frac{1,33}{1,33 - 1}
			 	\right] \cdot 0,90
			\right\rbrace = 1049,1 \/\ К
		$$
	\item Определим давление перед свободной турбиной:
		$$p_{0 \/\ тс}^* = p_{тнд}^*\sigma_{тнд} = 0,391 \cdot 0,98 = 0,384 \/\ МПа$$
	\item Определим удельный расход через силовую турбину:
	    $$g_{тс} = g_{тнд} \left( 1 - g_{ут \/\ тс} - g_{охл \/\ тс} \right) =
            1,037 \cdot
            \left(
                1 - 0,010 -
                0,000
            \right) = 1,027$$
    \item Определим давление торможения на выходе из свободной турбины $p_{тс}^*$:
		$$p_{тс}^* = p_a / \sigma_{вых} = 0,100 \cdot 0,93 = 0,108 \/\ МПа$$
	\item Зададим значение приведенной скорости на выходе из свободной турбины:
		$$\lambda_{вых} = 0,30$$
	\item Определим статическое давление на выходе из свободной турбины, принимая показатель адиабаты газа на выходе из свободной турбины $k_{тс \/\ вых} = 1,36$:
		$$p_{тс} = p_{тс}^* \cdot \pi \left( \lambda_{вых}, \/\ k_{тс \/\ вых} \right)
        =
			0,108
			\cdot \pi \left( 0,30, \/\ 1,36 \right)
        = 0,102 \/\ МПа$$
	\item Определим статическую температуру на выходе из свободной турбины, принимая показатель адиабаты газа $k_{г \/\ тс} = 1,35$::
		$$
			T_{тс} = T_{тнд}^*
			\left\lbrace
			 	1 -
			 	\left[
			 		1 -
			 			\left(
			 				\frac{p_{0 \/\ тс}^*}{p_{тс}}
			 			\right) ^ \frac{k_{г \/\ тс}}{k_{г \/\ тс} - 1}
			 	\right] \eta_{тс}
			\right\rbrace =
		$$
		$$
			= 1049,1
			\left\lbrace
			 	1 -
			 	\left[
			 		1 -
			 			\left(
			 				\frac{
			 					0,384
			 				}{
			 					0,102
			 				}
			 			\right) ^ \frac{1,35}{1,35 - 1}
			 	\right] \cdot 0,92
			\right\rbrace = 769,6 \/\ К
		$$
	\item Определим температуру торможения на выходе из силовой турбины:
		$$T_{тс}^* = 
			\frac{T_{тс}}{\tau\left( \lambda_{вых}, \/\ k_{тс \/\ вых} \right)} =
			\frac{T_{тс}}{\tau\left( 0,30, \/\ 1,36 \right)} =
			= 780,2 \/\ К$$
	\item Определим значение теплоемкости газа в свободной турбине:
		$$c_{p \/\ тс} = 
			\frac{k_{г \/\ тс}}{k_{г \/\ тс} - 1} = 
			\frac{1,35}{1,35 - 1} = 1133,2 \/\ Дж / \left( кг \cdot К \right)$$
	\item Определим удельную работу силовой турбины:
		$$L_{тс} = c_{p \/\ тс} \left( T_{тнд}^* - T_{тс}^* \right) = 
			1133,2 \cdot \left( 1049,1 - 780,2 \right) =
			0,305 \cdot 10^6\/\ Дж/кг$$
	\item Определим удельную работу ГТД:
		$$L = L_{тс} \/\ g_{тс} =
			0,30 \cdot 10^6 \cdot 1,027 =
			0,305 \cdot 10^6 Дж/кг$$
	\item Определим экономичность ГТД:
		$$C_e = \frac{3600}{N_{e уд}} g_{тс} =
			\frac{3600}{0,305 \cdot 10^6} \cdot 1,03 =
			0,192 \cdot 10^{-3} кг/\left( кВт/ч \right)$$
	\item Определим КПД ГТД:
		$$\eta_e = \frac{3600}{C_e Q_н^р} =
			\frac{3600}{0,192 \cdot 10^{-3} \cdot 49,030 }
			= 0,382$$
	\item Определим потребную мощность ГТД:
		$$
			N = N_e / \eta_р = 16000 \cdot 10^3 \cdot \ 0,98 = 16327 \cdot 10^3 \/\ Вт
		$$
	\item Определим расход воздуха:
		$$G_в = \frac{N}{L} =
			\frac{16327 \cdot 10^3}{0,305 \cdot 10^6} =
			51,1 \/\ кг/с$$
\end{enumerate}
\section{Поступенчатый расчет турбины}
Для данного проекта выбрана одноступенчатая турбина.
Зададим параметры, необходимые для поступенчатого расчета турбины:

% \begin{center}
% 	\begin{tabular}{|p{7cm}|c|c|c|}
% 		\hline
% 		\textbf{Величина} & \textbf{Обозначение} & \textbf{Размерность} & \textbf{Значение} \\ \hline
% 			Реактивность ступени & $\rho$ & - & \\ \hline
% 			Радиальный зазор & $\delta_r$ & - & $<DeltaR>$ \\ \hline
% 			Относительная длина лопатки статора & $\left( \frac{l}{D} \right)_1$ & - & $<StatorLRel>$ \\ \hline
% 			Удлинение лопатки статора & $\left( \frac{l}{b_a} \right)_{СА}$ & - & $<StatorElongation>$ \\ \hline
% 			Удлинение лопатки ротора & $\left( \frac{l}{b_a} \right)_{РК}$ & - & $<RotorElongation>$ \\ \hline
% 			Относительная ширина зазора между лопатками ротора и лопатками статора & $\left( \frac{\delta}{b_a} \right)_{СА}$ & - & $<StatorDeltaRel>$ \\ \hline
% 			Угол раскрытия на втулке & $\gamma_{вт}$ & \degree & $<GammaIn>$ \\ \hline
% 			Угол раскрытия на периферии & $\gamma_{пер}$ & \degree & $<GammaOut>$ \\ \hline

% 	\end{tabular}
% \end{center}

\begin{enumerate}
	\item Определим теплоперепад на сопловом аппарате:
		$$H_с = \left( 1 - \rho \right) H_т =
		\left( 
			1 - <-<.Reactivity | Round1>-> 
		\right) \cdot <-<.Ht | DivideE6 | Round3>-> \cdot 10^6 = 
			<-<.Hs | DivideE6 | Round3>-> \cdot 10^6 \/\ Дж/кг$$
	\item Определим скорость адиабатного истечения из СА:
		$$c_{1 ад} = \sqrt{2 H_с} = 
			\sqrt{2 \cdot <-<.Hs | DivideE6 | Round3>-> \cdot 10^6} = <-<.C1Ad | Round1>-> \/\ м/с$$
	\item Определим скорость действительного истечения из СА:
		$$c_1 = \phi c_{1 ад} =
			<-<.Phi | Round2>-> \cdot <-<.C1Ad | Round1>-> = <-<.InletTriangle.C | Round1>-> \/\ м/с$$
	\item Определим температуру на выходе из СА:
		$$T_1 = T_г - \frac{c_1^2}{2c_{pг}} =
			<-<.Tg>-> - 
			\frac{
				{<-<.InletTriangle.C | Round1>->}^2
			}{
				2 \cdot <-<.StatorGas.CpMean | Round1>->
			} = <-<.T1 | Round1>-> \/\ К$$
	\item Определим температуру конца адиабатного расширения:
		$$T_1^\prime = T_г - \frac{H_c}{c_{pг}} =
			<-<.Tg | Round1>-> - 
			\frac{
				<-<.Hs | DivideE6 | Round3>-> \cdot 10^6
			}{
				<-<.StatorGas.CpMean | Round1>->
			} = <-<.T1Prime | Round1>-> \/\ К$$
	\item Определим давление на выходе из СА:
		$$p_1 = p_г \left( \frac{T_1^\prime}{T_г} \right)^\frac{k_г}{k_г - 1} =
			<-<.PStagIn | DivideE6 | Round3>-> \cdot \left(
				 \frac{
				 	<-<.T1Prime | Round1>->
				 }{
				 	<-<.Tg | Round1>->
				 } 
			\right)^\frac{
				<-<.StatorGas.KMean | Round2>->
			}{
				<-<.StatorGas.KMean | Round2>-> - 1
			} = <-<.P1 | DivideE6 | Round3>-> \/\ МПа$$
	\item Определим плотность газа на выходе из СА:
		$$\rho_1 = \frac{p_1}{R_г T_1} =
			\frac{
				<-<.P1 | DivideE6 | Round3>-> \cdot 10^6
			}{
				<-<.StatorGas.R | Round1>-> \cdot <-<.T1 | Round1>->
			} = <-<.Rho1 | Round2>-> \/\ кг/м^3$$
	\item Зададим угол на выходе из СА:
		$$\alpha_1 = <-<.InletTriangle.Alpha | Degree | Round1>-> \degree$$
	\item Определим осевую скорость на выходе из СА:
		$$c_{1a} = c_1 \cdot \sin \alpha_1 =
			<-<.InletTriangle.C | Round1>-> \cdot 
			\sin<-<.InletTriangle.Alpha | Degree | Round1>->\degree 
			= <-<.InletTriangle.CA | Round1>-> \/\ м/с$$
	\item Определим площадь на выходе из СА:
		$$A_1 = \frac{G}{c_{1a} \rho_1} =
			\frac{
				<-<.MassRate | Round1>->
			}{
				<-<.InletTriangle.CA | Round1>-> \cdot <-<.Rho1 | Round2>->
			} = <-<.StatorGeom.AreaOut | Round2>-> \/\ м^2$$
	\item Определим средний диаметр турбины на выходе из СА:
	$$D_1 = \sqrt{
		\frac{A_1}{\pi \left( \frac{l}{D} \right)_1}
		} = \sqrt{
			\frac{
				<-<.StatorGeom.AreaOut | Round2>->
			}{
				\pi \cdot <-<.StatorGeom.LRelOut | Round2>->
			}
		} = <-<.StatorGeom.DMeanOut | Round3>-> \/\ м $$
	\item Определим окружную скорость на среднем диаметре на входе в РК:
		$$u_1 = \frac{\pi D_1 n}{60} = 
			\frac{
				\pi \cdot <-<.RotorGeom.DMeanIn | Round3>-> \cdot <-<.RPM | Round1>->
			}{60} = <-<.InletTriangle.U | Round1>-> \/\ м/с$$
	\item Определим относительную скорость на входе в РК:
		$$w_1 = \sqrt{c_1^2 + u_1^2 - 2 c_1 u_1 \cos \alpha_1} =
			\sqrt{
				{<-<.InletTriangle.C | Round1>->}^2 + 
				{<-<.InletTriangle.U | Round1>->}^2 - 
				2 \cdot <-<.InletTriangle.C | Round1>-> \cdot <-<.InletTriangle.U | Round1>-> \cdot 
				\cos <-<.InletTriangle.Alpha | Degree | Round1>-> \degree
			} = <-<.InletTriangle.W>-> \/\ м/с$$
	\item Определим температуру торможения в относительном движении на входе в РК:
		$$T_{w1} = T_1 + \frac{w_1^2}{2c_{p г}} = 
			<-<.T1 | Round1>-> + 
			\frac{
				<-<.InletTriangle.W | Round1>->^2
			}{
				2 \cdot <-<.RotorGas.CpMean | Round1>->
			} = <-<.Tw1 | Round1>-> \/\ К$$
	\item Определим давление торможения в относительном движении на входе в РК:
		$$p_{w1} = p_1 \left( \frac{T_{w1}}{T_1} \right)^\frac{k_г}{k_г - 1} =
	 		<-<.P1 | DivideE6 | Round3>-> \cdot \left( 
	 			\frac{
	 				<-<.Tw1 | Round1>->
	 			}{
	 				<-<.T1 | Round1>->
	 			} 
	 		\right)^\frac{
	 			<-<.RotorGas.KMean | Round2>->
	 		}{
	 			<-<.RotorGas.KMean | Round2>-> - 1
	 		} = <-<.Pw1 | DivideE6 | Round3>-> \/\ МПа$$
	 \item Определим теплоперепад на РК:
	 	$$H_л = H_т \rho \frac{T_1}{T_1^\prime} =
	 		<-<.Ht | DivideE6 | Round3>-> \cdot 10^6 \cdot <-<.Reactivity | Round1>-> \cdot \frac{
	 			<-<.T1 | Round1>->
	 		}{
	 			<-<.T1Prime | Round1>->
	 		} = <-<.Hr | DivideE6 | Round3>-> \cdot 10^6 \/\ Дж/кг$$

	\item Определим расстояние в осевом направлении между выходными кромками лопаток СА и выходными кромками лопаток РК:
		$$x = \frac{
		 	\frac{\delta_a}{ \left( \frac{l}{b_a} \right)_1 }	+
		 	\frac{1}{\left( \frac{l}{b_a} \right)_2 }
		}{
		 	1 - \frac{\tan \gamma_п + \tan \gamma_в}
		 	{2 \left( \frac{l}{b_a} \right)_2}
		} D_1 \left( \frac{l}{D} \right)_1 =
		\frac{
		 	\frac{
		 		<-<.StatorGeom.DeltaRel | Round2>->
		 	}{
		 		<-<.StatorGeom.Elongation | Round2>->
		 	}	+
		 	\frac{
		 		1
		 	}{
		 		<-<.RotorGeom.Elongation | Round2>->
		 	} 
		}{
			1 - \frac{
				\tan <-<.StatorGeom.GammaOut | Degree | Round1>-> + \tan <-<.StatorGeom.GammaIn | Degree | Round1>->
			}{
				2 \cdot <-<.RotorGeom.Elongation | Round2>->
			}
		} \cdot <-<.StatorGeom.DMeanOut | Round3>-> \cdot <-<.StatorGeom.LRelOut | Round2>-> = 
			<-<.X | Round3>-> \/\ м
		$$
	 \item Определим средний диаметра на выходе из РК:
		 $$D_2 = D_1 + \frac{\tan \gamma_1 - \tan \gamma_2}{2} x =
	   		<-<.StatorGeom.DMeanOut | Round3>-> + 
	   		\frac{
	   			\tan <-<.StatorGeom.GammaIn | Degree | Round1>-> \degree - 
	   			\tan <-<.StatorGeom.GammaOut | Degree | Round1>-> \degree
	   		}{2} \cdot <-<.X | Round3>-> =
   		<-<.RotorGeom.DMeanOut | Round3>-> \/\ м$$
	 \item Определим длину лопатки на выходе из РК:
		 $$l_2 = 
		 	D_1 \left( \frac{l}{D} \right)_1 + 
		 	\frac{\tan \gamma_1 + \tan \gamma_2}{2} x =
		 	<-<.StatorGeom.DMeanOut | Round3>-> \cdot 
		 	<-<.StatorGeom.Elongation | Round2>-> + 
		 	\frac{
		 		\tan <-<.StatorGeom.GammaIn | Degree | Round1>-> \degree + 
		 		\tan <-<.StatorGeom.GammaOut | Degree | Round1>-> \degree
		 	}{2} \cdot <-<.X | Round3>-> =
		 		<-<.RotorGeom.LOut | Round3>-> \/\ м$$
	 \item Определим относительную длину лопаток на выходе из РК:
		 $$\left( \frac{l}{D} \right)_2 = \frac{l_2}{D_2} = 
		 	\frac{
		 		<-<.RotorGeom.LOut | Round3>->
		 	}{
		 		<-<.RotorGeom.DMeanOut | Round3>->
		 	} = <-<.RotorGeom.LRelOut | Round2>->$$

	 \item Определим окружную скорость на среднем диаметре на выходе из РК:
		 $$u_2 = \frac{\pi D_2 n}{60} = 
		 	\frac{
		 		\pi 
		 		\cdot <-<.RotorGeom.DMeanOut | Round3>-> 
		 		\cdot <-<.RPM | Round1>->
		 	}{60} = <-<.OutletTriangle.U | Round1>-> \/\ м/с$$
	 \item Определим адиабатическую относительную скорость истечения газа из РК:
	 	$$w_{2 ад} = \sqrt{w_1^2 + 2H_л + \left( u_2^2 - u_1^2 \right)} =
	 		\sqrt{
	 			{<-<.InletTriangle.W | Round1>->}^2 + 
	 			2 \cdot <-<.Hr | DivideE6 | Round3>-> \cdot 10^6 + 
	 			\left( {<-<.OutletTriangle.U | Round1>->}^2 - {<-<.InletTriangle.U | Round1>->}^2 \right)
	 		} = <-<.W2Ad | Round1>-> \/\ м/с$$
	 \item Определим относительную скорость истечения газа из РК:
	 	$$w_2 = \psi w_{2 ад} =
	 		<-<.Psi | Round2>-> \cdot <-<.W2Ad | Round1>-> = 
	 		<-<.OutletTriangle.W | Round1>-> \/\ м/с$$
	 \item Определим статическую температуру на выходе из РК:
		 $$T_2 = T_1 + \frac{
		 	\left(
		 		w_1^2  - w_2^2
		 	\right) + \left(
		 		u_2^2 - u_1^2
		 	\right)
		 }{2 c_{p г}} =
		 <-<.T1 | Round1>-> + \frac{
		 	\left(
		 		{<-<.InletTriangle.W | Round1>->}^2  - {<-<.OutletTriangle.W | Round1>->}^2 
		 	\right) + 
		 	\left( 
		 		{<-<.InletTriangle.U | Round1>->}^2  - {<-<.OutletTriangle.U | Round1>->}^2
		 	\right)
		 }{2 \cdot <-<.RotorGas.CpMean | Round1>->} = 
		 <-<.T2 | Round1>-> \/\ К$$
	 \item Определим статическую температуру при адиабатическом процессе в РК:
		 $$T_2^\prime = T_1 + \frac{
		 	\left(
		 		w_1^2  - w_{2 ад}^2
		 	\right) + 
		 	\left(
		 		u_2^2 - u_1^2
		 	\right)
		 }{2 c_{p г}} =
		 <-<.T1 | Round1>-> + \frac{
		 	\left(
		 		{<-<.InletTriangle.W | Round1>->}^2  - {<-<.W2Ad | Round1>->}^2 
		 	\right) + 
		 	\left( 
		 		{<-<.InletTriangle.U | Round1>->}^2  - {<-<.OutletTriangle.U | Round1>->}^2
		 	\right)
		 }{2 \cdot <-<.RotorGas.CpMean | Round1>->} = 
		 <-<.T2Prime | Round1>-> \/\ К$$
	 \item Определим давление на выходе из РК:
	 	$$p_2 = p_1 
	 		\left( 
	 			\frac{
	 				T_2^\prime
	 			}{
	 				T_1
	 			} 
	 		\right)^{
	 			\frac{
	 				k_г
	 			}{
	 				k_г - 1
	 			}
	 		} =
	 		<-<.P1 | DivideE6 | Round3>-> 
	 		\left( 
	 			\frac{
	 				<-<.T2Prime | Round1>->
	 			}{
	 				<-<.T1 | Round1>->
	 			} 
	 		\right)^{
	 			\frac{
	 				<-<.RotorGas.KMean | Round2>->
	 			}{
	 				<-<.RotorGas.KMean | Round2>-> - 1
	 			}
	 		} = <-<.P2 | DivideE6 | Round1>-> \/\ МПа$$
	 \item Определим угол в относительном движении на выходе из РК:
	 	$$\beta_2 = \arcsin\frac{c_{2a}}{w_2} = 
	 	\arcsin\frac{
	 		<-<.OutletTriangle.CA | Round1>->
	 	}{
	 		<-<.OutletTriangle.W | Round1>->
	 	} = <-<.OutletTriangle.Beta | Degree | Round1>-> \degree$$
	 \item Определим угол выхода из РК в абсолютном движении:
	 	$$\alpha_2 = \arctan\frac{w_2 \cos \beta_2 - u_2}{c_{2a}} =
	 	\arctan\frac{
	 		<-<.OutletTriangle.W | Round1>-> \cdot 
	 		\cos <-<.OutletTriangle.Beta | Degree | Round1>-> \degree - 
	 		<-<.OutletTriangle.U | Round1>->
	 	}{
	 		<-<.OutletTriangle.CA | Round1>->
	 	} = <-<.OutletTriangle.Alpha | Degree | Round1>-> \degree$$
	 \item Определим окружную составляющую скорости на выходе из РК:
	 	$$c_{2u} = w_2 \cos \beta_2 - u_2 =
		 	<-<.OutletTriangle.W | Round1>-> \cdot 
		 	\cos <-<.OutletTriangle.Beta | Degree | Round1>-> \degree - 
		 	<-<.OutletTriangle.U | Round1>-> = 
		 	<-<.OutletTriangle.CU | Round1>-> \/\ м/с$$
	 \item Определим скорость потока на выходе из РК:
	 	$$c_2 = \sqrt{c_{2u}^2 + c_{2a}^2} = 
	 		\sqrt{
	 			{<-<.OutletTriangle.CU | Round1>->}^2 + {<-<.OutletTriangle.CA | Round1>->}^2
	 		} = <-<.OutletTriangle.C | Round1>-> \/\ м/с$$
	 \item Определим степень понижения давления в турбине:
	 	$$\pi_{т} = \frac{p_г}{p_2} = 
	 		\frac{
	 			<-<.PStagIn | DivideE6 | Round3>->
	 		}{
	 			<-<.P2 | DivideE6 | Round3>->
	 		} = <-<.Pi | Round2>-> $$
	 \item Определим осевую составляющую скорости газа за турбиной:
	 	$$c_{2a} = c_2 \sin \alpha_2 = 
	 		<-<.OutletTriangle.C | Round1>-> 
	 		\sin <-<.OutletTriangle.Alpha | Degree | Round1>-> = 
	 		<-<.OutletTriangle.CA | Round1>-> \/\ м/с$$
	 \item Опрдеелим плотность газа за турбиной:
	 	$$\rho_2 = \frac{G}{c_2a \pi D_2 l_2} = 
	 	\frac{
	 		<-<.MassRate | Round1>->
	 	}{
	 		<-<.OutletTriangle.CA | Round1>-> \cdot 
	 		\pi \cdot 
	 		<-<.RotorGeom.DMeanOut | Round3>-> \cdot 
	 		<-<.RotorGeom.LOut | Round3>->
	 	} = <-<.Rho2 | Round2>-> \/\ кг/м^3$$
	 \item Определим работу на окружности колеса:
	 $$L_u = c_{1u} u_1 + c_{2u} u_2 = 
	 	<-<.InletTriangle.CA | Round1>-> \cdot <-<.InletTriangle.U | Round1>-> + 
	 	<-<.OutletTriangle.CA | Round1>-> \cdot <-<.OutletTriangle.U | Round1>-> = 
	 	<-<.Lu | DivideE6 | Round3>-> \cdot 10^6 \/\ Дж/кг$$
	 \item Определим КПД на окружности колеса:
	 	$$\eta_u = \frac{L_u}{H_t} = 
	 		\frac{
	 			<-<.Lu | DivideE6 | Round3>-> \cdot 10^6
	 		}{
	 			<-<.Ht | DivideE6 | Round3>-> \cdot 10^6
	 		} = <-<.EtaU | Round2>-> $$
	 \item Определим удельные потери на статоре:
		 $$h_c = \left( \frac{1}{\phi^2} - 1 \right) \frac{c_1^2}{2} =
		 \left( 
		 	\frac{1}{
		 		{<-<.Phi | Round2>-> }^2} - 1 
		 	\right) \frac{
		 		{<-<.InletTriangle.C | Round1>->}^2
		 	}{2} = <-<.LossStator | DivideE3 | Round2>-> \cdot 10^3 \/\ Дж/кг$$
	 \item Определим удельные потери на роторе:
	 	$$h_р = 
	 		\left( 
	 			\frac{1}{\psi^2} - 1 
	 		\right) \frac{w_2^2}{2} =
	 		\left( 
	 			\frac{1}{{<-<.Psi | Round2>->}^2} - 1 
	 		\right) \frac{
	 			{<-<.OutletTriangle.W | Round1>->}^2
	 		}{2} = <-<.LossRotor | DivideE3 | Round2>-> \cdot 10^3 \/\ Дж/кг$$
	 \item Определим удельные потери с выходной скоростью:
	 	$$h_{вых} = \frac{c_2^2}{2}= 
	 		\frac{
	 			{<-<.OutletTriangle.C | Round1>->}^2
	 		}{2} = <-<.LossOutflow | DivideE3 | Round2>-> \cdot 10^3 \/\ Дж/кг$$
	 \item Определим удельные потери в радиальном зазоре:
	 	$$h_з = 1.37 \cdot \left( 1 + 1.6 \rho \right)
	 	\left[ 
	 		1 + 
	 		\left( 
	 			\frac{l}{D} 
	 		\right)_1 
	 	\right] \frac{
	 		\delta_r
	 	}{
	 		l_2
	 	} L_u = $$
	 $$ = 1.37 \cdot 
	 	\left( 
	 		1 + 1.6 \cdot <-<.Reactivity | Round1>-> 
	 	\right)
	 	\left[ 
	 		1 + <-<.RotorGeom.LRelOut | Round2>-> 
	 	\right] \frac{
	 		<-<.DeltaR | Round2>->
	 	}{
	 		<-<.RotorGeom.LOut | Round3>->
	 	} \cdot <-<.Lu | DivideE3 | Round>-> \cdot 10^3 =
	 	<-<.LossRadial | DivideE3 | Round2>-> \cdot 10^3 \/\ Дж/кг$$
	 \item Определим удельные потери на вентиляцию:
	 	$$h_{вент} = 1.07 D_2^2 \left( \frac{u_2}{100} \right)^3 \rho_2 \cdot 1000 =
	 		1.07 \cdot {<-<.RotorGeom.DMeanOut | Round3>->}^2 
	 			\left( 
		 			\frac{
		 				<-<.OutletTriangle.U | Round1>->
		 			}{
		 				100
		 			} 
	 			\right)^3 
	 			\cdot <-<.Rho2 | Round2>-> 
	 			\cdot 1000 = <-<.LossVent | DivideE3 | Round2>-> \cdot 10^3 \/\ Дж/кг$$
	 \item Определим температуру торможения за РК:
	 	$$T_2^* = T_2 + \frac{h_з + h_{вент} + h_{вых}}{c_{pг}} =
	 	<-<.T2 | Round1>-> + 
	 	\frac{
	 		<-<.LossRadial | DivideE3 | Round2>-> \cdot 10^3 + 
	 		<-<.LossVent | DivideE3 | Round2>-> \cdot 10^3 + 
	 		<-<.LossOutflow | DivideE3 | Round2>-> \cdot 10^3
	 	}{
	 		<-<.RotorGas.CpMean | Round1>->
	 	} = <-<.T2Stag | Round1>-> \/\ К$$
	 \item Определим давление торможения за РК:
	 	$$p_2^* = p_2 
	 		\left( 
	 			\frac{
	 				T_2^*
	 			}{
	 				T_2
	 			} 
	 		\right)^{
	 			\frac{
	 				k_г
	 			}{
	 				k_г - 1
	 			}
	 		} =
	 	<-<.P2 | DivideE6 | Round3>-> \cdot 
	 		\left( 
	 			\frac{
	 				<-<.T2Stag | Round1>->
	 			}{
	 				<-<.T2 | Round1>->
	 			} 
	 		\right)^{
	 			\frac{
	 				<-<.RotorGas.KMean | Round2>->
	 			}{
	 				<-<.RotorGas.KMean | Round2>-> - 1
	 			}
	 		} = <-<.PStagOut | DivideE6 | Round3>-> \/\ МПа$$
	 \item Определим мощностной КПД турбины:
	 	$$\eta_{т \/\ мощн} = 
	 		\eta_u - 
	 		\frac{
	 			h_з + h_{вент}
	 		}{
	 			H_т
	 		} =
	 		<-<.EtaU | Round2>-> - 
	 		\frac{
	 			<-<.LossRadial | DivideE3 | Round2>-> \cdot 10^3 + <-<.LossVent | DivideE3 | Round2>-> \cdot 10^3
	 		}{
	 			<-<.Ht | DivideE6 | Round3>-> \cdot 10^6
	 		} = <-<.EtaPower | Round2>->$$
	 \item Определим работу турбины:
	 	$$L_т = H_т \eta_т = 
	 		<-<.Ht | DivideE6 | Round3>-> \cdot 10^6 \cdot 
	 		<-<.EtaPower | Round2>-> = 
	 		<-<.Lt | DivideE6 | Round3>-> \cdot 10^6 \/\ Дж/кг$$
	 \item Определим теплоперепад по параметрам торможения:
	 	$$H_т^* = c_{pг} T_г 
	 		\left[ 
	 			1 - 
	 				\left( 
	 					\frac{
	 						p_2^*
	 					}{
	 						p_г^*
	 					} 
	 				\right)^\frac{
	 					k_г - 1
	 				}{
	 					k_г
	 				} 
	 		\right] =
	 	<-<.Gas.CpMean | Round1>-> \cdot <-<.Tg | Round1>-> 
	 		\left[ 1 - 
	 			\left( 
	 				\frac{
	 					<-<.PStagOut | DivideE6 | Round3>->
	 				}{
	 					<-<.PStagIn | DivideE6 | Round3>->
	 				} 
	 			\right)^\frac{
	 				<-<.RotorGas.KMean | Round2>-> - 1
	 			}{
	 				<-<.RotorGas.KMean | Round2>->
	 			} 
	 		\right] = <-<.HtStag | DivideE6 | Round3>-> \cdot 10^6 \/\ Дж/кг $$
	 \item Определим КПД турбины по параметрам торможения:
	 $$\eta_т = \frac{L_т}{H_т^*} =
	 	\frac{
	 		<-<.Lt | DivideE6 | Round3>-> \cdot 10^6
	 	}{
	 		<-<.HtStag | DivideE6 | Round3>-> \cdot 10^6
	 	} = <-<.EtaTStag | Round2>->$$

\end{enumerate}
\subsection{Профилирование ступени ТВД}
Исходными данными для данного этапа проектирования турбины являются результы расчета по средней линии тока.

Ступень была спрофилирована по закону.

Определим треугольники скоростей на произвольном радиусе лопатки.

\begin{enumerate}

	\item В этом случае значения абсолютной скорости на входе на рабочие лопатки на произвольном радиусе определялись по следующим формулам (в приведенных ниже формулах значения со штрихом относятся к среднему радиусу):
	\item Окружная скорость рабочей лопатки на произвольном радиусе была определена по закону вращения твердого тела:
	\item Относительная скорость на произвольном радиусе на входе в рабочие лопатки была определена по следующим формулам:
	\item Абсолютная скорость на выходе из рабочих лопаток была определена по условию постоянства работы, отводимой от газа на различных радиусах лопатки.

	По формуле Эйлера для правила отсчета углов, принятого в теории турбин удельная работа на окружности колеса определяется слеюущей формулой:
	Таким образом, зная работу на окружности колеса на среднем радиусе лопатки, мы можем определить значение окружной скорости на выходе из рабочих лопаток:
	\item Используя значения окружной и осевой скорости на среднем радиусе лопатки, определим значение осевой скорости на выходе из рабочих лопаток, проинтегрировав уравнение радиального равновесия:
	\item Значения проекций относительной скорости на выходе из лопаток находим так же, как и значения на входе в рабочие лопатки.

\end{enumerate}
Распределение углов на входе в рабочие лопатки турбины и на выходе из них представлено на рис.~\ref{img:profile_inlet_angles}
и~\ref{img:profile_outlet_angles}, соответственно:
	
\input{cooling_calc}
\subsection{Расчет профиля температур}

Для расчета профиля температур лопатки принимаем расход воздуха, а величину зазора между дефлектором и
внутренней поверхностью лопатки.

При расчете профиля температур лопатки при конвективно-пленочно охлаждении будем пользоваться следующей методикой:
\begin{enumerate}
	\item Зададим распределение приведенной скорости по корыту и спинке:
		где - длина профиля со стороны корыта, - длина профиля со стороны спинки - приведенная скорость на входе в лопаточный венец,  - приведенная скорость на выходе из лопаточного венца.

	\item Определим критическую скорость звука:
	\item Определим скорость газа на корыте и на спинке:
	Дальнейший расчет идентичен для спинки и корыта, поэтому скорость газа будем обозначать как.
	\item Определим эквивалентную ширину щели:
		где - количество отверстий, - диаметр отверстия, - высота профильной части лопатки.
	\item Определим скорость газа в точке выдува воздуха:
		где - криволинейная координата отверстия.
	\item Определим статическую температуру газа в точке выдува воздуха:
	\item Определим статическое давление газа в точке выдува воздуха:
	\item Определим статическую плотность газа в точке выдува воздуха:
	\item Определим скорость истечения воздуха из отверстия:
	 	где - коэффициент скорости - температура воздуха в точке выдува, - давление воздуха.
	\item Определим статическую плотность воздуха на выходе из отверстия:
	\item Определим плотность торможения воздуха на входе в отверстия:
	\item Определим параметр вдува:
	\item Определим число Рейнольдса по ширине щели:
	\item Определим температурный фактор:
	\item Определим эффективность пленки:
	\item Определим темперутуру пленки в случае нескольких рядов отверстий:
	\item Определим коэффициент теплоотдачи пленки в случае нескольких рядов отверстий:
	\item По формуле истечения из сопла определим расход через ряд отверстий:
	\item В общем случае зависимость расхода воздуха в зазоре от криволинейной координаты имеет вид:

В данном расчете суммарный расход на охлаждение сопловых лопаток принимается равным
на лопатку, что при числе лопаток статора, равном 54, равно 4.89\% от суммарного расхода
воздуха.
В результате расчетов получим значения характерных параметров в отверстиях.

Значения характерных параметров в отверстиях корыта представлены в табл.~\ref{cool2:ps_hole_parameters}.

Значения характерных параметров в отверстиях спинки представлены в табл.~\ref{cool2:ps_hole_parameters}.


	\item Определим коэффициент теплоотдачи от газа на входной кромке лопатки:
	\item Определим коэффициент теплоотдачи на спинке на расстоянии:
	\item Определим коэффициент теплоотдачи на остальной выпуклой части (спинке):
	\item Определим коэффициет теплоотдачи на вогнутой части профиля (корыте):
	\item Коэффициент теплоотдачи от стенки к охлаждающему воздуху зависит от его температуры и определяется следующим уравнением:
	\item Уравнение теплообмена между охлаждающим воздухом и газом имеет вид:
	где - коэффициент теплопередачи, определяемый уравнением
	\item Уравнение теплового баланса малого элемента стенки лопатки
	где, К - температура материала лопатки,
	Вт/м - теплопроводность материала лопатки,
	, м - толщина стенки,
	, - коэффициент теплоотдачи пленки пленки газа снаружи лопатки,
	, T_{ст} - коэффициент теплоотдачи воздуха внутри лопатки,
	, К - температура пленки,
	, К - температура охлаждаюущего воздуха.

	Численно решая уравнение теплообмена, получим распределение параметров по спинке и корыту.
	Распределение параметров газа по спинке представлено в табл.~\ref{cool2:ss_gas_parameters}.
		
	Распределение параметров газа по корыту представлено в табл.~\ref{cool2:ps_gas_parameters}.
\end{enumerate}

Распределение температуры газа, воздуха и металла по профилю лопатки при исходном варианте установки
(без дожимающего компрессора) показано на рис.~\ref{img:cool_gas_parameters_no_front}.

Вариант с выдувом в лобовой точке (с дожимающим компрессором) показан на рис.~\ref{img:cool_gas_parameters_front}.

Таким образом, из расчета следует, что ни в одной точке температура материала лопатки не превышает 1000 К (максимальная температура
равна 998 К), что обеспечивает достаточную прочность лопаток
~\cite{js_36_properties}.

Из сравнения полученных распределений можно заметить, что выдув в лобовой точке приводит к существенному уменьшнию
неравномерности температуры материала сопловой лопатки (с 256,7 до 141,4 К), что приводит к увеличению ресурса горячей
части, так как коррозия и термические напряжения являются основными причинами разрушения сопловых лопаток турбины.

\end{document}
