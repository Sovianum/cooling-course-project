\section{Вариантные расчеты}
Для определения оптимальных степеней повышения давления в компрессорах
построим графики зависимости КПД, удельной мощности и расхода через компрессоры от суммарной степени повышения давления в компрессорах.
При этом для наглядности отнесем абсолютные значения рассматриваемых величин к максимальному значению,
достигающемуся на заданном промежутке.

Ниже представлен график зависимостей КПД,
мощности и расхода ГТА от суммарной степени повышения давления в компрессорах. Распределение степеней повышения
давления между компрессорами соответствует оптимальному по КПД:
\begin{center}
	\includegraphics[scale=0.7]{cycle_eta_plot}
\end{center}

Экстремум по КПД достигается при следующих значения функций:
\begin{center}
	\begin{tabular}{|c|c|c|c|c|}
	\hline
		$G, кг/с$ & $N_e, Вт/кг$ & $\eta_e$ & $\pi_{кнд}$ & $\pi_{квд}$ \\ \hline
		$<-<.MaxEta.MassRate | Round1>->$ &
		$<-<.MaxEta.SpecificPower | DivideE6 | Round3>-> \cdot 10^6$ &
		$<-<.MaxEta.Efficiency | Round3>->$ &
		$<-<.MaxEta.PiLow | Round1>->$ &
		$<-<.MaxEta.PiHigh | Round1>->$ \\ \hline
	\end{tabular}
\end{center}

Экстремум по удельной мощности достигается при следующих значениях функций:
\begin{center}
	\begin{tabular}{|c|c|c|c|c|}
	\hline
		$G, кг/с$ & $N_e, Вт/кг$ & $\eta_e$ & $\pi_{кнд}$ & $\pi_{квд}$ \\ \hline
		$<-<.MaxLabour.MassRate | Round1>->$ &
		$<-<.MaxLabour.SpecificPower | DivideE6 | Round3>-> \cdot 10^6$ &
		$<-<.MaxLabour.Efficiency | Round3>->$ &
		$<-<.MaxLabour.PiLow | Round1>->$ &
		$<-<.MaxLabour.PiHigh | Round1>->$ \\ \hline
	\end{tabular}
\end{center}

Поскольку большую часть времени ГПА работает на режиме частичной мощности, принимается $\pi_{\sum} = <-<.PiTotal | Round1>->$, $\pi_{кнд} = <-<.PiLow | Round1>->$, $\pi_{квд} = <-<.PiHigh | Round1>->$.

Ниже представлен расчет цикла ГТА при $\pi_{кнд} = <-<.PiLow | Round1>->$, $\pi_{квд} = <-<.PiHigh | Round1>->$.
\clearpage